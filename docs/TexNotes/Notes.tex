\documentclass[11pt]{article}

% PACKAGES
\usepackage{graphicx, amsmath, amssymb, amsfonts, mathtools, mathrsfs, color}
\usepackage{comment, enumerate, tabularx}
\usepackage{natbib, hyperref, url}
%\usepackage[justification=RaggedRight]{caption}

%%%%%%%%%%%%%%%%%%%%%%%%%%%%%%%%%%%%%%
%% LATEX DEFINITIONS
%%%%%%%%%%%%%%%%%%%%%%%%%%%%%%%%%%%%%%

% Basic editing
\newcommand{\tocite}{{\color{blue}(to cite)}}
\newcommand{\vsp}[1]{\vspace{#1 pc} \noindent}
\newcommand{\np}{\newpage \noindent}
% Derivatives
\newcommand{\pd}[2]    { \frac{\partial #1} {\partial #2} }
\newcommand{\ppd}[2]  { \frac{\partial^2 #1}{{\partial #2}^2} }
\newcommand{\pdi}[2] { {\partial_#2} #1 }
\newcommand{\td}[2] { \frac{d #1} { d #2 } }
\newcommand{\grad}{\nabla}
\newcommand {\Lap} {\grad^2}
% Vectors and operators
\newcommand{\bvec}[1]{\ensuremath{\boldsymbol{#1}}}
\newcommand{\abs}[1]{\left| #1 \right|}
\newcommand{\norm}[1]{\left\| #1 \right\|}
\newcommand{\mean}[1]{\left< #1 \right>}
% Specific variables
\newcommand {\bu} {\bvec{u}}
\newcommand{\nhat}{\hat{\bvec{n}}}
\newcommand{\shat}{\hat{\bvec{s}}}
\newcommand{\eps}{\varepsilon}
\newcommand{\atau}{\abs{\tau}}
\newcommand{\thh}{\hat{\theta}}
\newcommand{\GG}{\mathcal{G}}
\newcommand{\Dt}{\Delta t}
\newcommand{\X}{\bvec{X}}
\newcommand {\bq} {\bvec{q}}
% Domains
\newcommand {\bdry} {\partial B}
\newcommand {\onBdry} {\text{on } \bdry }
\newcommand {\xfar} {\text{as } \abs{\bvec{x}} \to \infty }


%%%%%%%%%%%%%%%%%%%%%%%%%%%%%%%%%%%%%%
%% TITLE
%%%%%%%%%%%%%%%%%%%%%%%%%%%%%%%%%%%%%%
\begin{document}
\title{Brief notes on erosion}
\author{}
\maketitle

%\begin{abstract}
%\end{abstract}

\section{Shear stress and erosion law}
The deformation, or rate-of-strain, tensor is given by
\begin{equation}
\bvec{E} = \frac{1}{2} \left( \grad \bu + \grad \bu^T \right)
\end{equation}
where $\bu = (u,v)$ is the velocity field. The stress tensor is given by
\begin{equation}
\bvec{\sigma} = -p \bvec{I} + 2 \mu \bvec{E} = -p \bvec{I} + \mu \left( \grad \bu + \grad \bu^T \right)
\end{equation}
The surface traction is given by
\begin{equation}
\bvec{f} = \bvec{\sigma} \cdot \nhat 	\qquad \onBdry
\end{equation}
where $\nhat$ is the (outward) unit normal and $\bdry$ is the solid boundary. The shear stress $\tau$ is defined as the tangential component of the traction, i.e.~
\begin{equation}
\label{stressdef}
\tau = \bvec{f} \cdot \shat 
= \mu \left( \grad \bu + \grad \bu^T \right) \cdot \nhat \cdot \shat	\qquad \onBdry
\end{equation}
where $\shat$ is the unit tangent vector. It turns out that the shear stress is simply given by the surface vorticity
\begin{equation}
\tau = \mu \left( u_y - v_x \right) \qquad \onBdry
\end{equation}
which can be shown by a direct calculation, using the no-slip boundary conditions (which implies that $\pdi{u}{s} = \pdi{v}{s} = 0$ on $\bdry$).
The interface velocity of the eroding body is linearly proportional to the absolute shear stress
\begin{equation}
V_n = -C \atau
\end{equation}
where $C$ is a constant.



\subsection{The calculation to show stress is equal to vorticity}
First, calculate use Eq.~(\ref{stressdef}) directly to get
\begin{equation}
\tau/\mu = (u_y+v_x) (n_2 s_1 + n_1 s_2 ) + 2 u_x n_1 s_1 + 2 v_y n_2 s_2
\end{equation}
Next, use $(n_1,n_2) = (-s_2,s_1)$ to get
\begin{equation}
\tau/\mu = (u_y+v_x) (s_1^2 - s_2^2) + 2s_1 s_2 (v_y - u_x)
\end{equation}
Then do a complete the square type step
\begin{equation}
\tau/\mu = (u_y - v_x) (s_1^2 + s_2^2) - 2 u_y s_2^2 + 2v_x s_1^2 + 2s_1 s_2 (v_y - u_x)
\end{equation}
Then regroup the last several terms as
\begin{equation}
-2 s_2 ( u_x s_1 + u_y s_2 )
\end{equation}
and
\begin{equation}
2 s_1 ( v_x s_1 + v_y s_2)
\end{equation}
Both of these terms are zero because $\pdi{u}{s} = 0$ and $\pdi{v}{s} = 0$, which shows that
\begin{equation}
\tau/\mu = u_y - v_x
\end{equation}

\vsp{1}
Note: By a similar calculation, I think it is true that
\begin{equation}
\bvec{E} \cdot \nhat \cdot \nhat = 0
\end{equation}
so that $\bvec{f} \cdot \nhat = -p \nhat$. I confirmed this in the simple case of $x=s, y=n$.

\subsection{Drag}
The drag on a body is given by integrating the traction on the surface. If my above claim is true, this reduces to
\begin{equation}
\bvec{F}_D = \int_{\bdry} \bvec{f} \, dS = \int_{\bdry} -p \nhat + \tau \shat \, dS
\end{equation}


\section{Scaling law and interface smoothing}
For Stokes flow, the shear stress scales like
\begin{equation}
\tau \sim \mu V/L
\end{equation}
where $V$ and $L$ are a characteristic velocity and length scale respectively. To evolve the interface in our numerical method, we will probably need to introduce some smoothing to mitigate high frequencies. A common approach is to introduce a small curvature term into the interface velocity, sometimes called the method of vanishing viscosity. The most basic implementation would be
\begin{equation}
V_n = - C \atau - \eps \kappa
\end{equation}
where $\eps$ is a small parameter and $\kappa$ is the surface curvature. However, we have two additional preferences
\begin{enumerate}
\item As the body changes size, we would like the influence of the curvature term relative to the shear stress to remain the same.
\item We prefer the curvature term not to change the size of the body.
\end{enumerate}
Fortunately, the first requirement is already satisfied since both $\tau$ and $\kappa$ scale like $1/L$. To satisfy the second requirement, we replace $\kappa$ with this same quantity minus its mean. To allow for some generality, we allow the curvature term by to be scaled by a power of $L$, giving
\begin{equation}
V_n = - C \atau - \eps L^{\beta} \left(\kappa - \frac{2 \pi}{L} \right)
\end{equation}
Thus, for the present case of Stokes flow $\beta=0$, but for the previous case of high-Reynolds-number flow $\beta=1/2$.

\section{Numerical interface evolution}

To evolve the interface numerically, we use the $\theta$-$L$ method. Rather than re-derive the formulas, I will use the convention from Shelley 1994, with the curve parameterized in the CCW direction and the normal vector pointing {\em inward}. With this convention, the normal interface velocity is the negative of what it was previously, giving
\begin{equation}
\label{Vn1}
V_n = C \atau + \eps L^{\beta} \left(\kappa - \frac{2 \pi}{L} \right)
\end{equation}
Recall that for Stokes flow, $\beta = 0$.

Now, the interface $\X = (x,y)$ evolves according to
\begin{equation}
\pd{\bvec{X}}{t} = V_n \nhat + V_s \shat
\end{equation}
where $\nhat = (-y_s, x_s)$ is the inward pointing normal vector and $\shat = (x_s, y_s)$ is the tangential vector (running in the CCW direction). Here, we have introduced a tangential velocity $V_s$ that will be chosen to keep the curve in the equal-arclength parameterization. We will frequently use a normalized arclength $\alpha = s/L$, where $L$ is the total length of the curve.

We introduce the tangent angle $\theta$, defined by
\begin{equation}
(x_s, y_s) = (\cos \theta, \sin \theta)
\end{equation}
so that we have
\begin{equation}
\tan \theta = y_s/x_s = y_\alpha / x_\alpha
\end{equation}
Remember the identity that $x_s^2 + y_s^2 = 1$ as this is very useful.
The curvature is given by
\begin{equation}
\kappa = x_s y_{ss} - x_{ss} y_s = \pd{\theta}{s} = L^{-1} \pd{\theta}{\alpha}
\end{equation}
We are using the standard convention, where a convex shape, like a circle, has positive curvature. The normal velocity now can be written as
\begin{equation}
\label{Vn2}
V_n = C \atau + \eps L^{\beta-1} \left(\pd{\theta}{\alpha} - 2 \pi \right)
\end{equation}

We now discuss how to choose $V_s$ to remain in the equal-arclength frame of reference. We will use the term
\begin{equation}
\gamma = \pd{\theta}{\alpha} V_n
\end{equation}
Then the necessary tangential velocity is determined by
\begin{equation}
\td{V_s}{\alpha} = \gamma - \mean{\gamma}
\end{equation}
where $\mean{}$ indicates the mean value.

We now discuss how to evolve $\theta$ and $L$ forward in time according to the interface velocity given above. The resulting evolution equations are
\begin{align}
& \td{L}{t} = - \mean{\gamma} \\
& \pd{\theta}{t} = L^{-1} \left( \pd{V_n}{\alpha} + \pd{\theta}{\alpha} V_s \right)
\end{align}
Then, using the above $V_n$, the equation for $\theta$ simplifies to
\begin{equation}
\label{thetaEvolve}
\pd{\theta}{t} = \eps L^{\beta-2} \ppd{\theta}{\alpha} + L^{-1} \left( C \pd{\atau}{\alpha} + \pd{\theta}{\alpha} V_s \right)
\end{equation}

We now discuss the numerical methods used to solve the above evolution equations. For $L$, we use an explicit method
\begin{equation}
L^{(n+1)} = L^{(n)} + \frac{\Dt}{2} \left( 3 M^{(n)} - M^{(n-1)} \right)
\end{equation}
where
\begin{equation}
M = \td{L}{t} = -\mean{\gamma}
\end{equation}
For $\theta$, we use a spectral method, giving
\begin{equation}
\td{\thh_k}{t} + 4 \pi^2 k^2 \eps L^{\beta-2} \thh_k = \hat{N}_k
\end{equation}
where the nonlinear terms are grouped together as
\begin{equation}
N = L^{-1} \left( C \pd{\atau}{\alpha} + \pd{\theta}{\alpha} V_s \right)
\end{equation}
We evolve with the integrating-factor method. When recast in terms of Gaussian filters, this gives
\begin{equation}
\label{intfact}
\thh^{(n+1)} = \GG(\theta^{(n)}, \sigma_1) 
+ \frac{\Dt}{2} \left( 3 \GG(N^{(n)}, \sigma_1) - \GG(N^{(n-1)}), \sigma_2 \right)
\end{equation}
where
\begin{align}
& \sigma_1 = 2 \pi \sqrt{\eps \Dt} \sqrt{(L^{n})^{\beta-2} + (L^{n+1})^{\beta-2} } 	\\
& \sigma_2 = 2 \pi \sqrt{\eps \Dt} \sqrt{(L^{n-1})^{\beta-2} + 2 (L^{n})^{\beta-2} + (L^{n+1})^{\beta-2} }
\end{align}


\subsection{Different scaling of the stress}
Some numerical tests reveal that the stress scales like $\tau \sim 1/(L \log L)$, instead of the naive $1/L$. This might because we are in a 2D bounded domain (Stokes paradox type stuff). So we need to modify the curvature driven flow to have the same scaling. To this end, modify Eq.~(\ref{Vn1}) to read
\begin{equation}
V_n = \atau + \eps f(L) \left(\kappa - \frac{2 \pi}{L} \right)
\end{equation}
where for 2D Stokes flow, we will have $f(L) = -1/\log L$. Carrying through this change, Eq.~(\ref{Vn2}) becomes
\begin{equation}
V_n = \atau + \eps L^{-1} f(L) \left(\pd{\theta}{\alpha} - 2 \pi \right)
\end{equation}
and Eq.~(\ref{thetaEvolve}) becomes
\begin{equation}
\pd{\theta}{t} = \eps L^{-2} f(L) \ppd{\theta}{\alpha} + L^{-1} \left( \pd{\atau}{\alpha} + \pd{\theta}{\alpha} V_s \right)
\end{equation}
Then to evolve $\theta$ spectrally, we use
\begin{equation}
\td{\thh_k}{t} + 4 \pi^2 k^2 \eps L^{-2} f(L)  \thh_k = \hat{N}_k
\end{equation}
where the nonlinear terms are grouped together as
\begin{equation}
N = L^{-1} \left( \pd{\atau}{\alpha} + \pd{\theta}{\alpha} V_s \right)
\end{equation}
We evolve with the integrating-factor method. When recast in terms of Gaussian filters, this gives
\begin{equation}
\label{intfact}
\thh^{(n+1)} = \GG(\theta^{(n)}, \sigma_1) 
+ \frac{\Dt}{2} \left( 3 \GG(N^{(n)}, \sigma_1) - \GG(N^{(n-1)}), \sigma_2 \right)
\end{equation}
where
\begin{align}
& \sigma_1 = 2 \pi \sqrt{\eps \Dt} \sqrt{g(L^{n}) + g(L^{n+1}) } 	\\
& \sigma_2 = 2 \pi \sqrt{\eps \Dt} \sqrt{g(L^{n-1}) + 2 g(L^{n}) + g(L^{n+1}) }
\end{align}
where $g(L) = L^{-2} f(L)$.
%This will also modify the scaling law for area vs time!!!

\vsp{2}
{\bf  Alternative}: Perhaps I just want to calculate the mean of the shear stress directly and use it to scale the curvature driven flow. This idea gives
\begin{equation}
V_n = \atau + \eps \mean{\atau} \left(\kappa - \frac{2 \pi}{L} \right)
\end{equation}
which gives
\begin{equation}
V_n = \atau + \eps L^{-1} \mean{\atau} \left(\pd{\theta}{\alpha} - 2 \pi \right)
\end{equation}
Then we get
\begin{equation}
\pd{\theta}{t} = \eps L^{-2} \mean{\atau} \ppd{\theta}{\alpha} + L^{-1} \left( \pd{\atau}{\alpha} + \pd{\theta}{\alpha} V_s \right)
\end{equation}
Then to evolve $\theta$ spectrally, we use
\begin{equation}
\td{\thh_k}{t} + 4 \pi^2 k^2 \eps L^{-2} \mean{\atau}  \thh_k = \hat{N}_k
\end{equation}
Then the Gaussian filter stuff becomes
\begin{equation}
\label{intfact}
\thh^{(n+1)} = \GG(\theta^{(n)}, \sigma_1) 
+ \frac{\Dt}{2} \left( 3 \GG(N^{(n)}, \sigma_1) - \GG(N^{(n-1)}), \sigma_2 \right)
\end{equation}
where
\begin{align}
& \sigma_1 = 2 \pi \sqrt{\eps \Dt} \sqrt{g^{n} + g^{n+1} } 	\\
& \sigma_2 = 2 \pi \sqrt{\eps \Dt} \sqrt{g^{n-1} + 2 g^{n} + g^{n+1} }
\end{align}
where $g(t)= L^{-2} \mean{\atau}$ and $g^{(n)} = g(t^{(n)})$.
BUT the problem with this is that I do not yet know $\mean{\atau}$ at time-step $n+1$, so I would have to change my time-stepping strategy which is a significant change to the code structure. Thus, I prefer the first method of using $L \log L$.


\subsection{The vanishing rate of the area}
The rate of area loss is given by
\begin{equation}
\dot{A} = -C \abs{\tau} L
\end{equation}
We then insert the scaling $\abs{\tau} \sim -1/(L \log L)$ and simplify to get
\begin{equation}
\dot{A} = C/\log{A}
\end{equation}
which we solve to get
\begin{equation}
A (1 - \log A) = C (t_f - t)
\end{equation}
In particular, I believe the signs are correct here.

\subsection{How to scale smoothing parameters?}
We have two smoothing parameters: $\eps$ and $\sigma$. How should they scale with the simulation resolution parameters? The most important resolution parameter is $N$, the total number of points on each body, but there is also $\Delta t$ and maybe some others. Figuring out $\sigma$ is the easier part: $\sigma$ is a lengthscale and should go like $1/N$. On the other hand, $\eps$ is a diffusivity in dimensions. I did a calculation and found that $\eps$ should scale like $2/(\pi^2 N^2)$.

The calculation is based on looking at Eq.~(\ref{intfact}), where $\sigma_1 \sim 2 \pi \sqrt{\eps \Delta t}$. For the largest wavenumbers, $k_{max} \sim N/2$, we want the smoothing to beat the nonlinear term (so that overall, these modes decay). So we need $e^{-\pi^2 \eps \Delta t N^2 / 2} < 1 - C\Delta t$, which gives $\eps > 2C/(\pi^2 N^2)$, independent of $\Delta t$. If $C=1$, this gives roughly $\eps > 0.2/ N^2$. However, this analysis did not take into account how the spectrum of the nonlinear term decays (or maybe it even grows???)


\subsection{Tracking a reference point: the surface-averaged coordinates}
To evolve the body, I need to track the motion of one reference point. The initial choice might be center of mass, but this turns out to be complicated. Instead, it is simpler to consider the surface-mean coordinates
\begin{equation}
(x_{sm}, y_{sm}) = \int_0^1 (x(\alpha), y(\alpha)) \, d\alpha
\end{equation}
These coordinates move according to
\begin{align}
& \td{}{t} x_{sm} = \int_0^1 V_n \, n_x + V_s \, s_x \, d\alpha \\
& \td{}{t} y_{sm} = \int_0^1 V_n \, n_y + V_s \, s_y \, d\alpha
\end{align}
Given $\nhat = (-\sin \theta, \cos \theta)$ and $\shat = (\cos \theta, \sin \theta)$, these simplify to
\begin{align}
& \td{}{t} x_{sm} = \int_0^1 -V_n \sin \theta + V_s \cos \theta \, d\alpha \\
& \td{}{t} y_{sm} = \int_0^1 V_n \cos \theta + V_s \sin \theta \, d\alpha
\end{align}
Note 1: Initially, I was unsure whether $\int_0^1 V_s \hat{s} d\alpha$ makes a non-zero contribution or not, but based on some tests, it seems that it does make a difference. The order of magnitude is the same as the first term involving $V_n$. \\
Note 2: If I need to calculate the area, the divergence theorem gives
\begin{equation}
A = \int x n_x ds = \int y n_y ds
\end{equation}

%% CENTER OF MASS calculations... OBSELETE
\begin{comment}
\subsubsection{The center of mass}
Consider the body's center of mass
\begin{equation}
(x_{cm}, y_{cm}) = \frac{1}{A} \iint (x, y) \, dA
\end{equation}
I did a calculation to determine the motion of the center of mass, given the interface velocity. The result that I get is
\begin{equation}
\td{}{t} (x_{cm}, y_{cm})  = -\frac{1}{A} \int (x(s), y(s)) V_n(s) \, ds
\end{equation}
I believe the sign is correct for $\nhat$ the inward pointing normal, so that $V_n$ is positive if the body is shrinking.
To calculate the center of mass, I can use the divergence theorem
\begin{equation}
\iint \grad \cdot \bvec{F} \, dA = \int \bvec{F} \cdot \nhat \, ds
\end{equation}
Take $\bvec{F} = \frac{1}{2} (x^2, y^2)$, so that $\grad \cdot \bvec{F} = (x,y)$. Then
\begin{equation}
(x_{cm}, y_{cm} ) = \frac{1}{2 A} \int x^2 n_x + y^2 n_y \, ds
\end{equation}
To calculate the total area, $A$, I can use the divergence theorem again, to get
\begin{equation}
A = \int x n_x ds = \int y n_y ds
\end{equation}
\end{comment}
% END CENTER OF MASS

\subsection{Benchmark the interface evolution}
It would be good to test the interface evolution, in particular verify that the curvature-driven-flow (CDF) is 2nd order accurate in time. Initially, I thought I would be able to get exact solutions for CDF, but it turns out that the PDE (\ref{thetaEvolve}) does not reduce to simply a diffusion equation, but instead still has nonlinear terms, thus making exact solutions impractical. I also tried to recast CDF in polar coordinates for star-shapes, but still the governing PDE is not simple enough to get exact solutions. Also, in the $\theta$-$L$ formulation I tried to determine the normal velocity to make the nonlinear terms vanish, but this does not appear possible either. Instead, I can do a completely numerical convergence test.


\section{Permeability, conductivity, drag, resistance, and all that}

Darcy's law is
\begin{equation}
\bq = - \bvec{K} \grad \phi
\end{equation}
where $\bq$ is the specific discharge, $\phi = z + p/\gamma$ is the hydraulic head, and $\gamma = \rho g$. The rank-2 tensor $\bvec{K}$ is the {\em conductivity}, which is related to the {\em permeability} $\bvec{k}$ via $\bvec{K} = \bvec{k} \gamma/\mu$, where $\mu$ is the viscosity of the fluid. The units of conductivity are length-per-time (i.e.~velocity) and the units of permeability are length squared.
Perhaps the most standard parameter to characterize the porous medium is the permeability, so lets focus on that. Let us also eliminate gravity from consideration to get
\begin{equation}
\bq = - \bvec{k} \grad p / \mu
\end{equation}
Consider first an isotropic medium with permeability $k$ (scalar). As a warmup, consider flow in a channel of width $2w$ due to a pressure difference of $\Delta p$ occurring over a length $2 \ell$, so that the pressure gradient is $\grad p = -\Delta p/(2 \ell)$. If I did it correctly, the velocity is
\begin{equation}
u(z) = \frac{\Delta p}{4 \mu \ell} (w^2 - z^2)
\end{equation}
Then we can calculate the discharge as
\begin{equation}
q = \frac{1}{2w} \int_{-w}^{w} u(z) dz = \frac{\Delta p w^2}{6 \mu \ell} = - \frac{w^2}{3 \mu} \grad p
\end{equation}
so that the permeability due to the channel walls is given by
\begin{equation}
k_w = w^2/3
\end{equation}
This result at least has the correct units (length squared).

Next, consider measuring the permeability for flow through a matrix of bodies. To accomplish this, we should measure the pressure and velocity along two vertical lines, one at $x = x_0$ and another at $x = -x_0$, and average to get $p^+ = \left< p(x_0) \right>$, $p^- = \left<p(-x_0) \right>$ and the same for $q$. Then the measured value of the total permeability (setting $\mu = 1$), is
\begin{equation}
k_{tot} = \frac{x_0 (q^+ + q^-)}{p^- - p^+}
\end{equation}
Presumably, $q$ (the average along a vertical cross-section) should be the same everywhere, but we should check this. 

The `total' permeability above includes resistance from both the bodies and the channel walls, and so the value is artificially low (the true permeability of just the bodies will be a bit higher). In electrical circuits, the conductivity and resistivity are simply reciprocals of one another, so I presume the same holds for porous media flow, giving
\begin{equation}
\frac{1}{k_{tot}} = \frac{1}{k_w} + \frac{1}{k_b}
\end{equation}
 where $k_w = w^2/3$ is from the channel walls and $k_b$ is from the erodible bodies. Then, to compute the permeability from the bodies, we would remove the contribution from the walls. First, lets consider the resistances (the reciprocals of the permeabilities)
\begin{equation}
\frac{1}{k_b} = \frac{1}{k_{tot}} - \frac{1}{k_w} = \frac{k_w - k_{tot}}{k_w k_{tot}}
\end{equation}
which gives
\begin{equation}
k_b = \frac{k_w k_{tot}}{k_w - k_{tot}} \, , .
\end{equation}

Things are more straightforward in terms of the resistivity, $r = 1/k$, which gives
\begin{equation}
r_{tot} = \frac{p^- - p^+}{2 x_0 q_{avg}}
\end{equation}
and the contribution from the walls should be $r_w = 3/w^2$, so we get
\begin{equation}
r_{b} = r_{tot} - 3/w^2
\end{equation}



\end{document}
