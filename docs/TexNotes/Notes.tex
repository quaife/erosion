\documentclass[11pt]{article}

% PACKAGES
\usepackage{graphicx, amsmath, amssymb, amsfonts, mathtools, mathrsfs, color}
\usepackage{comment, enumerate, tabularx}
\usepackage{natbib, hyperref, url}
%\usepackage[justification=RaggedRight]{caption}

%%%%%%%%%%%%%%%%%%%%%%%%%%%%%%%%%%%%%%
%% LATEX DEFINITIONS
%%%%%%%%%%%%%%%%%%%%%%%%%%%%%%%%%%%%%%

% Basic editing
\newcommand{\tocite}{{\color{blue}(to cite)}}
\newcommand{\vsp}[1]{\vspace{#1 pc} \noindent}
\newcommand{\np}{\newpage \noindent}
% Derivatives
\newcommand{\pd}[2]    { \frac{\partial #1} {\partial #2} }
\newcommand{\ppd}[2]  { \frac{\partial^2 #1}{{\partial #2}^2} }
\newcommand{\pdi}[2] { {\partial_#2} #1 }
\newcommand{\td}[2] { \frac{d #1} { d #2 } }
\newcommand{\grad}{\nabla}
\newcommand {\Lap} {\grad^2}
% Vectors and operators
\newcommand{\bvec}[1]{\ensuremath{\boldsymbol{#1}}}
\newcommand{\abs}[1]{\left| #1 \right|}
\newcommand{\norm}[1]{\left\| #1 \right\|}
\newcommand{\mean}[1]{\left< #1 \right>}
% Specific variables
\newcommand {\bu} {\bvec{u}}
\newcommand{\nhat}{\hat{\bvec{n}}}
\newcommand{\shat}{\hat{\bvec{s}}}
\newcommand{\eps}{\varepsilon}
\newcommand{\atau}{\abs{\tau}}
\newcommand{\thh}{\hat{\theta}}
\newcommand{\GG}{\mathcal{G}}
\newcommand{\Dt}{\Delta t}
\newcommand{\X}{\bvec{X}}
\newcommand {\bq} {\bvec{q}}
% Domains
\newcommand {\bdry} {\partial B}
\newcommand {\onBdry} {\text{on } \bdry }
\newcommand {\xfar} {\text{as } \abs{\bvec{x}} \to \infty }

% FROM METHODS TEX
\newcommand{\xx}{{\mathbf{x}}}
\renewcommand{\ss}{{\mathbf{s}}}
\newcommand{\nn}{{\mathbf{n}}}
% Derivatives
\newcommand{\pderiv}[2]{\frac{\partial #1}{\partial #2}}
\newcommand{\tderiv}[2]{\frac{d #1}{d #2}}
% Nick's commands
\newcommand{\thL}{$\theta$--$L$}
\newcommand{\Vn}{V_n}
\newcommand{\Vs}{V_s}
\newcommand{\thalpha}{\pderiv{\theta}{\alpha}}
\newcommand{\elfun}{\zeta}
\newcommand{\thhat}{\hat{\theta}}
\newcommand{\dt}{\Delta t}
% New
\newcommand{\pup}{p_{1}}
\newcommand{\pdn}{p_{2}}
\newcommand{\qavg}{q_{avg}}

%%%%%%%%%%%%%%%%%%%%%%%%%%%%%%%%%%%%%%
%% TITLE
%%%%%%%%%%%%%%%%%%%%%%%%%%%%%%%%%%%%%%
\begin{document}
\title{Brief notes on erosion}
\author{}
\maketitle

%\begin{abstract}
%\end{abstract}

\section{Shear stress and erosion law}
The deformation, or rate-of-strain, tensor is given by
\begin{equation}
\bvec{E} = \frac{1}{2} \left( \grad \bu + \grad \bu^T \right)
\end{equation}
where $\bu = (u,v)$ is the velocity field. The stress tensor is given by
\begin{equation}
\bvec{\sigma} = -p \bvec{I} + 2 \mu \bvec{E} = -p \bvec{I} + \mu \left( \grad \bu + \grad \bu^T \right)
\end{equation}
The surface traction is given by
\begin{equation}
\bvec{f} = \bvec{\sigma} \cdot \nhat 	\qquad \onBdry
\end{equation}
where $\nhat$ is the (outward) unit normal and $\bdry$ is the solid boundary. The shear stress $\tau$ is defined as the tangential component of the traction, i.e.~
\begin{equation}
\label{stressdef}
\tau = \bvec{f} \cdot \shat 
= \mu \left( \grad \bu + \grad \bu^T \right) \cdot \nhat \cdot \shat	\qquad \onBdry
\end{equation}
where $\shat$ is the unit tangent vector. It turns out that the shear stress is simply given by the surface vorticity
\begin{equation}
\tau = \mu \left( u_y - v_x \right) \qquad \onBdry
\end{equation}
which can be shown by a direct calculation, using the no-slip boundary conditions (which implies that $\pdi{u}{s} = \pdi{v}{s} = 0$ on $\bdry$).
The interface velocity of the eroding body is linearly proportional to the absolute shear stress
\begin{equation}
V_n = -C \atau
\end{equation}
where $C$ is a constant. In what follows, we will scale time on an erosion timescale (non-dimensionalize) and thus set $C=1$.


\subsection{The calculation to show stress is equal to vorticity}
First, calculate use Eq.~(\ref{stressdef}) directly to get
\begin{equation}
\tau/\mu = (u_y+v_x) (n_2 s_1 + n_1 s_2 ) + 2 u_x n_1 s_1 + 2 v_y n_2 s_2
\end{equation}
Next, use $(n_1,n_2) = (-s_2,s_1)$ to get
\begin{equation}
\tau/\mu = (u_y+v_x) (s_1^2 - s_2^2) + 2s_1 s_2 (v_y - u_x)
\end{equation}
Then do a complete the square type step
\begin{equation}
\tau/\mu = (u_y - v_x) (s_1^2 + s_2^2) - 2 u_y s_2^2 + 2v_x s_1^2 + 2s_1 s_2 (v_y - u_x)
\end{equation}
Then regroup the last several terms as
\begin{equation}
-2 s_2 ( u_x s_1 + u_y s_2 )
\end{equation}
and
\begin{equation}
2 s_1 ( v_x s_1 + v_y s_2)
\end{equation}
Both of these terms are zero because $\pdi{u}{s} = 0$ and $\pdi{v}{s} = 0$, which shows that
\begin{equation}
\tau/\mu = u_y - v_x
\end{equation}

\vsp{1}
Note: By a similar calculation, I think it is true that
\begin{equation}
\bvec{E} \cdot \nhat \cdot \nhat = 0
\end{equation}
so that $\bvec{f} \cdot \nhat = -p \nhat$. I confirmed this in the simple case of $x=s, y=n$.

\subsection{Drag}
The drag on a body is given by integrating the traction on the surface. If my above claim is true, this reduces to
\begin{equation}
\bvec{F}_D = \int_{\bdry} \bvec{f} \, dS = \int_{\bdry} -p \nhat + \tau \shat \, dS
\end{equation}


\section{Scaling law and interface smoothing}
For 3D Stokes flow, the shear stress scales like
\begin{equation}
\tau \sim \mu U/a
\end{equation}
where $U$ and $a$ are a characteristic velocity and length scale respectively.  However, for 2D Stokes flow we have Stokes paradox and slender body effects. The analogous scaling of the stress is
\begin{equation}
\tau \sim \frac{\mu U}{a \log(\ell/a)}
\end{equation}
where $\ell$ is the width in the third dimension. 
%For example, we can take $\ell = 2\pi$, since a body cannot have a length greater than $2\pi$ and still fit inside the domain.



\subsection{The vanishing rate of the area}

The area loss rate is given by
\begin{align}
\dot{A} = - \int_{\gamma} \atau \, ds \,  = - \mean{\atau} L \, .
\end{align}
The above is an exact relation. Substitution of the scaling law gives
\begin{align}
\dot{A} \sim \frac{ - \mu U}{\log(\ell/a)} \, .
\end{align}
Inserting $A \sim a^2$ and rearranging gives
\begin{align}
\log (c_2 A/\ell^2) dA = c_3 \mu U dt
\end{align}
Integrating gives
\begin{align}
A \left(1 - \log(c_2 A/\ell^2) \right) = c_3 \mu U (t_f - t)
\end{align}
Rearranging and setting some constants based on knowing $A(t=0) = A_0$ gives
\begin{align}
\label{area_predict}
\frac{A}{A_0} \left( 1 - c_1 \log{\frac{A}{A_0}} \right) \sim \frac{t_f - t}{t_f} \quad \text{as } t \to t_f \, .
\end{align}
where $c_1$ is a constant, and I find $c_1 = 0.24$ gives an excellent fit to the data.

Now differentiation the result with respect to time gives
\begin{align}
\dot{A} = \frac{A_0/t_f}{1 - c_1 (1 + \log(A/A_0))} \sim - \frac{\mu U}{\log(\ell / a)}
\end{align}
Now, doing some manipulations, using that $A = c_3 a^2$ and $A_0 = c_3 a_0^2$, gives the nice relationship
\begin{align}
\ell = a_0 e^{\frac{1-c_1}{2 c_1}} = 4.87 a_0 
\end{align}

Now, to estimate the drag, use the slender-body formula for drag per unit span, $\ell$, which is
\begin{align}
F_D \sim \frac{4 \pi \mu U}{\log(\ell/a)}
\end{align}
The $4 \pi$ coefficient is exact (in the slender-body framework) for a circular cylinder. Substituting the value we found $\ell = 4.87 a_0$, gives
\begin{align}
F_D \sim \frac{4 \pi \mu U}{\log(4.87 a_0/a)}
\end{align}
For non-circular bodies, we can define $a = \sqrt{A/\pi}$ since this is the value that would be consistent with a circle. This gives
\begin{align}
F_D \sim \frac{4 \pi \mu U}{\log(4.87 \sqrt{A_0/A})}
\end{align}
or
\begin{equation}
F_D \sim \frac{8 \pi \mu U}{\log(23.7 A_0/A)}
\end{equation}

Alternatively we can just treat $\ell$ as a fit parameter.

\section{Numerical interface evolution}

To handle boundary evolution, we use the {\thL} method, which offers certain advantages in the ability to stabilize fluid-structure interaction problems. Throughout this section, we use the following conventions: the bodies are parametrized in the CCW direction and the normal vector points into the bodies (our out of the fluid).

In order to regularize boundary evolution, we modify the normal velocity as follows
\begin{equation}
\label{eqn:Vn}
\Vn = \atau + \frac{\eps}{\log \left(2\pi/L \right)} \left(\kappa - \frac{2 \pi}{L} \right)
\end{equation}
Here, $\kappa$ is the curvature, $\eps$ is a small parameter, and $L$ is the total arc length of the body. The second term on the right acts to smooth bodies that have high-frequency oscillations. This is sometimes called the method of vanishing viscosity. Since we have subtracted the mean curvature, $2\pi/L$, this term is area-preserving, so that the only material loss is due to the shear stress. We chose the prefactor $\log^{-1} \left(2\pi/L \right)$, so that, for 2D Stokes flow, this term scales the same as shear stress as $L \to 0$. Thus, the relative strength of the smoothing is always the same. 


Note: The curvature is given by
\begin{equation}
\kappa = x_s y_{ss} - x_{ss} y_s = \pd{\theta}{s} = L^{-1} \pd{\theta}{\alpha}
\end{equation}
We are using the standard convention, where a convex shape, like a circle, has positive curvature.

We now introduce the tangent angle, $\theta$, defined through the relation
\begin{equation}
\left( \pderiv{x}{s}, \pderiv{y}{s} \right) = \left(\cos \theta, \sin \theta \right)
\end{equation}
where $s$ is arc length. We also introduce the normalized arc length $\alpha = s / L$. 

Note: We have this relation
\begin{equation}
\tan \theta = y_s/x_s = y_\alpha / x_\alpha
\end{equation}
and remember the identity that $x_s^2 + y_s^2 = 1$ as this is very useful.

In terms of these new variables, the curvature can be expressed as
\begin{equation}
\kappa = \frac{1}{L} \pderiv{\theta}{\alpha}
\end{equation}
Substitution into Eq.~(\ref{eqn:Vn}) gives
\begin{equation}
\label{eqn:Vn2}
\Vn = \atau + \frac{\eps}{L \log \left(2\pi/L \right)}  \left(\pderiv{\theta}{\alpha} - 2 \pi \right)
\end{equation}
The main idea is to introduce an artificial tangential velocity that keeps collocation points equally spaced with respect to arc length; i.e.~the grid of $\alpha$ should always remain equally spaced. The tangential velocity that does the trick is determined by Shelley 1994 as
\begin{equation}
\pderiv{\Vs}{\alpha} = \thalpha \Vn - \mean{\thalpha \Vn}
\end{equation}
The interface evolution is then given by
\begin{equation}
\dot{\xx}(t) = \Vn \nn + \Vs \ss.
\end{equation}

Note: with the above conventions, we have $\nn = (-y_s, x_s)$ as the inward pointing normal vector and $\ss = (x_s, y_s)$ as the tangential vector (running in the CCW direction). 

Upon transforming to the variables $\theta(\alpha,t)$ and $L(t)$, we obtain the evolution equations
\begin{align}
& \tderiv{L}{t} = - \mean{\thalpha \Vn} \\
& \pderiv{\theta}{t} = \frac{1}{L} \left( \pderiv{\Vn}{\alpha} + \thalpha \Vs \right)
\end{align}
Inserting Eq.~(\ref{eqn:Vn2}) for the normal velocity gives
\begin{equation}
\pderiv{\theta}{t} = \frac{\eps}{L^{2} \log \left(2\pi/L \right)} \ppd{\theta}{\alpha} 
+ \frac{1}{L} \left( \pderiv{\atau}{\alpha} + \thalpha \Vs \right)
\end{equation}
For simplicity, we introduce the variables
\begin{align}
& M = - \mean{\thalpha \Vn} \\
& N = \frac{1}{L} \left( \pderiv{\atau}{\alpha} + \thalpha \Vs \right) \\
& \elfun = \frac{1}{L^{2} \log \left(2\pi/L \right)}
\end{align}
Then the evolution equations can be expressed as
\begin{align}
& \tderiv{L}{t} = M \\
& \pderiv{\theta}{t} = \eps \elfun \ppd{\theta}{\alpha} + N
\end{align}
We treat $\theta$ spectrally
\begin{equation}
\theta(\alpha,t) = 2 \pi \alpha + \sum_{k = -\infty}^{\infty} \thhat_k(t) e^{2\pi i k \alpha}
\end{equation}
where $2 \pi \alpha$ is the linear part and does not depend on time. Then we get
\begin{equation}
\label{thODEs}
\tderiv{\thhat_k}{t} +  4 \eps \pi^2 k^2  \elfun \thhat_k = \hat{N}_k
\end{equation}
We introduce the integrating factor
\begin{equation}
\mu_k = \exp \left( 4 \eps \pi^2 k^2 \int \elfun dt \right)
\end{equation}
so that our system becomes
\begin{align}
\label{Leq}
& \tderiv{L}{t} = M \\
\label{theq}
& \tderiv{}{t}\left( \mu_k \theta_k \right) = \mu_k \hat{N}_k
\end{align}
We discretize Eqs.~(\ref{Leq})--(\ref{theq}) with a second-order Runge-Kutta method (the midpoint RK2). See the methods paper for details.



\subsection{Tracking a reference point: the surface-averaged coordinates}
To evolve the body, I need to track the motion of one reference point. The initial choice might be center of mass, but this turns out to be complicated. Instead, it is simpler to consider the surface-mean coordinates
\begin{equation}
(x_{sm}, y_{sm}) = \int_0^1 (x(\alpha), y(\alpha)) \, d\alpha
\end{equation}
These coordinates move according to
\begin{align}
& \td{}{t} x_{sm} = \int_0^1 V_n \, n_x + V_s \, s_x \, d\alpha \\
& \td{}{t} y_{sm} = \int_0^1 V_n \, n_y + V_s \, s_y \, d\alpha
\end{align}
Given $\nhat = (-\sin \theta, \cos \theta)$ and $\shat = (\cos \theta, \sin \theta)$, these simplify to
\begin{align}
& \td{}{t} x_{sm} = \int_0^1 -V_n \sin \theta + V_s \cos \theta \, d\alpha \\
& \td{}{t} y_{sm} = \int_0^1 V_n \cos \theta + V_s \sin \theta \, d\alpha
\end{align}
Note 1: Initially, I was unsure whether $\int_0^1 V_s \hat{s} d\alpha$ makes a non-zero contribution or not, but based on some tests, it seems that it does make a difference. The order of magnitude is the same as the first term involving $V_n$. \\
Note 2: If I need to calculate the area, the divergence theorem gives
\begin{equation}
A = \int x n_x ds = \int y n_y ds
\end{equation}

%% CENTER OF MASS calculations... OBSELETE
\begin{comment}
\subsubsection{The center of mass}
Consider the body's center of mass
\begin{equation}
(x_{cm}, y_{cm}) = \frac{1}{A} \iint (x, y) \, dA
\end{equation}
I did a calculation to determine the motion of the center of mass, given the interface velocity. The result that I get is
\begin{equation}
\td{}{t} (x_{cm}, y_{cm})  = -\frac{1}{A} \int (x(s), y(s)) V_n(s) \, ds
\end{equation}
I believe the sign is correct for $\nhat$ the inward pointing normal, so that $V_n$ is positive if the body is shrinking.
To calculate the center of mass, I can use the divergence theorem
\begin{equation}
\iint \grad \cdot \bvec{F} \, dA = \int \bvec{F} \cdot \nhat \, ds
\end{equation}
Take $\bvec{F} = \frac{1}{2} (x^2, y^2)$, so that $\grad \cdot \bvec{F} = (x,y)$. Then
\begin{equation}
(x_{cm}, y_{cm} ) = \frac{1}{2 A} \int x^2 n_x + y^2 n_y \, ds
\end{equation}
To calculate the total area, $A$, I can use the divergence theorem again, to get
\begin{equation}
A = \int x n_x ds = \int y n_y ds
\end{equation}
\end{comment}
% END CENTER OF MASS


\section{Permeability, conductivity, drag, resistance, and all that}

\subsection{Total permeability and resistivity}
Darcy's law is
\begin{equation}
\bq = - \bvec{K} \grad \phi
\end{equation}
where $\bq$ is the specific discharge, $\phi = z + p/\gamma$ is the hydraulic head, and $\gamma = \rho g$. The rank-2 tensor $\bvec{K}$ is the {\em conductivity}, which is related to the {\em permeability} $\bvec{k}$ via $\bvec{K} = \bvec{k} \gamma/\mu$, where $\mu$ is the viscosity of the fluid. The units of conductivity are length-per-time (i.e.~velocity) and the units of permeability are length squared.
Perhaps the most standard parameter to characterize the porous medium is the permeability, so lets focus on that. Let us also eliminate gravity from consideration to get
\begin{equation}
\bq = - \bvec{k} \grad p / \mu
\end{equation}
Consider first an isotropic medium with permeability $k$ (scalar). Overall, it is more straightforward to work with the resistivity $r = 1/k$.

As a warmup, consider flow in a channel of width $2w$ due to uniform (negative) pressure gradient $p_x$ in the $x$-direction. The resulting velocity profile is
\begin{equation}
u(z) = -\frac{p_x}{2 \mu} (w^2 - z^2)
\end{equation}
Then we can calculate the discharge as
\begin{equation}
\qavg = \frac{1}{2w} \int_{-w}^{w} u(z) dz = - \frac{ p_x w^2}{3 \mu} 
\end{equation}
so that the permeability due to the channel walls is $k_w = w^2/3$ and the resistivity is $r_w = 3/w^2$.

If we define $u(z) = U(w^2 - z^2)/w^2$, then $U = -w^2 p_x / (2 \mu)$ and $U$ is the maximum velocity that is obtained. Calculation gives
\begin{align}
\qavg = \frac{2}{3} U
\end{align}

Next, consider the flow through a porous matrix (and set $\mu=1$ via nondimensionalization). Measure the pressure and velocity along two vertical lines, one upstream at $x = -x_0$ and another at the symmetric downstream location $x = x_0$. Average to get $\pup = \left<p(-x_0) \right>$ and $\pdn = \left< p(x_0) \right>$, and the same for $q$. Then the {\em total} resistivity (that of the porous matrix and the channel walls together) is 
\begin{equation}
\label{rtot}
r_{tot} = \frac{\pup - \pdn}{2 x_0 \, \qavg}
\end{equation}


%-------------------------------------------------------------%
% ISOLATING RESISTIVITY DUE TO BODIES
%-------------------------------------------------------------%
\subsection{Isolating the resistivity due to the bodies}
The above total resistivity reflects the resistivity from the porous matrix as well as the channel walls. There is a buffer region that lies outside the porous-matrix region ($x_0$ lies in the buffer region), and we would like to remove the contribution from this region. Let's regard the total resistivity as an area-integral of resistivity-density, so that we can write the total in terms of the areas of the regions
\begin{equation}
A_{t} r_{t} = A_{b} r_{b} + A_{m} r_{m}
\end{equation}
Here, the subscripts are as follows: $t$ stands for total, $b$ for buffer, and $m$ for matrix. Then we can solve for the resistivity in the matrix region
\begin{equation}
r_m = \frac{A_t r_t - A_b r_b}{A_m} = x_0(r_t - 3) + 3
\end{equation}
In the buffer region, resistivity is only due to channel walls so that $r_b = 3$. The area of the included buffer region depends on $x_0$ via $A_b = 4(x_0 - 1)$. The area of the matrix region is always $A_m = 4$, and $A_t = A_b + A_m = 4 x_0$. The resulting resistivity in the matrix, $r_m$, includes the contribution from the bodies as well as the channel walls. We can then subtract the contribution from the walls to get $\tilde{r}_m$, the resistivity due to the bodies alone as
\begin{equation}
\tilde{r}_m = r_m - 3 \, .
\end{equation}
Putting everything together, the final formula is
\begin{equation}
\label{resmatrix0}
\tilde{r}_m = x_0 (r_t - 3) \, .
\end{equation}
As convoluted as this calculation seems, it does appear to give convergent numerical results that are {\em independent of the choice of $x_0$} (to the expected order of accuracy).

Using \eqref{rtot} and $\qavg = \frac{2}{3} U$ gives
\begin{equation}
\label{resmatrix}
\tilde{r}_m = \frac{\pup - \pdn}{2 \qavg} - 3 x_0 = \frac{3}{4} \frac{\pup-\pdn}{U} - 3 x_0
\end{equation}

% New May 5, 2022
\subsubsection{Modification for free-slip BCs}
For the free-slip BCs, the resistivity of the buffer region vanishes, $r_b = 0$, giving
\begin{equation}
r_m = \frac{A_t}{A_m} r_t = x_0 r_t
\end{equation}


%-------------------------------------------------------------%
% RESISTIVITY AND DRAG
%-------------------------------------------------------------%
\subsection{Relationship between resistivity and total drag}

Consider the stress tensor
\begin{equation}
\bvec{\sigma} = -p \bvec{I} + \mu \left( \grad \bu + \grad \bu^T \right)
\end{equation}
Below we will set $\mu=1$. The Stokes equation is
\begin{equation}
\grad \cdot \bvec{\sigma} = 0
\end{equation}
By the divergence theorem we have
\begin{equation}
\label{divthm}
0 = \int_{\Omega} \grad \cdot \bvec{\sigma} \, dV 
= \int_{\partial \Omega} \bvec{\sigma} \cdot \nhat \, dS
\end{equation}
for any domain $\Omega$. Now consider $\Omega$ to be a section of the channel $-x_0 < x < x_0$ and spanning the entire width $-w < y < w$. Then the integral on the right-hand-side of \eqref{divthm} has two contributions: first, the total drag arising from all immersed bodies as well as from the top and bottom walls, and second, the difference between the upstream and downstream pressures (we have implicitly assumed that $x_0$ is sufficiently distant from the bodies so that $\grad \bvec{u}$ is nearly zero). Since the total integral above vanishes, we have the approximate relationship
\begin{equation}
F_D \approx 2 w \left( \pup - \pdn \right)
\end{equation}
where $F_D$ is the total drag force from all bodies and from the top and bottom walls. The error in this approximation is due to neglecting the contribution of $\grad \bvec{u}$ in the stress tensor at the upstream and downstream positions. We thus have for the permeability
\begin{equation}
k_{tot} = \frac{2 x_0 \, \qavg}{\pup-\pdn} \approx 4 x_0 w \qavg F_D^{-1}
\end{equation}
and the resistivity
\begin{equation}
r_{tot} = \frac{\pup-\pdn}{2 x_0 \, \qavg} \approx \frac{F_D}{4 x_0 w \qavg} = \frac{3 F_D}{8 x_0 w U}
\end{equation}
Where we have used $\qavg = \frac{2}{3} U$. These are the permeability and resistivity for the entire region $-x_0 < x < x_0$. In particular, the resistivity of the porous matrix will be averaged with the much lower resistivity of the buffer region. To remove the contribution from the buffer region, and thus isolate the resistivity due to the porous matrix of bodies, we can simply set $x_0 = 1$ above. With also setting $w=1$, we get
\begin{equation}
\label{resdrag1}
r_{tot} = \frac{3}{8} \frac{F_D}{U}
\end{equation}
With this simplification, I see a match between measuring resistivity via the drag \eqref{resdrag1} versus measuring resistivity directly via \eqref{resmatrix}.

% New May 5, 2022
\subsubsection{Modification for free-slip BCs}
If instead, we would like to deal with free-slip top and bottom boundaries, then $\qavg = U$, in which case
\begin{equation}
r_{tot} = \frac{1}{4 x_0} \frac{F_D}{U}
\end{equation}
where we have set $w=1$. In the post-process code, I set the default of $x_0 = 2$, which gives
\begin{equation}
r_{tot} = \frac{1}{8} \frac{F_D}{U}
\end{equation}
and this agrees pretty closely with what I see in the data.

Alternatively, if I am looking at only the matrix resistivity $r_m = x_0 r_t$, then
\begin{equation}
r_{m} = \frac{1}{4} \frac{F_D}{U}
\end{equation}

\end{document}
