\documentclass[preprint, 10pt]{elsarticle}

\newcommand{\mcaption}[2]{\caption{\small \em #1}\label{#2}} \newcommand{\secref}[1]{\ref{#1}}

%\usepackage{algorithmic}
%\usepackage{algorithm}
\usepackage{amsfonts}
\usepackage[fleqn,reqno]{amsmath}
\usepackage{amssymb}
%\usepackage{amsthm}
\usepackage[titletoc]{appendix}
\usepackage{array}
%\usepackage{bm}
%\usepackage{caption}
%\usepackage[usenames]{color}
\usepackage{enumitem}
%\usepackage{epsfig}
%\usepackage{fancybox}
\usepackage{filecontents}
\usepackage[top=1.2in,bottom=1.2in,left=1in, right=1in]{geometry}
\usepackage{graphics}
%%\usepackage{ifthen}
\usepackage{lineno}
%\usepackage{mathrsfs}
%\usepackage{mdframed}
%\usepackage{multirow}
%\usepackage{palatino}
%\usepackage{showkeys} %To see the labels for now.  Will remove later
%\usepackage{stmaryrd}
%\usepackage{subfigure}
%\usepackage{paralist}
\usepackage{pgfplots}
%\usepackage{tabularx}
\usepackage{tikz}
\usepackage{todonotes}
\usetikzlibrary{arrows}
\usepackage{comment}

%%%%%%  pdftex  %%%%%%%%%%%%%%%%%%%%%%%%%%%%%%%%%%%%%%%%%%%%%%%%%%%%%%
\usepackage[pagebackref=false,bookmarks=false]{hyperref} 

\hypersetup{
  bookmarksnumbered=true,
  bookmarksopen=false,
  hypertexnames=false,      
  breaklinks=true,          
  unicode=false,
  pdffitwindow=true,        
  pdfnewwindow=true,        
  colorlinks=true,         
  linkcolor=dblue,
  anchorcolor=red,
  citecolor=dorange,
  filecolor=magenta,
  urlcolor=dblue,
  pdfstartview = FitH,
  pdfkeywords = {},
  pdfcreator = {LaTeX with hyperref package}
}



\newcommand{\bd}{{\partial}}
\newcommand{\bigO}{{\mathcal{O}}}
\newcommand{\cc}{{\mathbf{c}}}
\newcommand{\DD}{{\mathcal{D}}}
\newcommand{\eeta}{{\boldsymbol\eta}}
\newcommand{\ff}{{\mathbf{f}}}
\newcommand{\grad}{{\nabla}}
\newcommand{\II}{{\mathbf{I}}}
\newcommand{\iin}{\mathrm{in}}
\newcommand{\llambda}{{\boldsymbol\lambda}}
\newcommand{\nn}{{\mathbf{n}}}
\newcommand{\NN}{{\mathcal{N}}}
\newcommand{\out}{\mathrm{out}}
\newcommand{\rr}{{\mathbf{r}}}
\newcommand{\RR}{{\mathbb{R}}}
\renewcommand{\ss}{{\mathbf{s}}}
\newcommand{\ssigma}{{\boldsymbol\sigma}}
\newcommand{\tar}{\mathrm{tar}}
\newcommand{\uu}{{\mathbf{u}}}
\newcommand{\UU}{{\mathbf{U}}}
\newcommand{\vv}{{\mathbf{v}}}
\newcommand{\xx}{{\mathbf{x}}}
\newcommand{\xxi}{{\boldsymbol{\xi}}}
\newcommand{\yy}{{\mathbf{y}}}

\def\gap{\hspace*{.2in}}

% Derivatives
\newcommand{\pderiv}[2]{\frac{\partial #1}{\partial #2}}
\newcommand{\tderiv}[2]{\frac{d #1}{d #2}}
\newcommand{\ppd}[2]{\frac{\partial^2 #1}{{\partial #2}^2}}

% Nick's commands
\newcommand{\vsp}[1]{\vspace{#1 pc} \noindent}
\newcommand{\abs}[1]{\lvert #1 \rvert}
\newcommand{\mean}[1]{\left< #1 \right>}
\newcommand{\thL}{$\theta$--$L$}
\newcommand{\eps}{\varepsilon}
\newcommand{\Vn}{V_n}
\newcommand{\Vs}{V_s}
\newcommand{\atau}{\abs{\tau}}
\newcommand{\thalpha}{\pderiv{\theta}{\alpha}}
\newcommand{\elfun}{\zeta}
\newcommand{\thhat}{\hat{\theta}}
\newcommand{\Dt}{\Delta t}
\newcommand{\NLterm}{\mathcal{N}}
\newcommand{\Mterm}{\mathcal{M}}
\newcommand{\FourierSum}{ \sum_{k = -N_\iin /2}^{N_\iin /2-1} }
\newcommand{\atausig}{\atau^{(\sigma)}}
\newcommand{\Vnsig}{\Vn^{(\sigma)}}
\newcommand{\Vssig}{\Vs^{(\sigma)}}


\newcommand{\tauD}[1]{\tau_{#1\text{D}}}
\newcommand{\atauD}[1]{\abs{\tau_{#1\text{D}}}}

\begin{document}

\title{Diffusion in Cool Domains with BIEs}

\author[Bryan1]{Jesse Cherry}
\author[Alan]{Alan Lindsay}
\author[Bryan1,Bryan2]{Bryan D.~Quaife}
\author[Nick]{M.~Nicholas J.~Moore}

\address[Nick]{Department of Mathematics, United States Naval Academy,
Annapolis, MD, 21402.}
\address[Bryan1]{Department of Scientific Computing, Florida State University, Tallahassee, FL, 32306.}
\address[Bryan2]{Geophysical Fluid Dynamics Institute, Florida State
University, Tallahassee, FL, 32306.}
\address[Alan]{Department of Applied \& Computational Math \&
Statistics, University of Notre Dame, South Bend, IN, 46656.}

\begin{abstract} 
\end{abstract}

\begin{keyword}
  keyword 1 \sep keyword 2 \sep keyword 3 
\end{keyword}

\maketitle

%%%%%%%%%%%%%%%%%%%%%%%%%%%%%%%%%%%%%%%%%%%%%%%%%%%%%%%%%%%%%%%%%%%%%%%
\section{Introduction\label{s:intro}}

\cite{chi-moo-qua2020}
\cite{qua-moo2018}


%%%%%%%%%%%%%%%%%%%%%%%%%%%%%%%%%%%%%%%%%%%%%%%%%%%%%%%%%%%%%%%%%%%%%%%
\paragraph{Contributions}


\paragraph{Limitations}

\paragraph{Related Work}

%%%%%%%%%%%%%%%%%%%%%%%%%%%%%%%%%%%%%%%%%%%%%%%%%%%%%%%%%%%%%%%%%%%%%%%
\paragraph{Outline of the Paper}

%%%%%%%%%%%%%%%%%%%%%%%%%%%%%%%%%%%%%%%%%%%%%%%%%%%%%%%%%%%%%%%%%%%%%%%
\section{Formulation}
\label{s:formulation}
We are interested in solving the two-dimensional dimensionless diffusion
equation in an unbounded domain $\Omega$ with a homogeneous Dirichlet
boundary condition. For the remainder of the paper, we assume the Peclet
number is one. We an initial condition that is a delta centered at
$\xx^* \in
\Omega$
\begin{subequations}
  \label{eqn:diffusion}
  \begin{alignat}{3}
    \pderiv{c}{t} &= \Delta c, &&\xx \in \Omega,  \\
    c &= 0 &&\xx \in \bd\Omega, \\
    c(\xx,0) &= \delta(\xx - \xx^*_k), \qquad &&\xx \in \Omega.
  \end{alignat}
\end{subequations}
In addition, $c(\xx,t)$ must go to zero as $|\xx|\rightarrow \infty$ for
all time. We note that our solver is easily extended to the case where
the initial condition is a finite sum of weighted delta functions.

To solve~\eqref{eqn:diffusion}, we introduced the Laplace transformed
variable
\begin{align}
  C(\xx,s) = \LL[c](s) = \int_{0}^{\infty} e^{-st} c(\xx,t)\, dt.
\end{align}
Taking the Laplace transform of equation~\eqref{eqn:diffusion}, we
arrive at the elliptic PDE
\begin{subequations}
  \begin{alignat}{3}
    (s - \Delta) C(\xx,s) &= \delta(\xx - \xx^*), \qquad 
      &&\xx \in \Omega, \\
    C(\xx,s) &= 0 &&\xx \in \bd\Omega.
  \end{alignat} 
\end{subequations}
Note that the solution of this PDE depends on $s \in \CC$. For a given
$s$, $C$ can be decomposed as
\begin{align}
  C(\xx,s) = C^{h}(\xx,s) + G(\xx-\xx^*),
\end{align}
where 
\begin{align}
  G(\xx) = \frac{1}{2\pi} K_0\left( \sqrt{s} \|\xx\| \right)
\end{align}
is the fundamental solution of the differential operator $s - \Delta$.
The homogeneous solution $C^{h}(\xx,s)$ must satisfy
\begin{subequations}
  \label{eqn:homoPDE}
  \begin{alignat}{3}
    (s - \Delta) C^{h} &= 0, &&\xx \in \Omega, \\
    C^{h}(\xx,s) &= -G(\xx-\xx^*), \qquad &&\xx \in \bd\Omega.
  \end{alignat}
\end{subequations}
and this guarantees that $C(\xx,s)$ satisfies the homogeneous Dirichlet
boundary condition.

In the following sections we describe integral equation methods to solve
both the homogeneous PDE~\eqref{eqn:homoPDE} and to construct $c(\xx,t)$
by computing a Bromwich integral.


%%%%%%%%%%%%%%%%%%%%%%%%%%%%%%%%%%%%%%%%%%%%%%%%%%%%%%%%%%%%%%%%%%%%%%%
\subsection{Boundary Integral Equation Formulation} 
\label{sec:bies}
To solve~\eqref{eqn:homoPDE}, we use a boundary integral equation (BIE)
formulation. In this manner, we are able to resolve complex geometries,
achieve high-order accuracy, and satisfy far-field boundary conditions.
Because of the Dirichlet boundary condition, we write $C^{h}(\xx,s)$ as
a double-layer potential 
\begin{align}
  \label{eqn:DLP}
  C^{h}(\xx,s) = \DD[\sigma](\xx) := 
    \frac{1}{2\pi} \int_{\bd \Omega} \pderiv{}{\nn}
    G(\xx - \yy) \sigma(\yy)\, ds_\yy, \quad \xx \in \Omega,
\end{align}
where $G$ is the fundamental solution of $s - \Delta$, and $\nn$ is the
unit outward normal of $\bd\Omega$ at $\yy$. The unknown density
function $\sigma$ satisfies the Fredholm second-kind integral equation
\begin{align}
  \label{eqn:BIE}
  \frac{1}{2} \sigma(\xx) + \DD[\sigma](\xx) = -G(\xx - \xx^*),
    \quad \xx \in \bd\Omega.
\end{align}

%%%%%%%%%%%%%%%%%%%%%%%%%%%%%%%%%%%%%%%%%%%%%%%%%%%%%%%%%%%%%%%%%%%%%%%
\subsection{Inverting the Laplace Transform}
\label{sec:bromwich}
Having a method to solve for $C^h(\xx,s)$ for any $s \in \CC$, we
require a method to recover the $c(\xx,t)$. The inverse Laplace
transform, also known as the Bromwich integral, is
\begin{align*}
  \mathcal{L}^{-1}[C^h](\xx,s) = \frac{1}{2\pi i} 
    \int_{\alpha - i\infty} ^{\alpha + i\infty} 
    e^{st} C^h(\xx,s)\, ds, \quad \xx \in \Omega,
\end{align*}
where $\alpha$ is chosen so that it is larger than the real part of
all the singularities of $C(\xx,s)$. The right-most singularity of
$C(\xx,s)$ has real part 0, so we can choose any value $\alpha > 0$.
This contour of integration is called the Bromwich integral and denoted
by $B$.

Along the Bromwich integral, the integrand $e^{st} C^h(s,t)$ oscillates
and decays algebraically. Practically, this means that numerical
quadrature can not be used to compute the Bromwich integral. However,
the contour $B$ can be replaced with any other contour, so long as it
remains to the right of all the singularities of $C^h(s,t)$. There are a
variety of choices~\cite{tre-wei2014}, and we opt to use the Talbot
contour parameterized as
\begin{align*}
  \alpha(\theta) = ...
\end{align*}
Along this contour, the integrand $e^{st} C^h(s,t)$ does not oscillate,
decays exponentially to zero, and is scaled appropriately with respect
to $t$.

%%%%%%%%%%%%%%%%%%%%%%%%%%%%%%%%%%%%%%%%%%%%%%%%%%%%%%%%%%%%%%%%%%%%%%%
\subsection{Computing the Flux Through $\bd\Omega$}
\label{sec:flux}
In many applications, the flux $c$ through $\bd\Omega$
\begin{align}
  \label{eqn:flux}
  J(t) = \int_{\bd\Omega} \pderiv{}{\nn_\xx}c(\xx,t), ds_\xx,
\end{align}
is required, where $\nn_\xx$ is the unit outward normal at $\xx$.
Computing the flux $J$ requires the flux of $C^h(\xx,s)$ through
$\bd\Omega$.


%%%%%%%%%%%%%%%%%%%%%%%%%%%%%%%%%%%%%%%%%%%%%%%%%%%%%%%%%%%%%%%%%%%%%%%
\section{Numerical methods}
\label{s:method}
To solve equation~\eqref{eqn:diffusion}, we require three quadrature
methods. The first is used to solve the second-kind Fredholm integral
equation~\eqref{eqn:BIE} and to evaluate the double-layer
potential~\eqref{eqn:DLP}, and this is described in
Section~\ref{sec:BIE}. The second is to compute the Talbot contour, and
this is described in Section~\ref{sec:talbot}. Finally,
Section~\ref{sec:flux} discusses how the flux is computed through
$\bd\Omega$.

%%%%%%%%%%%%%%%%%%%%%%%%%%%%%%%%%%%%%%%%%%%%%%%%%%%%%%%%%%%%%%%%%%%%%%%
\subsection{Solving the BIE}
\label{sec:BIE}
\begin{itemize}
  \item Collocation method
  \item Limiting value at diagonal
  \item Solved with GMRES
  \item Near-singular integration?
\end{itemize}


%%%%%%%%%%%%%%%%%%%%%%%%%%%%%%%%%%%%%%%%%%%%%%%%%%%%%%%%%%%%%%%%%%%%%%%
\subsection{Computing the Inverse Laplace Transform} 
\label{sec:talbot}
\begin{itemize}
  \item Plots of the different contours
  \item Locations of the singularities of $C^h(s,t)$?
  \item Apply the midpoint rule
  \item Show that it converges with a simple numerical example?
\end{itemize}


%%%%%%%%%%%%%%%%%%%%%%%%%%%%%%%%%%%%%%%%%%%%%%%%%%%%%%%%%%%%%%%%%%%%%%%
\section{Computing the Flux}
\label{sec:flux}
\begin{itemize}
  \item Need normal derivative of double-layer potential
  \item Trick where we add and subtract normal derivative of Laplace DLP
  \item Expected rate of convergence (we still don't fully understand
    this).
\end{itemize}






%%-----------------------------------------------------------------%%
% RESULTS SECTION
%%-----------------------------------------------------------------%%
\section{Numerical Results}


%%%%%%%%%%%%%%%%%%%%%%%%%%%%%%%%%%%%%%%%%%%%%%%%%%%%%%%%%%%%%%%%%%%%%%%
\section{Conclusions\label{s:conclusions}}

%%%%%%%%%%%%%%%%%%%%%%%%%%%%%%%%%%%%%%%%%%%%%%%%%%%%%%%%%%%%%%%%%%%%%%%
\paragraph{\bf Acknowledgments} 

\bibliographystyle{plainnat} 
%\biboptions{sort&compress}

\bibliography{refs}


\end{document}


