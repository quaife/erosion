%\documentclass[12pt]{amsart}
\documentclass[11pt]{article}

\usepackage{color}
\usepackage{fancyhdr}
\usepackage{lastpage}
%\usepackage{amsmath,amsfonts,amssymb}
\usepackage{tabularx}
\usepackage{mathpazo}
\usepackage{enumitem}

%\renewcommand\thesubsection{(\alph{subsection})}
%\renewcommand\thesubsubsection{(\roman{subsubsection})}

\addtolength{\oddsidemargin}{-0.75in}
\addtolength{\evensidemargin}{-0.75in}
\addtolength{\textwidth}{1.5in}
\addtolength{\topmargin}{-0.75in}
\addtolength{\textheight}{1.5in}
% For 11pt size

\pagestyle{fancy}
\lhead{\footnotesize Moore and Quaife}
\chead{\footnotesize Data Management Plan}
\rhead{\footnotesize \thepage{ of }\pageref{LastPage}}
\cfoot{}

\begin{document}
\thispagestyle{fancy}
\subsection*{Products of the Research}
The following types of data will be produced as part of this project:
\begin{itemize}[noitemsep,topsep=0pt,parsep=0pt,partopsep=0pt] 
\item numerical methods for simulating erosion;
\item scientific software for implementing these methods;
\item website for disseminating results;
\item peer-reviewed journal publications.
\end{itemize}

\subsection*{Data Formats}
The software product will consist of (i) exploratory MATLAB or Python
scripts for testing analyzing the algorithms, and (ii) Julia and Fortran
implementations to measure runtime efficiencies on computer
architectures available at Florida State University.  All scripts and
codes will include explanatory comments.  The website will be in ASCII
plaintext using html, php, and javascript.


\subsection*{Access to Data and Data Sharing Practices and Policies}
Source scripts and codes will be made available, often before the
acceptance of the peer-reviewed papers, using the relevant scripts and
libraries.  This software will then be available, without restriction,
to any user who wishes to download, use, or modify it.  The website will
be available on the Internet.  All materials, including movies, images,
and text, will be licensed for re-use with appropriate credit given
(using a license such as the GNU Copyleft license).  Data from the
research will be open source, and there are no significant privacy,
confidentiality, or intellectual property requirements that pertain to
this research. Educational videos will be prodced by the CPALMS
organization and made freely available to the public


\subsection*{Policies for Re-Use, Re-Distribution and Production of
Derivatives}
All data created as a result of this project will be freely available
and usable by the public, and by other researchers.  The only condition
will be a citation to the appropriate publications.  The PIs will
further request, but not require, that users of our software contribute
any changes, improvements, or additions that they make back to the
appropriate open-source software project.  Data or images that are
copyrighted (by a journal, for example) will be marked as such by
watermark or text header in the file.

\subsection*{Archiving of Data} 
Source code will be uploaded to Google Code, BitBucket, GitHub, and
other Cloud-based data providers.  Websites relating to this project,
and all of the data on the aforementioned websites, will be hosted by
Florida State University's (FSU) Information Technology group on
servers that are backed up regularly.  This data will be available to
the public for as long as the PIs remain actively employed at FSU.  If
the PIs leave FSU for any reason, steps will be taken to ensure that
this data remains available, either by moving it to another university
or making it a part of a larger, more persistent project at FSU.


\end{document}
