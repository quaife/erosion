\documentclass[11pt]{article}
\usepackage{color}
\usepackage{fancyhdr}
\usepackage{lastpage}
%\usepackage{amsmath,amsfonts,amssymb}
\usepackage{tabularx}
\usepackage{todonotes}
\usepackage{mathpazo}

%\renewcommand\thesubsection{(\alph{subsection})}
%\renewcommand\thesubsubsection{(\roman{subsubsection})}

\addtolength{\oddsidemargin}{-0.75in}
\addtolength{\evensidemargin}{-0.75in}
\addtolength{\textwidth}{1.5in}
\addtolength{\topmargin}{-0.75in}
\addtolength{\textheight}{1.5in}
% For 11pt size

\pagestyle{fancy}
\lhead{\footnotesize Bryan D.~Quaife}
\chead{\footnotesize FSU Scope of Work}
\rhead{\footnotesize \thepage{ of }\pageref{LastPage}}
\cfoot{}



\begin{document}
\section*{Personnel}
The research at FSU will be performed by PI Quaife and Ph.D.~student
Jake Cherry for the duration of the award.

\section*{Deliverables}
\paragraph{1.~Simulating mechanical and chemical erosion, as well as
transport, including sedimentation and compaction, of eroding bodies.}
Transport of mobile eroding and dissolving bodies will be simulated.
Dissolution will be simulated using advection-diffusion solvers for
complex geometries that will combine semi-Lagrangian methods with a
diffusion solver for the Laplace transformed concentration.
    
\paragraph{2.~Simulating the transport of finite Peclet number
concentrations in eroded porous medium.} The algorithms to simulate
dissolution will also be used to simulate transport of chemical species
in eroded and dissolved geometries.

\paragraph{3.~Improving numerical methods including fast summation
methods, preconditioners, and quadrature:} Novel computational tools
including quadrature, preconditioning, and fast summation methods will
be used to accurately, efficiently, and stably simulate hundreds or
thousands of mobile eroded or dissolving bodies.

\section*{Timeline}
Deliverable 1 will be a focus of Year 1. Deliverables 2 and 3 will be
developed over all three years of the project.

\section*{Coordination with co-PI Moore}
Throughout the project, the FSU personnel will work closely with co-PI
Moore. This will include frequent emails and phone conversations, 1--2
Zoom and/or Google Meet meetings per month. The code and papers will be
maintained on GitHub.





\end{document}
