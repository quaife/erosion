\documentclass[11pt]{article}
\usepackage{color}
\usepackage{fancyhdr}
\usepackage{lastpage}
%\usepackage{amsmath,amsfonts,amssymb}
\usepackage{tabularx}
\usepackage{todonotes}
\usepackage{mathpazo}

%\renewcommand\thesubsection{(\alph{subsection})}
%\renewcommand\thesubsubsection{(\roman{subsubsection})}

\addtolength{\oddsidemargin}{-0.75in}
\addtolength{\evensidemargin}{-0.75in}
\addtolength{\textwidth}{1.5in}
\addtolength{\topmargin}{-0.75in}
\addtolength{\textheight}{1.5in}
% For 11pt size

\pagestyle{fancy}
\lhead{\footnotesize M.J.N.~Moore}
\chead{\footnotesize USNA Scope of Work}
\rhead{\footnotesize \thepage{ of }\pageref{LastPage}}
\cfoot{}



\begin{document}
\section*{Personnel}
The research at the US Naval Academy will be performed by co-PI Moore.

\section*{Deliverables}
The deliverables from the US Naval Academy will include algorithms,
computational tools, peer-reviewed publications, presentations at
conferences, and outreach. The deliverables will be centered around the
following tasks taken directly from the Project Description. 

\paragraph{1: Characterizing anisotropy, permeability, and pore size
dynamics to inform homogenized models such as Darcy’s relation.}
Resistivity of eroded and dissolved bodies will be used define the
anisotropy, permeability, and to infer horizontal permeability.

\paragraph{2: Developing and validating new parameterizations using machine
learning techniques and numerical simulations.} Large batches of
high-fidelity simulations will be used to train machine learning and
deep-neural networks. These new models will parameterize the effects of
erosion on the evolution macroscopic medium properties of the flow.


\section*{Timeline}
Years 1 and 2 will focus on deliverable 1. Years 2 and 3 will focus on
deliverable 2.




\end{document}
