\documentclass[11pt]{article}
\usepackage{color}
\usepackage{fancyhdr}
\usepackage{lastpage}
%\usepackage{amsmath,amsfonts,amssymb}
\usepackage{tabularx}
\usepackage{todonotes}
\usepackage{mathpazo}

%\renewcommand\thesubsection{(\alph{subsection})}
%\renewcommand\thesubsubsection{(\roman{subsubsection})}

\addtolength{\oddsidemargin}{-0.75in}
\addtolength{\evensidemargin}{-0.75in}
\addtolength{\textwidth}{1.5in}
\addtolength{\topmargin}{-0.75in}
\addtolength{\textheight}{1.5in}
% For 11pt size

\pagestyle{fancy}
\lhead{\footnotesize M.J.N.~Moore}
\chead{\footnotesize USNA Scope of Work}
\rhead{\footnotesize \thepage{ of }\pageref{LastPage}}
\cfoot{}



\begin{document}
\section*{Personnel}
The research at the US Naval Academy will be performed by co-PI Moore.

\section*{Deliverables}
The deliverables from the US Naval Academy will include algorithms,
computational tools, peer-reviewed publications, presentations at
conferences, and outreach. The deliverables will be centered around the
following tasks taken directly from the Project Description. 

\paragraph{1: Characterizing permeability, anisotropy, and pore size
dynamics to inform Darcy-based homogenized models.}
The high-fidelity numerical simulations permit precise measurements of coarse-grained quantities such as permeability and anisotropy, as they depend on medium features such as porosity and microstructure geometry. Statistical measurements of these properties taken from large batches of the numerical simulations will inform reduced models based on Darcy's law with variable permeability to represent unresolved pore-structure that varies dynamically in time due to erosion. 
%Resistivity of eroded and dissolved bodies will be used define the anisotropy, permeability, and to infer horizontal permeability.

\paragraph{2: Developing and validating new parameterizations using machine
learning techniques and numerical simulations.} The batches of high-fidelity simulations will be used to train machine learning and deep-neural networks. These new models will parameterize the effects of erosion on the evolution of macroscopic properties. Once trained, the networks will enable predictions of how medium properties respond dynamically to changes in environmental conditions without reliance on the (more expensive) direct physical simulations.


\section*{Timeline}
Years 1 and 2 will focus on deliverable 1. Years 2 and 3 will focus on
deliverable 2.

\section*{Coordination with PI Quaife}
Throughout the project, co-PI Moore will work closely with both PI Quaife and the funded Ph.D.~student at FSU, Jake Cherry. This will include frequent emails and phone conversations, 1--2 Zoom and/or Google Meet meetings per month. The code and papers will be maintained on GitHub.


\end{document}
