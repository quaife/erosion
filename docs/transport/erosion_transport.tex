\documentclass[preprint, 10pt]{elsarticle}

\newcommand{\mcaption}[2]{\caption{\small \em #1}\label{#2}} \newcommand{\secref}[1]{\ref{#1}}

%\usepackage{algorithmic}
%\usepackage{algorithm}
\usepackage{amsfonts}
\usepackage[fleqn,reqno]{amsmath}
\usepackage{amssymb}
%\usepackage{amsthm}
\usepackage[titletoc]{appendix}
\usepackage{array}
%\usepackage{bm}
%\usepackage{caption}
%\usepackage[usenames]{color}
\usepackage{enumitem}
%\usepackage{epsfig}
%\usepackage{fancybox}
\usepackage{filecontents}
\usepackage[top=1.2in,bottom=1.2in,left=1in, right=1in]{geometry}
\usepackage{graphics}
%%\usepackage{ifthen}
\usepackage{lineno}
%\usepackage{mathrsfs}
%\usepackage{mdframed}
%\usepackage{multirow}
%\usepackage{palatino}
%\usepackage{showkeys} %To see the labels for now.  Will remove later
%\usepackage{stmaryrd}
%\usepackage{subfigure}
\usepackage{subcaption}
%\usepackage{paralist}
\usepackage{pgfplots}
%\usepackage{tabularx}
\usepackage{tikz}
\usepackage{todonotes}
\usetikzlibrary{arrows}
\usepackage{comment}
\usepackage{float}

%%%%%%  pdftex  %%%%%%%%%%%%%%%%%%%%%%%%%%%%%%%%%%%%%%%%%%%%%%%%%%%%%%
\usepackage[pagebackref=false,bookmarks=false]{hyperref} 

\hypersetup{
  bookmarksnumbered=true,
  bookmarksopen=false,
  hypertexnames=false,      
  breaklinks=true,          
  unicode=false,
  pdffitwindow=true,        
  pdfnewwindow=true,        
  colorlinks=true,         
  linkcolor=dblue,
  anchorcolor=red,
  citecolor=dorange,
  filecolor=magenta,
  urlcolor=dblue,
  pdfstartview = FitH,
  pdfkeywords = {},
  pdfcreator = {LaTeX with hyperref package}
}



\newcommand{\bd}{{\partial}}
\newcommand{\bigO}{{\mathcal{O}}}
\newcommand{\cc}{{\mathbf{c}}}
\newcommand{\CC}{{\mathbb{C}}}
\newcommand{\DD}{{\mathcal{D}}}
\newcommand{\DDD}{{\boldsymbol{\mathcal D}}}
\newcommand{\eeta}{{\boldsymbol\eta}}
\newcommand{\ff}{{\mathbf{f}}}
\newcommand{\grad}{{\nabla}}
\newcommand{\II}{{\mathbf{I}}}
\newcommand{\iin}{\mathrm{in}}
\newcommand{\llambda}{{\boldsymbol\lambda}}
\newcommand{\nn}{{\mathbf{n}}}
\newcommand{\NN}{{\mathcal{N}}}
\newcommand{\out}{\mathrm{out}}
\newcommand{\rr}{{\mathbf{r}}}
\renewcommand{\Re}{{\operatorname{Re}}}
\renewcommand{\Im}{{\operatorname{Im}}}
\newcommand{\RR}{{\mathbb{R}}}
\renewcommand{\ss}{{\mathbf{s}}}
\newcommand{\ssigma}{{\boldsymbol\sigma}}
\newcommand{\tar}{\mathrm{tar}}
\newcommand{\bary}{\mathrm{bary}}
\newcommand{\trap}{\mathrm{trap}}
\newcommand{\uu}{{\mathbf{u}}}
\newcommand{\UU}{{\mathbf{U}}}
\newcommand{\vv}{{\mathbf{v}}}
\newcommand{\xx}{{\mathbf{x}}}
\newcommand{\xxi}{{\boldsymbol{\xi}}}
\newcommand{\yy}{{\mathbf{y}}}
\newcommand{\mcaption}[2]{\caption{\small \em #1}\label{#2}} \newcommand{\secref}[1]{\ref{#1}}


\def\gap{\hspace*{.2in}}

% Derivatives
\newcommand{\pderiv}[2]{\frac{\partial #1}{\partial #2}}
\newcommand{\tderiv}[2]{\frac{d #1}{d #2}}
\newcommand{\ppd}[2]{\frac{\partial^2 #1}{{\partial #2}^2}}
\newcommand{\pdi}[2]{\partial {#1}/\partial {#2}}

% Nick's commands
\newcommand{\vsp}[1]{\vspace{#1 pc} \noindent}
\newcommand{\abs}[1]{\lvert #1 \rvert}
\newcommand{\mean}[1]{\left< #1 \right>}
\newcommand{\thL}{$\theta$--$L$}
\newcommand{\eps}{\varepsilon}
\newcommand{\Vn}{V_\nn}
\newcommand{\Vs}{V_\ss}
\newcommand{\atau}{\abs{\tau}}
\newcommand{\thalpha}{\pderiv{\theta}{\alpha}}
\newcommand{\elfun}{\zeta}
\newcommand{\thhat}{\hat{\theta}}
\newcommand{\Dt}{\Delta t}
\newcommand{\NLterm}{\mathcal{N}}
\newcommand{\Mterm}{\mathcal{M}}
\newcommand{\FourierSum}{ \sum_{k = -N_\iin /2}^{N_\iin /2-1} }
\newcommand{\atausig}{\atau^{(\sigma)}}
\newcommand{\Vnsig}{\Vn^{(\sigma)}}
\newcommand{\Vssig}{\Vs^{(\sigma)}}
\newcommand{\umax}{U}
\newcommand{\areaconst}{c_A}
\newcommand{\FFD}{\mathbf{F}_{D}}
\newcommand{\oangle}{\beta}
\newcommand{\thb}{\theta_0}
\newcommand{\CE}{C_E}


\begin{document}

\title{Viscous Transport in Eroding Porous Media}


% OTHER TITLE POSSIBILITIES
% Viscous Erosion in Porous Media (change 'in' to 'of'?)
% A fast and accurate method for viscous erosion of a 2D porous medium
% A boundary-integral method for viscous erosion of a porous medium
% A computational framework to simulate the viscous erosion of a porous medium

\author[SH]{Shang-Huan Chiu}
\author[Nick]{M.~Nicholas J.~Moore}
\author[Bryan]{Bryan D.~Quaife}

\address[SH]{Department of Scientific Computing, Florida State
University, Tallahassee, FL, 32306.}
\address[Nick]{Department of Mathematics and Geophysical Fluid Dynamics Institute, Florida State University, Tallahassee, FL, 32306.}
\address[Bryan]{Department of Scientific Computing and Geophysical Fluid Dynamics Institute, Florida State University, Tallahassee, FL, 32306.}

\begin{abstract} 
  An abstract
\end{abstract}

\begin{keyword}
  Keyword 1 \sep Keyword 2 \sep Keyword 3 
\end{keyword}

\maketitle

%%%%%%%%%%%%%%%%%%%%%%%%%%%%%%%%%%%%%%%%%%%%%%%%%%%%%%%%%%%%%%%%%%%%%%%
\section{Introduction\label{s:intro}}
A continuation of the methods paper~\cite{qua-moo2018}.


%%%%%%%%%%%%%%%%%%%%%%%%%%%%%%%%%%%%%%%%%%%%%%%%%%%%%%%%%%%%%%%%%%%%%%%
\paragraph{Contributions}
\begin{itemize}
  \item Use novel near-singular Barycentric integration scheme for
    Stokes equation outlined in~\cite{bar-wu-vee2015}.
  \item Need to compute one additional derivative since we need the
    derivative of the velocity to find the vorticity and stress tensor
  \item Tracer trajectories at different time steps
  \item Compute the anomalous diffusion rate using similar techniques as
    the paper with Pietro~\cite{dea-qua-bir-jua2018}
  \item Need a novel way to do time stepping through the complex
    geometry
\end{itemize}

\paragraph{Limitations}
\begin{itemize}
  \item Bodies are still pinned
  \item Bodies are still two dimensional
\end{itemize}

\paragraph{Related Work}

%%%%%%%%%%%%%%%%%%%%%%%%%%%%%%%%%%%%%%%%%%%%%%%%%%%%%%%%%%%%%%%%%%%%%%%
\paragraph{Outline of the Paper}

%%%%%%%%%%%%%%%%%%%%%%%%%%%%%%%%%%%%%%%%%%%%%%%%%%%%%%%%%%%%%%%%%%%%%%%
\section{Formulation}
\label{s:formulation}
Similar to the JCP methods paper.  However, need to put everything into
complex variables formulation so that we can use Barycentric ideas.


%%%%%%%%%%%%%%%%%%%%%%%%%%%%%%%%%%%%%%%%%%%%%%%%%%%%%%%%%%%%%%%%%%%%%%%
\subsection{Boundary integral equation formulation} 
\label{sec:bies}

%%%%%%%%%%%%%%%%%%%%%%%%%%%%%%%%%%%%%%%%%%%%%%%%%%%%%%%%%%%%%%%%%%%%%%%
\subsection{Computing the shear stress}
\label{sec:shearStressLP}

%%%%%%%%%%%%%%%%%%%%%%%%%%%%%%%%%%%%%%%%%%%%%%%%%%%%%%%%%%%%%%%%%%%%%%%
\subsection{Boundary evolution in the {\thL} framework} 
\label{sec:thetaL}


%%%%%%%%%%%%%%%%%%%%%%%%%%%%%%%%%%%%%%%%%%%%%%%%%%%%%%%%%%%%%%%%%%%%%%%
\section{Numerical methods}
\label{s:method}

%%%%%%%%%%%%%%%%%%%%%%%%%%%%%%%%%%%%%%%%%%%%%%%%%%%%%%%%%%%%%%%%%%%%%%%
\subsection{Fast Summation Methods}
\label{sec:FMM}

%%%%%%%%%%%%%%%%%%%%%%%%%%%%%%%%%%%%%%%%%%%%%%%%%%%%%%%%%%%%%%%%%%%%%%%
\subsection{Solving the BIE}
\label{sec:BIE}

 
%%%%%%%%%%%%%%%%%%%%%%%%%%%%%%%%%%%%%%%%%%%%%%%%%%%%%%%%%%%%%%%%%%%%%%%
\subsection{Shear stress}
\label{sec:shearStress}


%%%%%%%%%%%%%%%%%%%%%%%%%%%%%%%%%%%%%%%%%%%%%%%%%%%%%%%%%%%%%%%%%%%%%%%
%% TIME-STEPPING %%
\subsection{Time-stepping with the {\thL} method} 
\label{sec:timeStepping}


%%%%%%%%%%%%%%%%%%%%%%%%%%%%%%%%%%%%%%%%%%%%%%%%%%%%%%%%%%%%%%%%%%%%%%%
\section{Post processing quantities of interest}
\label{s:qoi}

%%%%%%%%%%%%%%%%%%%%%%%%%%%%%%%%%%%%%%%%%%%%%%%%%%%%%%%%%%%%%%%%%%%%%%%
\subsection{Vorticity}
%%%%%%%%%%%%%%%%%%%%%%%%%%%%%%%%%%%%%%%%%%%%%%%%%%%%%%%%%%%%%%%%%%%%%%%
\subsection{Pressure}
\label{sec:pressure}

%%%%%%%%%%%%%%%%%%%%%%%%%%%%%%%%%%%%%%%%%%%%%%%%%%%%%%%%%%%%%%%%%%%%%%%
\subsection{Drag}
\label{sec:drag}

%%%%%%%%%%%%%%%%%%%%%%%%%%%%%%%%%%%%%%%%%%%%%%%%%%%%%%%%%%%%%%%%%%%%%%%
\subsection{Near-singular integration}
\label{sec:NSI}

%%-----------------------------------------------------------------%%
% RESULTS SECTION
%%-----------------------------------------------------------------%%
\section{Results: Single-body erosion}
\label{s:SingleResults}

\subsection{A single body eroding in Stokes flow}

\subsection{Scaling laws for area and drag}
\label{sec:scaling}


%%%%%%%%%%%%%%%%%%%%%%%%%%%%%%%%%%%%%%%%%%%%%%%%%%%%%%%%%%%%%%%%%%%%%%%
\section{Conclusions\label{s:conclusions}}


%%%%%%%%%%%%%%%%%%%%%%%%%%%%%%%%%%%%%%%%%%%%%%%%%%%%%%%%%%%%%%%%%%%%%%%
\paragraph{\bf Acknowledgments} We would like to thank Manas Rachh for
supplying the FMM for the Stokes double-layer potential. BQ and NM were
supported by Florida State University startup funds and Simons
Foundation Mathematics and Physical Sciences-Collaboration Grants for
Mathematicians.

\bibliographystyle{plainnat} 
\biboptions{sort&compress}
\bibliography{refs}


\end{document}


