\documentclass[11pt]{article}


\usepackage{todonotes}
\usepackage{fullpage}
\usepackage{amsmath,amsfonts,amssymb,stmaryrd}
\usepackage{color}
\newcommand{\comment}[1]{{\color{blue} #1}}

\begin{document}

\noindent
Dear Silas Alben,
\\ \\
\noindent
Thank you for handling the manuscript, and we would like to thank the
reviewer for their positive and constructive comments.  The attached
manuscript addresses all the reviewer's comments and the result is what
we believe is a stronger manuscript.  We also note that formatting
issues were resolved in the bibliography, and minor grammatical errors
were corrected throughout. An itemized list of the changes addressing
the reviewer's comments are below.  \\ \\
\noindent
Sincerely,
\\
\noindent
Shang-Huan Chiu, Nick Moore, and Bryan Quaife

\section*{Response to Reviewer 3}


\noindent
\comment{Pores in natural porous media care of three-dimensional
geometry.  The authors may include some discussion on how to extend
their numerical method to three-dimensional problems.  In addition, they
may also discuss the limitations of the erosion analysis in
2-dimensional flow to be applied for three-dimensional flow. If the
erosion process in three-dimensional flow is very different from that in
two-dimensional flow, the analysis of this paper may have less relevant
to the analysis of erosion in natural porous media.}
\begin{itemize}
  \item The reviewer is correct that by assuming the geometry is
    two-dimensional, we have limited the applicability for natural
    porous media. There are several numerical challenges to simulate
    three-dimensional erosion, with the most significant being the
    $\theta$--$L$ formulation, quadrature methods, spectral
    representations of the geometry, and the boundary integral equation
    solver.

  \item In three-dimensions, others have developed methods to extend the
    $\theta$--$L$ formulation to remove stiffness in interfacial flow,
    spectrally represented geometries, and efficiently solve BIEs.  The
    challenge with these methods is their computational cost. To
    consider suspensions with the resolution and number of grains that
    we consider, the BIE would most likely require a parallel
    implementation that would run on high-performance computing
    architectures. At this point, this is not the goal of our work.
    Rather, we are interested in using very high-order methods to
    provide an analysis of the effects of erosion on transport which
    seems to be missing from the literature.

  \item The Barycentric quadrature method relies on a complex variable
    formulation, so it is inherently two-dimensional. However, there are
    other quadrature methods that can be applied in three dimensions,
    but they are also computationally expensive. For example, if the
    geometry is represented with spherical harmonics, a
    Gaussian-Legendre quadrature method requires $\mathcal{O}(p^3)$
    operations, where $p$ is the highest-order spherical harmonic.

  \item The manuscript has been modified as discussed in the response to
    the reviewer's second comment.
\end{itemize}
 
\noindent
\comment{In the numerical examples of the paper the grains are all
separated. In natural porous media, grains are compressed to form grain
contacts. The authors may include discussion on numerical challenges in
studying erosion process in natural porous media.}
\begin{itemize}
  \item The reviewer is correct that porous media grains are compressed
    and form grain contact. This is the result of processes such as
    sedimentation and transport of the grains themselves. This, again,
    is a limitation of our method since we only consider immobile
    bodies.

  \item We plan to consider mobile bodies in the future, and this will
    require additional computational effort.

  \item To the best of our knowledge, nobody has studied the effect
    erosion has on transport through a porous media. Therefore, as a
    first step, we limit ourselves to two-dimensional suspensions of
    immobile bodies, and we plan to remove these limitations in the
    future. In the Introduction, we have added the following paragraph:
    \begin{quotation}
      \noindent
      Since high-order methods have never been used to study transport
      in eroding porous media, we consider, as a first step,
      two-dimensional and immobile grains. These assumptions allow us to
      simulate large densely-packed suspensions (up to $100$ in this
      manuscript), and to clearly interpret the results. More realistic
      three-dimensional and mobile configurations would require
      overcoming a few challenges. First, a three-dimensional extension
      of the $\theta$--$L$ formulation exists (Ambrose {\em et
      al.}~2013), but the Barycentric quadrature method used here relies
      on a complex variable formulation of the BIE, making it inherently
      two-dimensional. Therefore, a different quadrature method would
      be required. Additional challenges include spectral
      representations of three-dimensional geometries and efficiently
      solving discretizations of the BIE. While high-order methods for
      three-dimensional BIEs have been developed, their computational
      cost limits the number of bodies that can be practically
      simulated. Lastly, mechanisms such as sedimentation play a role in
      realistic media and affect the pore structure. Simulating mobile
      grains introduces the additional challenge of resolving contact,
      which we plan to address in future work.
    \end{quotation}
\end{itemize}



\end{document}
