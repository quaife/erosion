\documentclass[11pt]{article}


\usepackage{todonotes}
\usepackage{fullpage}
\usepackage{amsmath,amsfonts,amssymb,stmaryrd}
\usepackage{color}
\newcommand{\comment}[1]{{\color{blue} #1}}

\begin{document}

\noindent
Dear Silas Alben,
\\ \\
\noindent
Thank you for handling the manuscript, and we would like to thank the
reviewer for their positive and constructive comments.  The attached
manuscript addresses all the reviewer's comments and the result is what
we believe is a stronger manuscript.  We also note that formatting
issues were resolved in the bibliography. An itemized list of the
changes addressing the reviewer's comments are below.
\\ \\
\noindent
Sincerely,
\\
\noindent
Shang-Huan Chiu, Nick Moore, and Bryan Quaife

\section*{Response to Reviewer 3}


\noindent
\comment{Pores in natural porous media care of three-dimensional
geometry.  The authors may include some discussion on how to extend
their numerical method to three-dimensional problems.  In addition, they
may also discuss the limitations of the erosion analysis in
2-dimensional flow to be applied for three-dimensional flow. If the
erosion process in three-dimensional flow is very different from that in
two-dimensional flow, the analysis of this paper may have less relevant
to the analysis of erosion in natural porous media.}
\begin{itemize}
  \item The reviewer is correct that by assuming the geometry is
    two-dimensional, we have limited the applicability for natural
    porous media. There are several numerical challenges to simulate
    three-dimensional erosion, with the most significant being the
    $\theta-L$ formulation, quadrature method, spectral representation
    of the geometry, and the boundary integral equation solver.

  \item Others have developed three-dimensional numerical methods for
    removing stiffness in interfacial flow ($\theta-L$ formulation),
    spectrally representing geometries, and efficiently solving BIEs.
    The challenge with these methods is their computational cost. To
    consider suspensions with the resolution and number of grains that
    we consider, the BIE would most likely require a parallel
    implementation that would run on high-performance computing
    architectures. At this point, this is not the goal of our work.

  \item The Barycentric quadrature method relies on a complex variable
    formulation, so it is inherently two-dimensional. However, there are
    other quadrature methods that can be applied in three dimensions,
    but they are also computationally expensive.

  \item Our response to the reviewer's second comment points to a
    paragraph that has been added to the Introduction.


\end{itemize}

\noindent
\comment{In the numerical examples of the paper the grains are all
separated. In natural porous media, grains are compressed to form grain
contacts. The authors may include discussion on numerical challenges in
studying erosion process in natural porous media.}
\begin{itemize}
  \item The reviewer is correct that porous media grains are compressed
    and form grain contact. This is the result of processes such as
    sedimentation and transport of the grains themselves. This, again,
    is a limitation of our method since we only consider immobile
    bodies.

  \item We plan to consider mobile bodies in the future, and this will
    require a significant computational effort.

  \item To the best of our knowledge, nobody has studied the effect
    erosion has on transport through a porous media. Therefore, as a
    first step, we limit ourselves to two-dimensional suspensions of
    immobile bodies, and we plan to remove these limitations in the
    future. In the Introduction, we have added the following paragraph:
    \begin{quotation}
      \noindent
      While natural porous media are three-dimensional, there are
      several challenges that must be overcome to extend our high-order
      methods to three dimensions.  A three-dimensional version of a
      {$\theta-L$} formulation exists (Ambrose {\em et al.} 2013), but
      the Barycentric quadrature method relies upon a complex variable
      formulation of the BIE, making it inherently two-dimensional.
      Therefore, a different quadrature method is required. Additional
      challenges of three-dimensional simulations include spectral
      representations of the geometry and efficiently solving
      discretizations of the BIE.  While high-order methods for
      three-dimensional BIEs have been developed, their computational
      cost limits the number of bodies that can be practically
      simulated. Another limitation of this work is the assumption of
      immobile grains.  Mechanisms such as sedimentation play a role in
      real applications, and this will effect the transport properties.
      Simulating mobile grains introduces the additional challenge of
      resolving contact.  Since high-order methods have never been used
      to study transport in eroded porous media, as a first step we
      consider the case  with two-dimensional immobile grains. This
      assumption allows for a clear interpretation of the results.  For
      example, in this work we are able to simulate a large
      densely-packed suspension (up to $100$ in this manuscript).


    \end{quotation}
\end{itemize}



\end{document}
