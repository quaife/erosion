\documentclass[11pt]{article}


\usepackage{todonotes}
\usepackage{fullpage}
\usepackage{amsmath,amsfonts,amssymb,stmaryrd}
\usepackage{color}
\newcommand{\comment}[1]{{\color{blue} #1}}

\begin{document}

\noindent
Dear Silas Alben,
\\ \\
\noindent
Thank you for handling the manuscript, and we would like to thank the
reviewer for their positive and constructive comments.  The attached
manuscript addresses all the reviewer's comments and the result is what
we believe is a stronger manuscript.  We also note that formatting
issues were resolved in the bibliography. An itemized list of the
changes addressing the reviewer's comments are below.
\\ \\
\noindent
Sincerely,
\\
\noindent
Shang-Huan Chiu, Nick Moore, and Bryan Quaife

\section*{Response to Reviewer 1}

\noindent
\comment{The Barycentric formulae (5.9 \& 5.10) for the first
derivatives of Cauchy integral are incorrect. They are NOT true
barycentric form and straightforward numerical experiments will show
that all accuracies are lost up to the boundary using these formulae,
even if $v(x)$ is computed to high accuracy using the formulae (5.6 \&
5.7). (See e.g.~Sec 3.3 of Barnett et.~al.~2015 for more details.)}
\begin{itemize}
  \item response
\end{itemize}

\noindent
\comment{Likewise, the second derivative ``barycentric formulae" (5.15 \& 5.18) are incorrect, either. So the authors will need to explain how they managed to get high accuracies with the stated formulae?}
\begin{itemize}
  \item response
\end{itemize}

\noindent
\comment{In my opinion, the claim of having developed a ``new quadrature
method" is too strong. After all, there is nothing new about the
barycentric quadrature itself. Rather, I suggest that maybe a more
appropriate claim would be that the authors have developed a ``new
numerical scheme" for the Stokes porous media flow, where the accurate
quadrature is one ingredient in this scheme.}
\begin{itemize}
  \item response
\end{itemize}


\noindent
\comment{The transient increase of tortuosity is interesting, but the
explanation in Sec 6.4 is not satisfying. The authors proposed that the
increase of tortuosity can be explained by the sudden increase of
$x$-velocity when a grain vanishes. But in formula (4.4) or (4.5), the
$x$-velocity appears both in the numerator and the denominator, which
seems to me that the increase would just cancel. Then, as the authors
have pointed out, wouldn't the straighter streamlines resulted from the
vanishing of grains have decreased the tortuosity?}
\begin{itemize}
  \item response
\end{itemize}

\noindent
\comment{The porosity-tortuosity models (4.6c \& 4.6d) are not
explicitly compared in the results section. How much worse are they
comparing to the other models?}
\begin{itemize}
  \item response
\end{itemize}

\noindent
\comment{The coefficient CE in equation (2.1) is undefined.}
\begin{itemize}
  \item Thank you for pointing this out. We also realized that $\mu$ is
  undefined. We now state:
  \begin{quotation}
    \noindent
    Parameters include the fluid viscosity $\mu$ and the
    material-dependent erosion constant $C_E$. By non-dimensionalizing
    equations (2.1) (Quaife \& Moore 2018), we set both these parameters
    to one.
  \end{quotation}
\end{itemize}

\noindent
\comment{Should the shear stress formula (2.2) contain viscosity in it
based on the governing equation (2.1)?}
\begin{itemize}
  \item In general, it would. However, since we have non-dimensionalized
    following our previous work, the viscosity is assumed to be 1.
\end{itemize}

\noindent
\comment{In Figure 5, do the colors represent vorticity itself or the
log of the abs vorticity? (Which would be consistent with the colors in
Figure 4.)}
\begin{itemize}
  \item The color represents the logarithm of the absolute value of the
    vorticity.
  \item We have updated the caption to read:
    \begin{quotation}
    \noindent
      The eroded bodies from figure 4 at time $t = 0.1$. The color is
      the logarithm of the magnitude of the vorticity. The initial
      distances from the body to the solid wall are: (a) $h$, (b) $h/2$,
      and (c) $h/10$.
    \end{quotation}
  \item We have also updated the text to read:
    \begin{quotation}
    \noindent
      To visualize the flow, in figure 5 we plot the logarithm of the
      absolute value of the vorticity from the final time step in figure
      4.
    \end{quotation}
\end{itemize}

\noindent
\comment{When first introduced in Sec 4.3, the ``pore sizes" are not
clearly defined. (My first impression was that pore size = portion of an
edge of the Delaunay diagram. Only at the end in Sec 6.4 do we know that
it is defined as the closest distance between grains.)}
\begin{itemize}
  \item In Figure 3, we clarify the definition by adding the statement:
    {\em represented by the length of the red line segments}.
  \item In the text, we now state:
    \begin{quotation}
    \noindent
      Finally, we define the pore sizes to be the minimum distance
      between all pairs of neighboring grains, and these distances are
      represented by the red line segments in figure 3(a).
    \end{quotation}
\end{itemize}

\noindent
\comment{``Porosity" is undefined throughout the paper.}
\begin{itemize}
  \item We have now formally defined the porosity by rewriting the
    description of coarse-grained models as:
  \begin{quotation}
  \noindent
    Examples of coarse-grained models for porous media flow include
    tortuosities (Matyka et al.~2008; Duda et al.~2011; Koponen et
    al.~1996), geometry connectivity (Knudby \& Carrera 2005), anomalous
    dispersion (Dentz et al.~ 2004), and permeability-porosity
    relationships (Dardis \& McCloskey 1998; Carman 1937), where the
    porosity is the fraction of the geometry that is fluidized.
  \end{quotation}


\end{itemize}

\noindent
\comment{``Area fraction" is plotted in (and only in) Figure 7, while
the rest of the paper used porosity. Maybe plotting porosity here would
give a more consistent and smoother reading experience?}
\begin{itemize}
  \item We examined the area fraction and flow rate in figure 7 so that
    it could be compared to results from our previous work. The goal is
    to demonstrate that we now able to resolve smaller area fractions
    and correspondingly lower flow rates.

  \item We have updated the text to read:
    \begin{quotation}
      \noindent
      In our previous work (Quaife \& Moore 2018), we investigated the
      effects of erosion on the area fraction and flow rate of two
      different geometries. Here, we do the same analysis, but with the
      denser packing of the grains.
    \end{quotation}

\end{itemize}


\end{document}
