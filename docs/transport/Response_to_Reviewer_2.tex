\documentclass[11pt]{article}


\usepackage{todonotes}
\usepackage{fullpage}
\usepackage{amsmath,amsfonts,amssymb,stmaryrd}
\usepackage{color}
\newcommand{\comment}[1]{{\color{blue} #1}}

\begin{document}

\noindent
Dear Silas Alben,
\\ \\
\noindent
Thank you for handling the manuscript, and we would like to thank the
reviewer for their positive and constructive comments.  The attached
manuscript addresses all the reviewer's comments and the result is what
we believe is a stronger manuscript.  We also note that formatting
issues were resolved in the bibliography. An itemized list of the
changes addressing the reviewer's comments are below.
\\ \\
\noindent
Sincerely,
\\
\noindent
Shang-Huan Chiu, Nick Moore, and Bryan Quaife

\section*{Response to Reviewer 2}

\noindent
\comment{equations 2.1.  I am confused as to how the no-slip condition
is implemented as zero velocity when in fact the interface is receding
at a rate Vn.}
\begin{itemize}
  \item This comes from a quasi-steady approximation that we have used
    in our previous work as well as by others. For example, see Section
    5.2 from Mitchell and Spagnolie's JCP paper {\em A generalized
    traction integral equation for Stokes flow, with applications to
    near-wall particle mobility and viscous erosion}.

  \item In Section 2, we now state:
    \begin{quotation}
      \noindent
      Since the rate of erosion is much slower than the background flow,
      a quasi-steady approximation justifies the no-slip boundary
      condition.
    \end{quotation}
\end{itemize}

\noindent
\comment{end of section 4.1.  It seems that this protocol (matching
velocity but not position for reinsertion) results in each particle
moving through a different landscape.  Can the authors comment on the
significance of this?}
\begin{itemize}
  \item The reinsertion approach we take has been shown to not introduce
    any significantly artificial decorrelation.

  \item In Section 4.1, We have added that
    \begin{quotation}
      \noindent
      This method of reinsertion does not artificially decorrelate the
      particle velocities (Puyguiraud et al.~2019), and this allows us
      to reliably use the statistical moments (4.2) to characterize the
      anomalous dispersion.
    \end{quotation}

\end{itemize}

\noindent
\comment{p.10, end of section 4.  The authors do not explicitly say how
pore size is calculated.  Is it simply the length of the portion of a
Delaunay side that lives outside the grains?}
\begin{itemize}
 \item In Figure 3, we clarify the definition by adding the statement:
    {\em represented by the length of the red line segments}.
  \item In the text, we now state:
    \begin{quotation}
    \noindent
      Finally, we define the pore sizes to be the minimum distance
      between all pairs of neighboring grains, and these distances are
      represented by the red line segments in figure 3(a).
    \end{quotation}
\end{itemize}

\noindent
\comment{p.17 ``single body close to a wall", the 'experiment' mentioned
here is, I believe, a numerical `experiment'.  It might be good to
indicate that clearly.}
\begin{itemize}
  \item The reviewer is correct that this is a numerical experiment.

  \item There were other instances in the paper where we claimed to be
    performing an `experiment'. We have clarified each of the instances
    by referring to them as a `numerical experiment'.
\end{itemize}

\noindent
\comment{sections 6.2 and 6.3.  How were these geometries of 20 grains
chosen?  Both the sizes and positions of these grains vary between
examples - it's not just the porosity that is different.}
\begin{itemize}
  \item The initial placement and radii of the bodies are chosen with a
    randomized method. The algorithm is descried at the start of Section
    6.

  \item We have updated the start of section 6.2 to read:
    \begin{quotation}
      \noindent
      We randomly initialize 20 eroding grains with an initial porosity
      of $\phi = 37.68\%$ and the Hagen-Poiseuille flow
      $\mathbf{U}(\mathbf{x}) = U (1-y^2, 0)$ imposed on $\Gamma$
    \end{quotation}


  \item We have updated the start of section 6.3 to read:
    \begin{quotation}
      \noindent
      We consider a second example with of 20 randomly initialized
      eroding bodies, but with a smaller initial porosity of $\phi =
      30.67\%$.
    \end{quotation}


  \item We have updated the start of section 6.4 to read:
    \begin{quotation}
      \noindent
        As a final example, we consider 100 randomly initialized eroding
        bodies with an initial porosity near $50\%$.
    \end{quotation}

\end{itemize}




\noindent
\comment{Figure 8.  Due to what I presume is an artifact of plotting, it
looks like lots of streamlines suddenly end, which is confusing.}
\begin{itemize}
  \item In figure 8, we freeze an eroded geometry and plot the
    trajectory of passive tracers at five different time steps. Since
    the flow is steady and the tracers are passive, the trajectories
    trace out streamlines.

  \item We have updated the caption to read:
    \begin{quotation}
      \noindent
      200 particle trajectories in the second geometry from figure 6.
      The particles are initially equispaced at $(−1,y)$, where $y \in
      (−1, 1)$. The first five snapshots are equispaced in time. The
      bottom right frame is a magnification of the fifth snapshot, but
      with additional trajectories.
    \end{quotation}

  \item We have also updated the paragraph describing figure 8 to
    clarify that it is showing tracer trajectories that trace out
    streamlines.
\end{itemize}

\noindent
\comment{Figure 9.  While the sharp features in subfigure (b) are
apparently finite cusps, they raise the question of whether any infinite
dwell times (divergent $\tau$) are ever observed in these simulations -
i.e.~are particles ever trapped inside recirculating zones?}
\begin{itemize}
  \item Thank you for pointing out this possibility. The
    recirculation zones are close to the eroding bodies and are
    typically very small. Because we measure the local tortuosity by
    initiating streamlines significantly far from the grains, each of
    them connect the inlet to the outlet.

  \item We have updated the text to read:
    \begin{quotation}
      \noindent
      We note that if a streamline originates in a recirculation zone,
      which is possible in two dimensions (Higdon 1985), then the local
      tortuosity will be infinite. However, in each of our numerical
      experiments, all the streamlines connect the inlet to the outlet.
      Having only finite local tortuosities, the hydraulic tortuosity is
      defined by taking the average over all points on the inlet
      cross-section
    \end{quotation}
\end{itemize}

\noindent
\comment{p.26 bottom paragraph typo: ``an ... media"}
\begin{itemize}
  \item This typo has been fixed
\end{itemize}



\end{document}
