  % JFM
%\NeedsTeXFormat{LaTeX2e}
%\documentclass {jfm}

% PHYSICS OF FLUIDS
%\documentclass[aip, pof, reprint]{revtex4-1} 
% the files aip4-1.rtx seems to be necessary
\documentclass[11pt]{article}

% PACKAGES, minimal set?
\usepackage{graphicx}
\usepackage{amsmath, amssymb, amsfonts}
%\usepackage[justification=RaggedRight]{caption}


%%%%%%%%%%%%%%%%%%%%%%%%%%%%%%%%%%%%%%
%% LATEX DEFINITIONS
%%%%%%%%%%%%%%%%%%%%%%%%%%%%%%%%%%%%%%

% Partial differentiation, standard fraction notation
\newcommand{\pd}[2]    { \frac{\partial #1} {\partial #2} }
\newcommand{\ppd}[2]  { \frac{\partial^2 #1}{{\partial #2}^2} }
\newcommand{\pdpd}[3]  { \frac{\partial^2 #1}{\partial #2 \partial #3 } }
\newcommand{\pppd}[2]  { \frac{\partial^3 #1}{{\partial #2}^3} }
% Differentiation
\newcommand{\pdi}[2] { {\partial_#2} #1 }
\newcommand{\td}[2] { \frac{d #1} { d #2 } }
\newcommand{\grad}{\nabla}
\newcommand {\Lap} {\grad^2}
% Vectors and operators
\newcommand{\bvec}[1]{\ensuremath{\boldsymbol{#1}}}
\newcommand{\abs}[1]{\left| #1 \right|}
\newcommand{\norm}[1]{\left\| #1 \right\|}
\newcommand{\avg}[1] { \left\langle {#1} \right\rangle}
\newcommand{\dash} {\text{-}}
\newcommand{\conj}[1] {\overline{#1}}

% Stuff
\newcommand {\bu} {\bvec{u}}
\newcommand{\nhat}{\hat{\bvec{n}}}
\newcommand{\shat}{\hat{\bvec{s}}}
\newcommand{\Vn}{V_n}


\newcommand{\Rey}{ \text{Re} }
% Domains
\newcommand {\bdry} {\partial B}
\newcommand {\onBdry} {\text{on } \bdry }
\newcommand {\xfar} {\text{as } \abs{\bvec{x}} \to \infty }
\newcommand {\nfar} {\text{as } \, \ny / \dchar \to \infty }



\newcommand{\nn}{\mathbf{n}}
\newcommand{\rr}{\mathbf{r}}
\newcommand{\ssigma}{\boldsymbol{\sigma}}










%%%%%%%%%%%%%%%%%%%%%%%%%%%%%%%%%%%%%%
%% TITLE
%%%%%%%%%%%%%%%%%%%%%%%%%%%%%%%%%%%%%%
\begin{document}
\title{Brief notes on erosion}
\author{}
\maketitle

%\begin{abstract}
%\end{abstract}

\section{Governing equations}
The deformation, or rate-of-strain, tensor is given by
\begin{equation}
\bvec{E} = \frac{1}{2} \left( \grad \bu + \grad \bu^T \right)
\end{equation}
where $\bu = (u,v)$ is the velocity field. The stress tensor is given by
\begin{equation}
\bvec{\sigma} = -p \bvec{I} + 2 \mu \bvec{E} = -p \bvec{I} + \mu \left( \grad \bu + \grad \bu^T \right)
\end{equation}
The surface traction is given by
\begin{equation}
\bvec{f} = \bvec{\sigma} \cdot \nhat 	\qquad \onBdry
\end{equation}
where $\nhat$ is the (outward) unit normal and $\bdry$ is the solid
boundary. The shear stress $\tau$ is defined as the tangential
component of the traction, i.e.~
\begin{equation}
\tau = \bvec{f} \cdot \shat 
= \mu \left( \grad \bu + \grad \bu^T \right) \cdot \nhat \cdot \shat	\qquad \onBdry
\end{equation}
where $\shat$ is the unit tangent vector. It turns out that the shear
stress is simply given by the surface vorticity
\begin{equation}
\tau = \mu \left( u_y - v_x \right) \qquad \onBdry
\end{equation}
which can be shown by a direct calculation, using the no-slip boundary
conditions (which implies that $\pdi{u}{s} = \pdi{v}{s} = 0$ on
$\bdry$).  The interface velocity of the eroding body is linearly
proportional to the absolute shear stress
\begin{equation}
\Vn = -C \abs{\tau}
\end{equation}
where $C$ is a constant.


\section{Layer potentials}
The kernel of the Stokes double-layer potential is
\begin{equation}
  \frac{1}{\pi} \frac{\rr \otimes \nn}{\rho^{2}} 
  \frac{\rr \otimes \ssigma}{\rho^{2}} \rr.
\end{equation}

The deformation tensor is
\begin{align*}
  \bvec{E} &= \frac{1}{2} \left( \grad \bu + \grad \bu^T \right) \\
  &= \frac{1}{\pi} \frac{(\rr \cdot \nn)(\rr \cdot \ssigma)}{\rho^{4}}I 
   + \frac{1}{2\pi} \frac{(\rr \cdot \ssigma)}{\rho^{4}}
    (\rr \otimes \nn + \nn \otimes \rr) \\
  &+ \frac{1}{2\pi} \frac{(\rr \cdot \nn)}{\rho^{4}} 
    (\rr \otimes \ssigma + \ssigma \otimes \rr) 
   - \frac{4}{\pi}\frac{(\rr \cdot \nn)(\rr \cdot \ssigma)}{\rho^{6}}
    (\rr \otimes \rr).
\end{align*}

The shear stress $\tau$ is
\begin{align*}
  \tau &= \frac{1}{\pi}\frac{(\rr \cdot \ssigma)}{\rho^{4}}
      (\rr + (\nn \cdot \rr)\nn) \cdot \shat
   + \frac{1}{\pi}\frac{(\rr \cdot \nn)}{\rho^{4}}
      ((\ssigma \cdot \nn)\rr + (\rr \cdot \nn)\ssigma) \cdot \shat \\
  &- \frac{8}{\pi}\frac{(\rr \cdot \nn)(\rr \cdot \ssigma)}{\rho^{6}}
      (\rr \cdot \nn)(\rr \cdot \shat) \\
  &= \frac{1}{\pi}\frac{(\rr \cdot \ssigma)(\rr \cdot \shat)}{\rho^{4}}
   + \frac{1}{\pi}\frac{(\rr \cdot \nn)(\ssigma \cdot \nn)(\rr \cdot \shat)}{\rho^{4}} \\
  &+ \frac{1}{\pi}\frac{(\rr \cdot \nn)^{2}(\ssigma \cdot \shat)}{\rho^{4}}
   - \frac{8}{\pi}\frac{(\rr \cdot \nn)^{2}(\rr \cdot \ssigma)(\rr \cdot \shat)}{\rho^{6}}
\end{align*}


\end{document}
