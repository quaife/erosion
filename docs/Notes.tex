\documentclass[11pt]{article}

% PACKAGES
\usepackage{graphicx, amsmath, amssymb, amsfonts, mathtools, mathrsfs, color}
\usepackage{comment, enumerate, tabularx}
\usepackage{natbib, hyperref, url}
%\usepackage[justification=RaggedRight]{caption}

%%%%%%%%%%%%%%%%%%%%%%%%%%%%%%%%%%%%%%
%% LATEX DEFINITIONS
%%%%%%%%%%%%%%%%%%%%%%%%%%%%%%%%%%%%%%

% Basic editing
\newcommand{\tocite}{{\color{blue}(to cite)}}
\newcommand{\vsp}[1]{\vspace{#1 pc} \noindent}
\newcommand{\np}{\newpage \noindent}
% Derivatives
\newcommand{\pd}[2]    { \frac{\partial #1} {\partial #2} }
\newcommand{\ppd}[2]  { \frac{\partial^2 #1}{{\partial #2}^2} }
\newcommand{\pdi}[2] { {\partial_#2} #1 }
\newcommand{\td}[2] { \frac{d #1} { d #2 } }
\newcommand{\grad}{\nabla}
\newcommand {\Lap} {\grad^2}
% Vectors and operators
\newcommand{\bvec}[1]{\ensuremath{\boldsymbol{#1}}}
\newcommand{\abs}[1]{\left| #1 \right|}
\newcommand{\norm}[1]{\left\| #1 \right\|}
\newcommand{\mean}[1]{\left< #1 \right>}
% Specific variables
\newcommand {\bu} {\bvec{u}}
\newcommand{\nhat}{\hat{\bvec{n}}}
\newcommand{\shat}{\hat{\bvec{s}}}
\newcommand{\eps}{\varepsilon}
\newcommand{\atau}{\abs{\tau}}
\newcommand{\thh}{\hat{\theta}}
\newcommand{\GG}{\mathcal{G}}
\newcommand{\Dt}{\Delta t}
\newcommand{\X}{\bvec{X}}
% Domains
\newcommand {\bdry} {\partial B}
\newcommand {\onBdry} {\text{on } \bdry }
\newcommand {\xfar} {\text{as } \abs{\bvec{x}} \to \infty }


%%%%%%%%%%%%%%%%%%%%%%%%%%%%%%%%%%%%%%
%% TITLE
%%%%%%%%%%%%%%%%%%%%%%%%%%%%%%%%%%%%%%
\begin{document}
\title{Brief notes on erosion}
\author{}
\maketitle

%\begin{abstract}
%\end{abstract}

\section{Shear stress and erosion law}
The deformation, or rate-of-strain, tensor is given by
\begin{equation}
\bvec{E} = \frac{1}{2} \left( \grad \bu + \grad \bu^T \right)
\end{equation}
where $\bu = (u,v)$ is the velocity field. The stress tensor is given by
\begin{equation}
\bvec{\sigma} = -p \bvec{I} + 2 \mu \bvec{E} = -p \bvec{I} + \mu \left( \grad \bu + \grad \bu^T \right)
\end{equation}
The surface traction is given by
\begin{equation}
\bvec{f} = \bvec{\sigma} \cdot \nhat 	\qquad \onBdry
\end{equation}
where $\nhat$ is the (outward) unit normal and $\bdry$ is the solid boundary. The shear stress $\tau$ is defined as the tangential component of the traction, i.e.~
\begin{equation}
\tau = \bvec{f} \cdot \shat 
= \mu \left( \grad \bu + \grad \bu^T \right) \cdot \nhat \cdot \shat	\qquad \onBdry
\end{equation}
where $\shat$ is the unit tangent vector. It turns out that the shear stress is simply given by the surface vorticity
\begin{equation}
\tau = \mu \left( u_y - v_x \right) \qquad \onBdry
\end{equation}
which can be shown by a direct calculation, using the no-slip boundary conditions (which implies that $\pdi{u}{s} = \pdi{v}{s} = 0$ on $\bdry$).
The interface velocity of the eroding body is linearly proportional to the absolute shear stress
\begin{equation}
V_n = -C \atau
\end{equation}
where $C$ is a constant.

\section{Scaling law and interface smoothing}
For Stokes flow, the shear stress scales like
\begin{equation}
\tau \sim \mu V/L
\end{equation}
where $V$ and $L$ are a characteristic velocity and length scale respectively. To evolve the interface in our numerical method, we will probably need to introduce some smoothing to mitigate high frequencies. A common approach is to introduce a small curvature term into the interface velocity, sometimes called the method of vanishing viscosity. The most basic implementation would be
\begin{equation}
V_n = - C \atau - \eps \kappa
\end{equation}
where $\eps$ is a small parameter and $\kappa$ is the surface curvature. However, we have two additional preferences
\begin{enumerate}
\item As the body changes size, we would like the influence of the curvature term relative to the shear stress to remain the same.
\item We prefer the curvature term not to change the size of the body.
\end{enumerate}
Fortunately, the first requirement is already satisfied since both $\tau$ and $\kappa$ scale like $1/L$. To satisfy the second requirement, we replace $\kappa$ with this same quantity minus its mean. To allow for some generality, we allow the curvature term by to be scaled by a power of $L$, giving
\begin{equation}
V_n = - C \atau - \eps L^{\beta} \left(\kappa - \frac{2 \pi}{L} \right)
\end{equation}
Thus, for the present case of Stokes flow $\beta=0$, but for the previous case of high-Reynolds-number flow $\beta=1/2$.

\section{Numerical interface evolution}

To evolve the interface numerically, we use the $\theta$-$L$ method. Rather than re-derive the formulas, I will use the convention from Shelley 1994, with the curve parameterized in the CCW direction and the normal vector pointing {\em inward}. With this convention, the normal interface velocity is the negative of what it was previously, giving
\begin{equation}
V_n = C \atau + \eps L^{\beta} \left(\kappa - \frac{2 \pi}{L} \right)
\end{equation}
Recall that for Stokes flow, $\beta = 0$.

Now, the interface $\X = (x,y)$ evolves according to
\begin{equation}
\pd{\bvec{X}}{t} = V_n \nhat + V_s \shat
\end{equation}
where $\nhat = (-y_s, x_s)$ is the inward pointing normal vector and $\shat = (x_s, y_s)$ is the tangential vector (running in the CCW direction). Here, we have introduced a tangential velocity $V_s$ that will be chosen to keep the curve in the equal-arclength parameterization. We will frequently use a normalized arclength $\alpha = s/L$, where $L$ is the total length of the curve.

We introduce the tangent angle $\theta$, defined as
\begin{equation}
\tan \theta = y_s/x_s = y_\alpha / x_\alpha
\end{equation}
Remember the identity that $x_s^2 + y_s^2 = 1$ as this is very useful.
The curvature is given by
\begin{equation}
\kappa = x_s y_{ss} - x_{ss} y_s = \pd{\theta}{s} = L^{-1} \pd{\theta}{\alpha}
\end{equation}
We are using the standard convention, where a convex shape, like a circle, has positive curvature. The normal velocity now can be written as
\begin{equation}
V_n = C \atau + \eps L^{\beta-1} \left(\pd{\theta}{\alpha} - 2 \pi \right)
\end{equation}

We now discuss how to choose $V_s$ to remain in the equal-arclength frame of reference. We will use the term
\begin{equation}
\gamma = \pd{\theta}{\alpha} V_n
\end{equation}
Then the necessary tangential velocity is determined by
\begin{equation}
\td{V_s}{\alpha} = \gamma - \mean{\gamma}
\end{equation}
where $\mean{}$ indicates the mean value.

We now discuss how to evolve $\theta$ and $L$ forward in time according to the interface velocity given above. The resulting evolution equations are
\begin{align}
& \td{L}{t} = - \mean{\gamma} \\
& \pd{\theta}{t} = L^{-1} \left( \pd{V_n}{\alpha} + \pd{\theta}{\alpha} V_s \right)
\end{align}
Then, using the above $V_n$, the equation for $\theta$ simplifies to
\begin{equation}
\pd{\theta}{t} = \eps L^{\beta-2} \ppd{\theta}{\alpha} + L^{-1} \left( C \pd{\atau}{\alpha} + \pd{\theta}{\alpha} V_s \right)
\end{equation}

We now discuss the numerical methods used to solve the above evolution equations. For $L$, we use an explicit method
\begin{equation}
L^{(n+1)} = L^{(n)} + \frac{\Dt}{2} \left( 3 M^{(n)} - M^{(n-1)} \right)
\end{equation}
where
\begin{equation}
M = \td{L}{t} = -\mean{\gamma}
\end{equation}
For $\theta$, we use a spectral method, giving
\begin{equation}
\td{\thh_k}{t} + 4 \pi^2 k^2 \eps L^{\beta-2} \thh_k = \hat{N}_k
\end{equation}
where the nonlinear terms are grouped together as
\begin{equation}
N = L^{-1} \left( C \pd{\atau}{\alpha} + \pd{\theta}{\alpha} V_s \right)
\end{equation}
We evolve with the integrating-factor method. When recast in terms of Gaussian filters, this gives
\begin{equation}
\thh^{(n+1)} = \GG(\theta^{(n)}, \sigma_1) 
+ \frac{\Dt}{2} \left( 3 \GG(N^{(n)}, \sigma_1) - \GG(N^{(n-1)}), \sigma_2 \right)
\end{equation}
where
\begin{align}
& \sigma_1 = 2 \pi \sqrt{\eps \Dt} \sqrt{(L^{n})^{\beta-2} + (L^{n+1})^{\beta-2} } 	\\
& \sigma_2 = 2 \pi \sqrt{\eps \Dt} \sqrt{(L^{n-1})^{\beta-2} + 2 (L^{n})^{\beta-2} + (L^{n+1})^{\beta-2} }
\end{align}


\subsection{Motion of a reference point}

\subsubsection{The center of mass}
Consider the body's center of mass
\begin{equation}
(x_{cm}, y_{cm}) = \frac{1}{A} \iint (x, y) \, dA
\end{equation}
I did a calculation to determine the motion of the center of mass, given the interface velocity. The result that I get is
\begin{equation}
\td{}{t} (x_{cm}, y_{cm})  = -\frac{1}{A} \int (x(s), y(s)) V_n(s) \, ds
\end{equation}
I believe the sign is correct for $\nhat$ the inward pointing normal, so that $V_n$ is positive if the body is shrinking.

To calculate the center of mass, I can use the divergence theorem
\begin{equation}
\iint \grad \cdot \bvec{F} \, dA = \int \bvec{F} \cdot \nhat \, ds
\end{equation}
Take $\bvec{F} = \frac{1}{2} (x^2, y^2)$, so that $\grad \cdot \bvec{F} = (x,y)$. Then
\begin{equation}
(x_{cm}, y_{cm} ) = \frac{1}{2 A} \int x^2 n_x + y^2 n_y \, ds
\end{equation}
To calculate the total area, $A$, I can use the divergence theorem again, to get
\begin{equation}
A = \int x n_x ds = \int y n_y ds
\end{equation}

\subsubsection{The surface-average coordinates}
I believe this will be even simpler. Consider the average of the surface coordinates
\begin{equation}
(x_a, y_a) = \int (x(\alpha), y(\alpha)) \, d\alpha
\end{equation}
I believe that this point moves according to
\begin{equation}
\td{}{t} (x_a, y_a) = \int V_n \, (n_x, n_y) \, d\alpha
\end{equation}



\end{document}