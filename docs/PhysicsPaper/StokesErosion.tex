\documentclass[3p]{elsarticle}

% APS journals: PRL and PRF
% Note: All AIP and APS journals use the revtex package; both PRL and PRF are APS.
%\documentclass[preprint, superscriptaddress, notitlepage]{revtex4-1}

% Physica D is elsevier 
%\documentclass[review]{elsarticle}

%\documentclass{jfm}
% NOTE: jfm documentclass needed for \upi
%---------------------------------------------------%
% Packages from JFM 2020
%\usepackage[top=1.2in,bottom=1.2in,left=1in, right=1in]{geometry}
\usepackage{amsfonts, amssymb, array}
\usepackage[fleqn,reqno]{amsmath}
\usepackage{graphics, graphicx, subfigure}
% Replaced subcaption with subfigure
\usepackage{todonotes, comment, soul, float}
%---------------------------------------------------%

%^^^^^^^^^^^^^^^^^^^^^^^^^^^^^^^^^^^^^^^^^^^^^^^^^^^^^^^^^^^^%
% COMMANDS
% Basic editing
\newcommand{\vsp}[1]{\vspace{#1 pc} \noindent}
\newcommand{\np}{\newpage \noindent}
\newcommand{\tocite}{ {\color{blue}(to cite)} }
\newcommand{\nick}[1]{ {\color{red} #1} }
% For real and imaginary, could use \Re or \Im, or \mathcal{R}, or \text{Re}
\newcommand{\Real}{\Re}
\newcommand{\Imag}{\Im}

%---------------------------------------------------%
% From JFM 2020
\newcommand{\bd}{{\partial}}
\newcommand{\bigO}{{\mathcal{O}}}
\newcommand{\cc}{{\mathbf{c}}}
\newcommand{\CC}{{\mathbb{C}}}
\newcommand{\DD}{{\mathcal{D}}}
\newcommand{\DDD}{{\boldsymbol{\mathcal D}}}
\newcommand{\eeta}{{\boldsymbol\eta}}
\newcommand{\ff}{{\mathbf{f}}}
\newcommand{\grad}{{\nabla}}
\newcommand{\II}{{\mathbf{I}}}
\newcommand{\iin}{\mathrm{in}}
\newcommand{\llambda}{{\boldsymbol\lambda}}
\newcommand{\nn}{{\mathbf{n}}}
\newcommand{\NN}{{\mathcal{N}}}
\newcommand{\out}{\mathrm{out}}
\newcommand{\rr}{{\mathbf{r}}}
\renewcommand{\Re}{{\operatorname{Re}}}
\renewcommand{\Im}{{\operatorname{Im}}}
\newcommand{\RR}{{\mathbb{R}}}
\renewcommand{\ss}{{\mathbf{s}}}
\newcommand{\tar}{\mathrm{tar}}
\newcommand{\bary}{\mathrm{bary}}
\newcommand{\trap}{\mathrm{trap}}
\newcommand{\uu}{{\mathbf{u}}}
\newcommand{\UU}{{\mathbf{U}}}
\newcommand{\vv}{{\mathbf{v}}}
\newcommand{\xx}{{\mathbf{x}}}
\newcommand{\xxi}{{\boldsymbol{\xi}}}
\newcommand{\yy}{{\mathbf{y}}}
\newcommand{\mcaption}[2]{\caption{\small \em #1}\label{#2}} 
\newcommand{\secref}[1]{\ref{#1}}
\def\gap{\hspace*{.2in}}
% Nick's below
\newcommand{\pderiv}[2]{\frac{\partial #1}{\partial #2}}
\newcommand{\ppd}[2]{\frac{\partial^2 #1}{{\partial #2}^2}}
\newcommand{\abs}[1]{\left| #1 \right|}
\newcommand{\Vn}{V_\nn}
\newcommand{\Vs}{V_\ss}
\newcommand{\CE}{C_E}
%---------------------------------------------------%

% From Nick's Latex Notes
\newcommand{\bvec}[1]{\mathbf{#1}}
%{\ensuremath{\boldsymbol{#1}}}
\newcommand {\bq} {\bvec{q}}
\newcommand{\mean}[1]{\left< #1 \right>}
\newcommand{\qavg}{\bar{q}}
\newcommand{\pavg}{\bar{p}}
\newcommand{\pup}{p_u}
\newcommand{\pdn}{p_d}
\newcommand{\stress}{{\boldsymbol \sigma}}
\newcommand{\FD}{\bvec{F}_d}
\newcommand{\ex}{ {\bvec{e}}_1}

% New
\newcommand{\anis}{\mathcal{A}}


%^^^^^^^^^^^^^^^^^^^^^^^^^^^^^^^^^^^^^^^^^^^^^^^^^^^^^^^^^^^^%
% TITLE AUTHORS ABSTRACT
\begin{document}
\title{Fluid-mechanical erosion creates anisotropic porous media}
%\title{Fluid-mechanical erosion generates anisotropy in porous media}
%\title{Erosion leads to anisotropy in a porous medium}
%\title{The development of anisotropy in an eroding porous medium}


\author{Nicholas J.~Moore}
%\affiliation{Department of Mathematics and Geophysical Fluid Dynamics Institute, Florida State University, Tallahassee, FL, 32306.}
\author{Bryan D.~Quaife}
%\affiliation{Department of Scientific Computing and Geophysical Fluid Dynamics Institute, Florida State University, Tallahassee, FL, 32306.}
\author{Shang-Huan Chiu}
%\affiliation{Department of Scientific Computing, Florida State University, Tallahassee, FL, 32306.}

\begin{abstract}
We numerically simulate the erosion of a porous medium due to an internally flowing fluid.  The solid constituents of the porous medium erode under the action of surface shear stress. As the particles disintegrate, they elongate in the direction of the flow, giving rise to anisotropic conductivity of the porous medium.
\end{abstract}
\maketitle
%^^^^^^^^^^^^^^^^^^^^^^^^^^^^^^^^^^^^^^^^^^^^^^^^^^^^^^^^^^^^%

%Shear stress on solid surfaces drives the erosion of porous material

% dematerialize, disintegrate

\section{Introduction}

Effects of flow-induced erosion are visible across a range of scales in nature, from massive geological structures, to mesoscopic patterns, and down to the tiny granular constituents that comprise a porous medium. In the case of a porous medium specifically, it has been long recognized that bulk properties of mediums encountered in nature are typically anisotropic, so that the medium allows flow in a preferred direction more easily. This anisotropy is commonly attributed to how grains are deposited, with their longest dimension parallel to the settling bed, so that they allow flow preferentially in the horizontal direction. Controlled experiments, however, have not been performed to test this hypothesis. Here, we use highly-accurate numerical simulations to examine an alternative, and possibly complementary, mechanism: namely, that the flow-induced erosion of the medium's solid constituents contributes to its overall anisotropy.

Our method merges highly-efficient and highly-accurate boundary-integral methods with stable interface evolution methods to simulate the erosion of many solid-bodies in the Stokes regime of groundwater flow \cite{quaife2018boundary}. Originally inspired by related work in the high-Reynolds-number regime \cite{Ristroph2012, Moore2013, Huang2015, MooreCPAM2017}, our method is documented, validated, benchmarked in \cite{quaife2018boundary}, and can simulate the erosion of order hundreds of solid bodies (please see \cite{Mitchell2016} for related work). The recent incorporation of barycentric interpolation methods permits simulation of tightly packed configurations (cite new paper with Shang). 

% Need to include descriptor 'high-fidelity' for our simulation.

Figure \ref{FigVortVel} shows an example illustration of 80 eroding bodies


%Maybe cite: \cite{Rycroft2016}
% Cite paper by Ladd "Wormhole formation in dissolving fractures"

%^^^^^^^^^^^^^^^^^^^^^^^^^^^^^^%
\begin{figure*}[] \centering
        \subfigure[ Vorticity field]{ \label{FigVort}
            \includegraphics[width=0.9 \textwidth]{figs/80circ8vort.pdf} } \\
        \subfigure[ Velocity magnitude]{ \label{FigVel}
            \includegraphics[width=0.9 \textwidth]{figs/80circ8vel.pdf} }
   \caption{80 eroding bodies}
   \label{FigVortVel}
\end{figure*}
%^^^^^^^^^^^^^^^^^^^^^^^^^^^^^^%
% Data from 80circ8


\begin{comment}
%^^^^^^^^^^^^^^^^^^^^^^^^^^^^^^%
\begin{figure*}%[htbp]
\centering \label{fig1}
\includegraphics[width = 0.9 \textwidth]{./figs/80circ8vel.pdf}
\caption{caption}
\end{figure*}
 %^^^^^^^^^^^^^^^^^^^^^^^^^^^^^^%
\end{comment}



% NOTE: THIS SECTION HAS BEEN PLAGIARIZED FROM JFM 2020
% SO WE NEED TO PARAPHRASE IT
%\begin{comment}
%-----------------------------------------------------------------------------------------------%
\section{Governing Equations}
\label{sec:formulation}
%-----------------------------------------------------------------------------------------------%

Consider an incompressible, Stokes flow inside a domain $\Omega$ containing $M$ erodable bodies. We take the outer boundary $\Gamma$ to be a slightly smoothed version of the boundary of $[-3,3] \times [-1,1]$. The fluid flow is primarily from left to right, so that the inlet and outlet are located at approximately $x=-3$ and $+3$ respectively (the actual locations are slightly curved versions of these vertical lines due to domain smoothing). The erodable bodies, with boundaries $\gamma_\ell$, $\ell = 1,\ldots,M$, all sit inside of the central region $[-1,1] \times [-1,1]$. The boundary of the fluid domain is thus $\bd \Omega = \Gamma \cup \gamma_1 \cup \cdots \cup \gamma_M$. 
The empty space to the left and right of $[-1,1] \times [-1,1]$ serve as buffer regions to allow the flow profile imposed at the inlet and outlet to gradually adjust to the presence of the bodies. The equations governing the velocity $\uu$ and pressure $p$ of the fluid consist of the incompressible, Stokes equations coupled to boundary conditions:
\begin{equation}
\label{eqn:StokesEq}
  \begin{split}
    \mu \Delta \uu = \grad p, &\hspace{20pt} \xx \in \Omega, \gap 
      &&\mbox{\em conservation of momentum}, \\
    \grad \cdot \uu = 0, &\hspace{20pt} \xx \in \Omega, \gap 
      &&\mbox{\em conservation of mass}, \\
    \uu = \mathbf{0}, &\hspace{20pt} \xx \in \gamma, \gap 
      &&\mbox{\em no slip on the erodable bodies}, \\
    \uu = \UU, &\hspace{20pt} \xx \in \Gamma, \gap 
      &&\mbox{\em outer wall velocity}.
  \end{split}
\end{equation}
Above, $\UU$ represents the fluid velocity imposed along the outer boundary $\Gamma$, in particular at the inlet and outlet, as well as along the top and bottom walls. In this work, we impose a {\em uniform} flow profile along $\Gamma$, i.e.~$\UU = (U,0)$, although other choices are possible, for example a Poiseuille profile as employed in previous work \cite{quaife2018boundary, chiu2020viscous}. The advantages of the uniform profile are: (1) it will simplify the calculation of porous-medium properties, such as permeability and anisotropy, that will be described later; and (2), it may more realistically model the flow impinging upon a porous medium. We will allow the imposed flow speed to change with time $U = U(t)$ to enforce, for example, a desired pressure drop across the flow cell. To nondimensionalize the above system, we set the fluid viscosity to unity, $\mu = 1$.

The embedded bodies may erode in response to the shear stresses induced by the intervening fluid flow. Erosion typically occurs over much longer timescales than does the fluid flow, permitting a {\em quasi-steady} approximation. In this approximation, the configuration of bodies is held fixed in order to compute the {\em steady} Stokes flow determined by \eqref{eqn:StokesEq}, and then this flow field determines the stresses acting to erode each body. We employ an erosion law in which the local rate of material loss is linearly proportional to the magnitude of the shear stress $\tau$ acting on the surface (CITE previous erosion work). The material loss gives rise to an inward velocity of the solid surface, $\Vn$, pointing in the direction normal to the surface. The erosion law is thus expressed as
\begin{align}
\Vn = \CE \, \abs{\tau}, 
	&\hspace{20pt} \xx \in \gamma, &&\mbox{\em erosion model}, \\
\tau = -(\nabla \uu + \nabla \uu^T) \nn \cdot \ss
	&\hspace{20pt} \xx \in \gamma, &&\mbox{\em shear stress}.
  \label{eqn:shearStress}
\end{align}
where $\nn$ is a unit normal vector pointing into each body, $\ss$ is
a unit tangent vector pointing in the counterclockwise direction, and $\CE$ is a material-dependent erosion constant.

% SNIPPETS 
%The strength of $\UU$ is adjusted at each time step to achieve a constant pressure drop across the channel, motivated by the geological situation of a porous medium connecting two regions of  fixed hydraulic heads.

%to transition to the more complex flow intervening between the bodies.


%-----------------------------------------------------------------------------------------------%
\section{Boundary Integral Equation and Cauchy Integral Formulation}
\label{sec:DLP}
%-----------------------------------------------------------------------------------------------%

In order to accurately and efficiently compute the Stokes flow governed by \eqref{eqn:StokesEq}, we reformulate the system as a boundary integral equation (BIE). The boundary integral equations are first formulated in $\RR^2$ and then recast in terms of a Cauchy integral. The Cauchy integral representation enables a barycentric quadrature method to accurately compute the flow intervening between nearly contacting bodies, as is required by the dense configurations examined in this work.

\nick{Perhaps you can combine the real BIE and Cauchy representation into one section to condense it and to make the complex analysis take more center stage.}

% SNIPPET
%This has the advantage that only the one-dimensional boundary of the domain must be discretized, and, with appropriate quadrature formulas and fast summation methods, the result is a high-fidelity numerical simulation with near-optimal computational complexity.

%-----------------------------------------------------------------------------------------------%
\subsection{Boundary Integral Representation in $\RR^2$}
%-----------------------------------------------------------------------------------------------%

\vsp{2} \nick{Finished rewording up to here} \vsp{2}


Applying the same approach as our previous work~\citep{qua-moo2018}, we
start with the double-layer potential 
\begin{align}
  \DDD[\eeta](\xx) = \int_{\bd\Omega} D(\xx,\yy) \eeta(\yy)\, ds_\yy = 
  \frac{1}{\pi}\int_{\bd\Omega} 
    \frac{\rr \cdot \nn}{\rho^2} \frac{\rr \otimes \rr}{\rho^2}
    \eeta(\yy) \, ds_\yy, \quad \xx \in \Omega,
  \label{eqn:velocityDLP}
\end{align}
where $D$ is the kernel of the integral operator, $\rr = \xx - \yy$,
$\rho = \|\rr\|$, $\nn$ is the unit outward normal at $\yy$, and $\eeta$
is an unknown density function.  We complete the integral equation
formulation by adding the $M$ Stokeslets,
$S[\llambda_\ell,\cc_\ell](\xx)$, and $M$ rotlets,
$R[\xi_\ell,\cc_\ell](\xx)$, where $\cc_\ell$ is a point inside the
$\ell^{th}$ body~\citep{pow-mir1987}.  Here $\llambda_\ell$ and
$\xi_\ell$ are the Stokeslet and rotlet strengths, respectively,
corresponding to the $\ell^{th}$ body.  Then, for any sufficiently
smooth geometry $\Omega$, the solution of the incompressible Stokes
equation with a Dirichlet boundary condition $\ff$ is
\begin{align}
  \uu(\xx) = \DDD[\eeta](\xx) + 
    \sum_{\ell=1}^M S[\llambda_\ell,\cc_\ell](\xx) + 
    \sum_{\ell=1}^M R[\xi_\ell,\cc_\ell](\xx), \quad \xx \in \Omega,
\end{align}
where the density function, Stokeslets, and rotlets satisfy
\begin{subequations}
\label{eqn:BIE}
\begin{alignat}{3}
  \ff(\xx) &= -\frac{1}{2}\eeta(\xx) + \DDD[\eeta](\xx) + 
    \NN_0[\eeta](\xx) \nonumber \\
    &\quad + \sum_{\ell=1}^M S[\llambda_\ell,\cc_\ell](\xx) + 
    \sum_{\ell=1}^M R[\xi_\ell,\cc_\ell](\xx), 
    \quad &&\qquad\xx \in \bd\Omega, \\
  \llambda_\ell &= \frac{1}{2\pi} \int_{\gamma_\ell} 
    \eeta(\yy)\, ds_\yy, &&\qquad \ell = 1,\ldots,M, \\
  \xi_\ell &= \frac{1}{2\pi} \int_{\gamma_\ell}
    (\yy - \cc_\ell)^\perp \cdot \eeta(\yy)\, ds_\yy, 
    &&\qquad \ell = 1,\ldots,M.
\end{alignat}
\end{subequations}
Here, the null space associated with the flux-free condition of $\ff$ is
addressed with  $\NN_0$ which is the integral operator with kernel
$N_0(\xx,\yy) = \nn(\xx) \otimes \nn(\yy)$, $\xx,\yy \in \Gamma$.  In
this work, $\ff$ is the prescribed velocity, which is equal to $\UU$ on
the outer wall, $\Gamma$, and equal to zero on the eroding bodies,
$\gamma_\ell$, $\ell=1,\ldots,M$.

Once~\eqref{eqn:BIE} is solved for the density function $\eeta$, the
corresponding deformation tensor, pressure, and vorticity at $\xx \in
\Omega$ are written in terms of layer potentials~\citep{qua-moo2018}.
To compute the deformation tensor for $\xx \in \gamma$, we include the
jump term
\begin{align}
  \frac{1}{2} \left(\pderiv{\eeta}{\ss} \cdot \ss \right) \left[
    \begin{array}{cc}
      s_x^2 - s_y^2 & 2s_x s_y \\ 2s_x s_y & s_y^2 - s_x^2
    \end{array}
  \right].
  \label{eqn:deformationJump}
\end{align}
Finally, the deformation tensor, pressure, and vorticity due to the
Stokeslets and rotlets are readily available~\citep{poz1992}. Having
computed the deformation tensor on $\gamma$, the shear stress is
computed using equation~\eqref{eqn:shearStress}. 

%-----------------------------------------------------------------------------------------------%
\subsection{Cauchy Integral Formulation}
\label{sec:DLPcomplex}
%-----------------------------------------------------------------------------------------------%
The velocity double-layer potential~\eqref{eqn:velocityDLP}, and its
corresponding deformation tensor, pressure, and vorticity are all
written as layer potentials in $\RR^2$.  However, the quadrature method
we introduce in section~\ref{sec:method} requires complex-valued
representations. The first step to form a complex representation is to
write the Laplace double-layer potential as the complex integral
\begin{align}
  \DD[\eeta](\xx) = \frac{1}{2\pi} \int_{\bd\Omega} 
    \frac{\rr \cdot \nn}{\rho^2}\eeta(\yy)\, ds_\yy = \Real (v(x)),
\end{align}
where
\begin{align}
  v(x) = \frac{1}{2\pi i} \int_{\bd\Omega}
    \frac{\eta(y)}{x - y} \, dy, \quad x \in \Omega.
  \label{eqn:laplaceComplex}
\end{align}
Here $x = x_1 + i x_2,y = y_1 + i y_2 \in \CC$ are the complex
counterparts of $\xx = (x_1,x_2),\yy = (y_1,y_2) \in \RR^2$, and $\eta =
\eta_1 + i \eta_2$ is the complex counterpart of $\eeta =
(\eta_1,\eta_2)$. Therefore, depending on the formulation of the layer
potential, $\Omega$ is interpreted as a subset of $\RR^2$ or $\CC$.
Equation~\eqref{eqn:laplaceComplex} is converted to a Cauchy integral by
first finding the boundary data of $v$. If $\Omega$ is a
simply-connected interior domain, then the boundary data of $v$
satisfies the Sokhotski-Plemelj jump relation
\begin{align}
  \label{eqn:SPrelation}
  v(x) = - \frac{1}{2} \eta(x) + \frac{1}{2\pi i} \int_{\bd\Omega}
    \frac{\eta(y)}{x-y}\, dy, \quad x \in \bd\Omega.
\end{align}
For exterior domains, the jump term changes from $-1/2$ to $1/2$, and
for multiply-connected domains, such as a porous media, $\bd\Omega$ is
decomposed into its different connected components and the appropriate
jump relation is applied.  Having computed the boundary data of the
holomorphic function $v$, by the Cauchy integral theorem we have
\begin{subequations}
  \label{eqn:cauchy}
  \begin{alignat}{3}
  v(x) &= \frac{1}{2\pi i}\int_{\bd\Omega} 
    \frac{v(y)}{y-x} \,dy, \\
  v'(x) &= \frac{1}{2\pi i} \int_{\bd\Omega}
    \frac{v(y)}{(y-x)^2} \, dy, \\
  v''(x) &= \frac{1}{\pi i} \int_{\bd\Omega}
    \frac{v(y)}{(y-x)^3} \, dy,
  \end{alignat}
\end{subequations}
for $x \in \Omega$.  Since $v(x)$ depends on the complex-valued density
function $\eta$, we use the notation $v[\eta](x)$ for the holomorphic
function defined in equation~\eqref{eqn:laplaceComplex}, and its first
two derivatives are written as $v'[\eta](x)$ and $v''[\eta](x)$.  
  
Finally, the Stokes double-layer potential~\eqref{eqn:velocityDLP} can
be written using a Laplace double-layer
potential~\eqref{eqn:laplaceComplex} and its gradients 
\begin{equation}
  \label{eqn:Stokes2Laplace}
  \begin{aligned}
    \DDD[\eeta](\xx) &= 
      \frac{1}{2\pi} \int_{\bd\Omega} 
        \frac{\nn}{\rho^2} (\rr \cdot \eeta) \, ds_\yy + 
      \frac{1}{2\pi} \nabla \int_{\bd\Omega}
        \frac{\rr \cdot \nn}{\rho^2} (\yy \cdot \eeta) \, ds_\yy \\
      &- \frac{1}{2\pi} x_1 \nabla \int_{\bd\Omega}
        \frac{\rr \cdot \nn}{\rho^2}\eta_1(\yy) \, ds_\yy -
      \frac{1}{2\pi} x_2 \nabla \int_{\bd\Omega}
        \frac{\rr \cdot \nn}{\rho^2}\eta_2(\yy) \, ds_\yy.
  \end{aligned}
\end{equation}
Therefore, the Stokes double-layer potential can be written as a sum of
Cauchy integrals and its first derivative~\citep{bar-wu-vee2015}
\begin{equation}
  \begin{aligned}
    u_1(x) &= \Real (v[\psi_1](x)) + \Real (v'[y\cdot\eta](x)) 
             -x_1\Real (v'[\eta_1](x)) - x_2\Real (v'[\eta_2](x)), \\
    u_2(x) &= \Real (v[\psi_2](x)) - \Imag (v'[y\cdot\eta](x)) 
         +x_1\Imag (v'[\eta_1](x)) + x_2\Imag (v'[\eta_2](x)),
  \end{aligned}
  \label{eqn:cauchyVelocity}
\end{equation}
where $y \cdot \eta = y_1 \eta_1 + y_2 \eta_2$, 
\begin{align} 
  \psi_1=(\eta_1+i\eta_2)\frac{\Real(n)}{n}, \quad
  \psi_2=(\eta_1+i\eta_2)\frac{\Imag(n)}{n},
\end{align}
and $n \in \CC$ is the complex counterpart of the outward unit normal
$\nn \in \RR^2$.

%-----------------------------------------------------------------------------------------------%
\subsection{Cauchy Integral Representation for the Gradient of the
Double-Layer Potential}
\label{sec:gradDLPcomplex}
%-----------------------------------------------------------------------------------------------%
Computing the shear stress and vorticity requires a complex-valued layer
potential representation of the velocity gradient.  The deformation
tensor at $x \in \Omega$ is found by computing the derivatives of the
expressions for $u_1$ and $u_2$ in equation~\eqref{eqn:cauchyVelocity}  
\begin{equation}
\label{eqn:cauchyGradient}
  \begin{aligned}
    \pderiv{u_1}{x_1} &= +\Real (v'[\psi_1](x)) + 
    \Real (v''[y\cdot\eta](x)) - \Real (v'[\eta_1](x)) \\
    &- x_1\Real (v''[\eta_1](x)) - x_2\Real (v''[\eta_2](x)), \\
    \pderiv{u_1}{x_2} &= - \Imag (v'[\psi_1](x)) - 
    \Imag (v''[y\cdot\eta](x)) + x_1\Imag (v''[\eta_1](x)) \\
    &- \Real (v'[\eta_2](x)) + x_2\Imag (v''[\eta_2](x)), \\
    \pderiv{u_2}{x_1} &= +\Real (v'[\psi_2](x)) - 
    \Imag (v''[y\cdot\eta](x)) + \Imag (v'[\eta_1](x))  \\
    &+ x_1\Imag (v''[\eta_1](x)) + x_2\Imag (v''[\eta_2](x)), \\
    \pderiv{u_2}{x_2} &= -\Imag (v'[\psi_2](x)) - 
    \Real (v''[y\cdot\eta](x)) + x_1\Real (v''[\eta_1](x)) \\
    &+ \Imag (v'[\eta_2](x)) + x_2\Real (v''[\eta_2](x)).
  \end{aligned}
\end{equation}
The same expressions are used to compute the deformation tensor for $x
\in \bd\Omega$, except that the jump
condition~\eqref{eqn:deformationJump} is included.  Finally, to compute
the shear stress, the deformation tensor on $\bd\Omega$ is applied to
the normal and tangent vectors as in equation~\eqref{eqn:shearStress}.
The velocity gradient is also used to compute the vorticity in the fluid
bulk.  For $x \in \Omega$, the Cauchy integral representation of the
vorticity at $x \in \Omega$ is
\begin{align}
  \omega(x) = 
    \Real (v'[\psi_2](x)) + \Imag (v'[\psi_1](x))+ 
    \Real (v'[\eta_2](x))+ \Imag (v'[\eta_1](x)).
\end{align}
%\end{comment}

%-----------------------------------------------------------------------------------------------%
\section{Measuring permeability and other porous-media properties}
\label{sec:}
%-----------------------------------------------------------------------------------------------%

%-----------------------------------------------------------------------------------------------%
\subsection{Resistivity and permeability}
Darcy's law is given by
\begin{equation}
\bq = - \bvec{K} \grad \phi
\end{equation}
where $\bq$ is the specific discharge, $\phi = z + p/\gamma$ is the hydraulic head, and $\gamma = \rho g$. The rank-2 tensor $\bvec{K}$ is the {\em conductivity}, which is related to the {\em permeability} $\bvec{k}$ via $\bvec{K} = \bvec{k} \gamma/\mu$, where $\mu$ is the viscosity of the fluid. The units of conductivity are length-per-time (i.e.~velocity) and the units of permeability are length squared.
Perhaps the most standard parameter to characterize the porous medium is the permeability. We eliminate gravity from consideration to get
\begin{equation}
\bq = - \bvec{k} \grad p / \mu
\end{equation}
Henceforth, we set $\mu = 1$ by the non-dimensionalization.

Consider first an isotropic porous medium with {\em scalar} permeability $k$. Furthermore, we seek to characterize the medium by a single, bulk value permeability, so we assume that $k$ is spatially constant.
Equivalently, we can introduce a {\em resistivity} simply defined as the reciprocal of permeability, $r = 1/k$. Then taking the horizontal component gives
\begin{equation}
\label{q1}
q_1(x,y) = -\frac{1}{r} \pderiv{p}{x}(x,y)
\end{equation}
Consider the horizontal discharge, $q_1(x,y)$, averaged over a vertical slice
\begin{equation}
\qavg := \frac{1}{2} \int_{-1}^{1} q_1(x_0, y) dy = U
\end{equation}
By conservation of mass, $\qavg$ is independent of the horizontal position $x_0$ and is equal to the imposed value $U$. Since $\bq$ is a volume-average of the tracer velocity, $\qavg$ can be equivalently obtained by averaging the tracer velocity, which is how we will measure this quantity in the simulations.

Next, consider the pressure averaged over a vertical slice
\begin{equation}
\pavg(x_0) := \frac{1}{2} \int_{-1}^{1} p(x_0, y) dy
\end{equation}
In particular, the upstream and downstream pressures are respectively
\begin{equation}
\pup = \pavg(-1) \, , \quad \pdn = \pavg(1)
\end{equation}

Integrating \eqref{q1} over the porous-medium domain $[-1, 1] \times [-1,1]$, applying the fundamental theorem of calculus, and rearranging gives
\begin{equation}
r = \frac{\pup - \pdn}{2 U}
\end{equation}
This exact formula gives the resistivity based on measurements of the upstream/downstream pressure and the discharge. 

Probably should just use
\begin{equation}
k = \frac{2U}{\pup - \pdn}
\end{equation}
%Note: If a larger value of $x_0$ is chosen, then one would be in effect averaging the porous-medium resistivity with the vanishing resistivity of the empty buffer region.

%-----------------------------------------------------------------------------------------------%
\subsection{Relationship to total drag}

The resistivity is directly related to the drag exerted by the bodies. Indeed, another way to express the Stokes equations is $\grad \cdot \stress = 0$ where $\stress = -p \bvec{I} + \mu \left( \grad \uu + \grad \uu^T \right)$ is the stress tensor. The divergence theorem gives
\begin{equation}
\label{divthm}
0 = \int_{\Omega} \grad \cdot \stress \, dV 
= \int_{\partial \Omega} \stress \cdot \nn \, ds
\end{equation}
for any domain $\Omega$.  Consider $\Omega$ to be inside $[-x_0, x_0] \times [-1, 1]$, where we will choose $x_0 > 1$ so that $\Omega$ includes the entire porous region.
The boundary integral on the right-hand-side of \eqref{divthm} has contributions from the boundaries of the erodable bodies $\gamma$ and from the outer boundary $\Gamma$, since $\partial \Omega = \gamma \cup \Gamma$. In particular, the total drag exerted by the erodable bodies is given by
\begin{equation}
\FD = -\int_{\gamma} \stress \cdot \nn \, ds
\end{equation}
which includes both pressure and viscous drag. The minus sign is due to the orientation of the normal vector. We will mostly be interested in the horizontal drag $\FD \cdot \ex$. \\

\nick{Need to check sign on drag and check whether I want the drag {\em on} the bodies or exerted {\em by} the bodies.} \\

Consider now the contribution to \eqref{divthm} from the outer boundary $\Gamma$. The horizontal component of this term is
\begin{align}
\ex \cdot \int_{\Gamma} \stress \cdot \nn \, ds = 
& \int_{-1}^{1} p(-x_0, y) - p(x_0, y) - 2\mu (u_x(-x_0,y) - u_x(x_0,y)) \, dy \, + \\
& \int_{-x_0}^{x_0} \mu (u_y(x,1) - u_y(x,-1)) dx
\end{align}
If we take $x_0$ sufficiently far away, then the viscous terms involving $u_x$ will become small as the velocity field tends to a uniform flow. Meanwhile, the contribution from the pressure is well approximated by the upstream and downstream average pressures $\pup$ and $\pdn$ as introduced in the previous section. This gives 
\begin{equation}
\FD \cdot \ex \approx 2 (\pup - \pdn)
\end{equation}

Therefore we have the relationship
\begin{equation}
\label{ResistDrag}
1/k \approx \frac{1}{4 U}  \, \FD \cdot \ex
\end{equation}

%---------------------------------------------------------%
\begin{figure*}%[htbp]
\centering \label{fig2}
\includegraphics[width = 0.99 \textwidth]{./figs/fig2.pdf}
\caption{A single simulation of porous-medium erosion. The initial configuration consists of 100 circular bodies of random size and position, as seen in Fig.~1? a) As the medium erodes, the fraction of solid-body area decreases, or, equivalently, the porosity increases. b) Resistance to flow can be characterized by either the inverse permeability, $1/k$, or the cumulative drag, the two of which are related through \eqref{ResistDrag}. Both are seen to decrease as the medium erodes, and relationship \eqref{ResistDrag} is confirmed by the simulation data.
c) The relationship between permeability $k$ and porosity $\phi$ for the eroding medium. Both quantities increase with time as the medium erodes, but at different rates due to changes in the microstructure and channelization of the medium.
}
\end{figure*}
%---------------------------------------------------------%
% Data from 100-3

The anisotropy of permeability is
\begin{equation}
\anis = k_1 / k_2
\end{equation}

There are two possible explanations for how erosion creates anisotropy in the microstructure. First, the shear stresses could carve each individual body into a more slender form, thus creating anisotropy at the level of individual grains. Second, by removing material in a non-uniform way, could preferentially disintegrate certain bodies before others, leaving an overall configuration that exhibits anisotropy on the large scale. We will refer to these two possible mechanisms as {\em shape} anisotropy and {\em configurational} anisotropy respectively. As an extreme example, a tightly spaced row of circular bodies would exhibit configurational anisotropy but no shape anisotropy. Meanwhile, an array of highly eccentric ellipses, all oriented horizontally but positioned randomly, would exhibit high shape anisotropy and low configurational anisotropy.

We note that channelization is related to both shape and configurational anisotropy....

It is possible to device a test to isolate these two types of anisotropy.
Introduce circular bodies with same centers of mass...
Thus, the configurational anisotropy is defined as
\begin{equation}
\anis_C = k_1^{(c)} / k_2^{(c)}
\end{equation}

The total anisotropy is the produce of the shape and configurational
\begin{equation}
\anis = \anis_S \anis_C
\end{equation}


%---------------------------------------------------------%
\begin{figure*}%[htbp]
\centering \label{fig3}
\includegraphics[width = 0.99 \textwidth]{./figs/fig3.pdf}
\caption{
a)
}
\end{figure*}
%---------------------------------------------------------%
% Data from 100-3


Text for Fig 3: 
The anisotropy of permeability grows large, up to a value of 4 during the middle portion of the simulation. The figure also shows the anisotropy of the circular grid.

%---------------------------------------------------------%
\begin{figure*}%[htbp]
\centering \label{fig4}
\includegraphics[width = 0.99 \textwidth]{./figs/fig4.pdf}
\caption{
a) Statistics from an ensemble of runs.
}
\end{figure*}
%---------------------------------------------------------%
% Data from 100-3

\bibliographystyle{plain}
\bibliography{refs}
\end{document}
