\documentclass[11pt]{article}


\usepackage{fullpage}
\usepackage{graphicx,amsmath,amsfonts,amssymb,stmaryrd}
\usepackage{color}
\newcommand{\comment}[1]{{\color{blue} #1}}

%----------------------------- CUSTOM COMMANDS -----------------------------%

%----------------------------------------------------------------------------%
%% LATEX DEFINITIONS
%----------------------------------------------------------------------------%
% Basic editing
\newcommand{\tocite}{{\color{blue}(to cite)}}
\newcommand{\vsp}[1]{\vspace{#1 pc} \noindent}
\newcommand{\np}{\newpage \noindent}

% Derivatives
\newcommand{\pd}[2]    { \frac{\partial #1} {\partial #2} }
\newcommand{\ppd}[2]  { \frac{\partial^2 #1}{{\partial #2}^2} }
\newcommand{\pdi}[2] { {\partial_#2} #1 }
\newcommand{\td}[2] { \frac{d #1} { d #2 } }
\newcommand{\grad}{\nabla}
\newcommand {\Lap} {\grad^2}
% Vectors and operators
\newcommand{\bvec}[1]{\ensuremath{\boldsymbol{#1}}}
\newcommand{\abs}[1]{\left| #1 \right|}
\newcommand{\CC}{{\mathbb{C}}}
\newcommand{\RR}{{\mathbb{R}}}
\newcommand{\Real}{\text{Re}}
\newcommand{\Imag}{\text{Im}}
\newcommand{\dotp}{\boldsymbol{\cdot}}

% Edit
\newcommand{\edit}[1]{{\color{red} #1}}


% Math
\newcommand{\anis}{\mathcal{A}}



%----------------------------- DOCUMENT -----------------------------%
\begin{document}

\section*{Responses to Referee 2}

We would like to thank the referee for the careful and insightful review of our work. The referee's comments have gone a long way in helping us to improve the clarity and quality of the manuscript. Below we respond to each of the referee's comments in sequence. For your convenience, we are submitting a version of the revised manuscript with the changes highlighted. \\ \\


\noindent \comment{Reviewer \#1: 
The paper studies erosion within a porous medium comprising an array of (initially) close-packed circular cylinders, across which a Stokesian fluid flows. Erosion of the cylinders occurs, at a rate proportional to local shear stress at cylinder boundaries. The authors study how the erosion leads to the initially isotropic medium developing anisotropic features over time, and they present various measures to quantify the extent and nature of the anisotropy. \\

I found the paper well written and easy to read. The work is certainly worthy of publication; however several questions and comments came to mind as I read it, and I'd like the authors to consider these in a revised version.
\\ }

\noindent
We thank the referee for this positive evaluation of our work. \\ \\

\noindent \comment{Reviewer \#2: 
Stylistic comments: \\
- Section 3 (mainly section 3.1 and what comes before it) seems much longer than necessary and contains a lot of technical (sometimes trivial) detail. For example, do we really need to see the components of the deformation tensor (12a-d) in the main text? Do we really need to see the same function differentiated twice (16a-c)? Then a straightforward trapezoidal approximation (17)? Why not omit (16b,c) and combine (16a) and (17) in a single line? Then more approximations in (21-22). To my mind, much of the detail in this section belongs in an appendix. Or SI.
\\ }

\noindent
Thank you for bringing these stylistic points to our attention. We have taken the referee's advice and moved many of the more detailed equations into an appendix. 
\\ \\

\noindent \comment{Reviewer \#2: 
- Equation (2) for the erosion - but this is not in fact the law that is implemented in the paper. So it seems a bit misleading to use it in the model formulation. I can see the authors might prefer a cleaner model presentation, but I think at least a comment saying that in the implementation the law is modified would be appropriate.
\\ }

\noindent
This is a good point. In the updated manuscript, we have included a comment after Eqs.~(2)--(3) that the erosion law will be slightly modified in Section 3.2 for the sake of numerical stability.
\\ \\

\noindent \comment{Reviewer \#2: 
- Bottom of p11: ``statistical analysis of many simulations will show" - this sounds speculative.  I interpreted this on first reading to be the authors' conjecture, rather than something they actually do in the present paper. Why not rephrase to indicate where this is shown (Sec. 5.2)?
\\ }

\noindent
Thank you for bringing this ambiguity to our attention. We have rephrased this passage to make it clear that our statistical analysis in Section 5.2 demonstrates the point in question.
\\ \\

\noindent \comment{Reviewer \#2: 
Scientific comments \\ \\
- I didn't see any mention in the paper that the study is restricted to 2D flows. Of course, it's obvious from the presented study and the methods - but I think it's worth pointing this out explicitly, and it's also worth mentioning specifically in the paper where differences in the modeling and results would arise in a more general 3D study. (For instance, in the modeling, more care would be needed with the shear stress definition in (3) and the associated erosion law.) If the authors are considering real-world situations where the geometry is (quasi)-2D then this is worth mentioning also. But in 3D I would expect the anisotropy progression to be weaker than in the 2D case, due to the flow having an extra dimension to flow into. So I'd like to see some discussion of this point, if only at a speculative level.
\\ }

\noindent
Thank you for bringing up this point. We have now explicitly stated that the computations are two-dimensional in the abstract, the introduction, and in a few other places in the paper. We also included a brief discussion at the beginning of Section 3 on modifications that would be necessary for 3D. In short, many aspects of the numerical methods (not only the quadrature method) would need to change substantially.
\\ \\

\noindent \comment{Reviewer \#2: 
- In the introduction, first paragraph, it is stated: ``Typical materials are more permeable to flow in the longitudinal (or horizontal) direction than transversely…" - what does this even mean? What determines the longitudinal direction in such a typical material? The statement (and the short discussion that follows) seems almost circular to me - the material is more permeable in the direction in which it admits flow? That's almost the definition of permeability.
\\ }

\noindent
It is the direction of the pressure gradient that determines the direction of flow. The direction of the pressure gradient could change with time, and so, in principle, a porous material with fixed microstructure (and hence fixed permeability tensor) could admit flow in any direction. For example, a porous material might be isotropic, meaning that it admits flow in the vertical direction just as easily as it does in the horizontal direction, or any direction for that matter. The actual flow direction at any given time would thus be determined entirely by the pressure gradient that is imposed. As a second example, a porous medium might be composed of slender and primarily vertically-aligned particles. This material would preferably admit flow in the vertical direction over the horizontal direction, but if the pressure gradient is horizontal, the flow will be horizontal. Hence, the direction of the anisotropy in permeability does not determine the flow direction, although it is an influencing factor.

We have slightly rephrased this statement, removing the terms longitudinal and transversely and replacing them with horizontal and vertical, to prevent confusion.
\\ \\

\noindent \comment{Reviewer \#2: 
- Section 2, page 2, below equations (1): ``To nondimensionalize, we set the fluid viscosity to unity, $\mu=1$." This is a lazy way to nondimensionalize. The authors have already gone half-way to nondimensionalizing by this stage, without saying so, by choice of the computational domain. Why not just do it properly and say what the appropriate scalings for all physical quantities are? Then $\mu$ won't appear in any of the equations (which currently have an uncomfortable hybrid feel - are they meant to be dimensionless, or not?).
\\ }

\noindent
We have taken the referee's advice and removed $\mu$ from the governing equations and also stated the scales used in the non-dimensionalization. Please see the revised Section 2. We note that we have kept $\mu$ in Eq.~(9), and shortly thereafter set it to unity, because it may be useful for future work to see where the viscosity enters Darcy's law.
\\ \\


\noindent \comment{Reviewer \#2: 
- Section 3.2, first line: ``…vorticity, which reduces to shear stress $\tau$ on solid boundaries" - is that right?
\\ }

\noindent
Yes, the vorticity reduces to the shear stress, up to a sign, on a no-slip boundary. We note that the no-slip part is essential for this statement to be true. It can be seen in 2D by, without loss of generality, taking a horizontal surface and evaluating the vorticity $\omega =  v_x - u_y$ on the no-slip surface. By the no-slip assumption, $v_x = 0$, and so $\omega = -u_y$ which is the same, up to a sign, as the shear stress $\tau$. For any other alignment of the surface, the coordinate system can simply be rotated. Since we use $\abs{\tau}$ in the erosion law, the sign makes no difference.
\\ \\

\noindent \comment{Reviewer \#2: 
- Section 4.1, diagonal permeability tensor. I understand why the authors take the approach that they do, and most likely it's justified. However, the authors don't ever check the consistency of the diagonal assumption. How would they go about checking this? Could this be commented on?
\\ }

\noindent
This is a great question and one that we have given a fair amount of thought to. Unfortunately, it would be difficult to check the diagonal assumption in the current computational setup since rotating the configuration of bodies by any angle that is not a multiple of $90^\circ$ does not guarantee the bodies would remain in the square domain. Imagining a nice computational domain that is not subject to this limitation is tricky. In principle, we could confine the bodies to a circular sub-domain (as opposed to a square sub-domain) with the whole domain, including the buffer region, remaining a rectangle. This setup has its own pitfalls though. For one, the flow would impinge on the bodies in a non-uniform way. For example, the bodies near the left-most and right-most extreme of the circular subdomain would experience the `leading front' of the flow profile. In our mind, this would be worse than the left-most (or right-most) column of bodies experiencing the leading front, since in that situation, the effect is spread out over more bodies in a uniform way.
\\ \\

\noindent \comment{Reviewer \#2: 
- Paragraph below eq.(36) - ``keep the outer geometry fixed and rotate the inner configuration of bodies" - in the model introduction, the authors described the ``buffer zones" [-3,-1], [1,3] in the x-direction, used to allow the flow to adjust to the presence of the bodies (and presumably give more reliable indicators for the anisotropy measures calculated). But the computational domain in the y-direction is only given as [-1,1]. Do the authors extend it for the y-permeability calculation, to get similarly reliable estimates? Or maybe I'm misunderstanding the procedure used.
\\ }

\noindent
Please read on to the next sentence to answer your question: ``In practice, it is simpler to keep the outer geometry fixed and rotate the inner configuration of bodies, then simply apply the method from Section 4.1 to measure $k_{22}$.''

You are correct, if we were to keep the outer geometry fixed and rotate the inner configuration of bodies, then we would need to adjust the buffer region accordingly. However, this harry process can be circumvented by instead simply rotating the inner configuration of bodies and keeping the outer geometry, buffer region included, as is. Then imposing a horizontal flow in this altered configuration is equivalent to imposing a vertical flow in the original configuration.
\\ \\

\noindent \comment{Reviewer \#2: 
- Page 9, last paragraph of section 4.3: ``… for a fixed configuration of partially-eroded bodies, we replace each body with a circle having the same area and center of mass. Since the configurations resulted from the erosion of initially circular bodies, such a replacement does not lead to any overlap between bodies" - this certainly sounds reasonable, but is it obvious? I assume that, depending on the flow, the COM can change during erosion. Might there be some non-standard examples where the statement doesn't hold? [This is mainly curiosity on my side; for the purposes of publication, if the authors say that overlap never does occur in any simulations then that's fine!]
\\ }

\noindent
You are correct that this is not a theorem that we can prove since the COM could in principle move somewhat. But it is fairly intuitive that overlaps should not result from this process, since the amount the boundary moves inward would likely dominate any motion of the COM. In practice, we observe that overlaps do not result from this process.
\\ \\

\noindent \comment{Reviewer \#2: 
- Results, Section 5.1, Figure 2: Here the authors compare 1/k11 (from the Darcy law) with F1/(4U) (from the Stokes calculations). Why don't they make a similar comparison for k22, since they also compute the flow in the transverse configuration? 
\\ }

\noindent
The purpose of the drag-permeability comparison is to test the connection between the microscopic Stokes description and the coarse-grained Darcy description. We believe that purpose is adequately achieved with the $F_1$ data shown in the Figure. Computing $F_2$ would be redundant and possibly confusing to some readers for the following reason. Even in the original horizontal flow, there is typically a non-zero component of vertical drag, due to random up-down asymmetries of the bodies. Hence it might be unclear if $F_2$ refers to the vertical component of force in the original flow configuration or the component measured with the rotated flow. Of course, this could all be explained but it would add a significant amount of text, and for what purpose? We have already established the micro-macro connection with $F_1$.
\\ \\

\noindent \comment{Reviewer \#2: 
- Results, Section 5.2: What's the rationale for considering trends as the number of bodies M is increased? I'd have preferred to see more simulations (= better statistics) for large values of M, where the Darcy model is more applicable. Is there a specific reason to look at small M-values? Variations with M seems an odd choice to study.
\\ }

\noindent
It does not seem odd to us. Varying the number of bodies, $M$, is one way (perhaps the best way) to alter the permeability of a medium while keeping it's porosity fixed. In the study of porous media, considerable attention is given to characterizing both a medium's porosity and it's permeability, as these two properties can be altered independently by changing the microstructure. Studying fewer instances of small $M$ would not permit us to study more instances of large $M$ since the computational time required for the large-$M$ simulations is so much larger.

We remark that one key discovery we made was the late surge in configurational anisotropy for simulations featuring a large number of bodies, $M \ge 60$, and how this configurational anisotropy connects to channelization. Had we not varied $M$, we could not have discovered the threshold of 60 required for this phenomenon.
\\ \\

\noindent \comment{Reviewer \#2: 
In addition, I found a few typos as I read:
\\ }

\noindent
We have corrected all of these typos. Many thanks for spotting them!

\end{document}
