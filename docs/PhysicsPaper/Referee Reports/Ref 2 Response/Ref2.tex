\documentclass[11pt]{article}


\usepackage{fullpage}
\usepackage{graphicx,amsmath,amsfonts,amssymb,stmaryrd}
\usepackage{color}
\newcommand{\comment}[1]{{\color{blue} #1}}

%----------------------------- CUSTOM COMMANDS -----------------------------%

%----------------------------------------------------------------------------%
%% LATEX DEFINITIONS
%----------------------------------------------------------------------------%
% Basic editing
\newcommand{\tocite}{{\color{blue}(to cite)}}
\newcommand{\vsp}[1]{\vspace{#1 pc} \noindent}
\newcommand{\np}{\newpage \noindent}

% Derivatives
\newcommand{\pd}[2]    { \frac{\partial #1} {\partial #2} }
\newcommand{\ppd}[2]  { \frac{\partial^2 #1}{{\partial #2}^2} }
\newcommand{\pdi}[2] { {\partial_#2} #1 }
\newcommand{\td}[2] { \frac{d #1} { d #2 } }
\newcommand{\grad}{\nabla}
\newcommand {\Lap} {\grad^2}
% Vectors and operators
\newcommand{\bvec}[1]{\ensuremath{\boldsymbol{#1}}}
\newcommand{\abs}[1]{\left| #1 \right|}
\newcommand{\CC}{{\mathbb{C}}}
\newcommand{\RR}{{\mathbb{R}}}
\newcommand{\Real}{\text{Re}}
\newcommand{\Imag}{\text{Im}}
\newcommand{\dotp}{\boldsymbol{\cdot}}

% Edit
\newcommand{\edit}[1]{{\color{red} #1}}


% Math
\newcommand{\anis}{\mathcal{A}}



%----------------------------- DOCUMENT -----------------------------%
\begin{document}

\section*{Responses to Referee 2}

We would like to thank the referee for reviewing our work ....

Below we respond to each of the referee's comments in sequence. For your convenience, we are submitting a version of the revised manuscript with the changes highlighted. \\ \\ \\


\noindent \comment{Reviewer \#1: 
The paper studies erosion within a porous medium comprising an array of (initially) close-packed circular cylinders, across which a Stokesian fluid flows. Erosion of the cylinders occurs, at a rate proportional to local shear stress at cylinder boundaries. The authors study how the erosion leads to the initially isotropic medium developing anisotropic features over time, and they present various measures to quantify the extent and nature of the anisotropy. \\

I found the paper well written and easy to read. The work is certainly worthy of publication; however several questions and comments came to mind as I read it, and I'd like the authors to consider these in a revised version.
\\ }

\noindent
We thank the referee for this positive evaluation of our work. \\ \\

\noindent \comment{Reviewer \#2: 
Stylistic comments: \\
- Section 3 (mainly section 3.1 and what comes before it) seems much longer than necessary and contains a lot of technical (sometimes trivial) detail. For example, do we really need to see the components of the deformation tensor (12a-d) in the main text? Do we really need to see the same function differentiated twice (16a-c)? Then a straightforward trapezoidal approximation (17)? Why not omit (16b,c) and combine (16a) and (17) in a single line? Then more approximations in (21-22). To my mind, much of the detail in this section belongs in an appendix. Or SI.
\\ }

\noindent
Thank you for bringing this stylistic point to our attention. We have taken the referee's advice and moved many of the more detailed equations into an appendix. 
\\ \\

\noindent \comment{Reviewer \#2: 
- Equation (2) for the erosion - but this is not in fact the law that is implemented in the paper. So it seems a bit misleading to use it in the model formulation. I can see the authors might prefer a cleaner model presentation, but I think at least a comment saying that in the implementation the law is modified would be appropriate.
\\ }

\noindent
This is a good point. In the updated manuscript, we have included a comment after Eqs.~(2)--(3) that this law will be slightly modified in Section 3.2 for the sake of numerical stability.
\\ \\

\noindent \comment{Reviewer \#2: 
- Bottom of p11: "statistical analysis of many simulations will show" - this sounds speculative.  I interpreted this on first reading to be the authors' conjecture, rather than something they actually do in the present paper. Why not rephrase to indicate where this is shown (Sec. 5.2)?
\\ }

\noindent
Thank you for bringing this ambiguity to our attention. We have rephrased this passage to make it clear that our statistical analysis in Section 5.2 demonstrates the point in question.
\\ \\

\noindent \comment{Reviewer \#2: 
Scientific comments \\ \\
- I didn't see any mention in the paper that the study is restricted to 2D flows. Of course, it's obvious from the presented study and the methods - but I think it's worth pointing this out explicitly, and it's also worth mentioning specifically in the paper where differences in the modeling and results would arise in a more general 3D study. (For instance, in the modeling, more care would be needed with the shear stress definition in (3) and the associated erosion law.) If the authors are considering real-world situations where the geometry is (quasi)-2D then this is worth mentioning also. But in 3D I would expect the anisotropy progression to be weaker than in the 2D case, due to the flow having an extra dimension to flow into. So I'd like to see some discussion of this point, if only at a speculative level.
\\ }

\noindent
Thank you for bringing up this point. We have no explicitly stated that the computations are two-dimensional in the abstract, the introduction, and in a few other places in the paper. We also included a brief discussion at the beginning of Section 3 on modifications that would be necessary for 3D. In short, many aspects of the numerical methods (not only the quadrature method) would need to change substantially.

\end{document}
