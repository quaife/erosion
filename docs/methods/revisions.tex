\documentclass[11pt]{article}


\usepackage{fullpage}
\usepackage{amsmath,amsfonts,amssymb,stmaryrd}
\usepackage{color}
\newcommand{\comment}[1]{{\color{blue} #1}}

\begin{document}

\noindent
We thank the reviewers for their reviews.  Below is a summary of the
changes.  The revised manuscript also addresses a typo that was in
equation (10).

\section*{Reviewer 1}

\noindent
\comment{In section 3.1, formula 29 is missing a $\Delta s_{j}$}
\begin{itemize}
  \item We thank the reviewer for noticing this typo.  The arclength
    term is now included in equation (29).
\end{itemize}

\noindent
\comment{In section 4.1, the symbols $v$ and $u$ haven't been defined
previously}
\begin{itemize}
  \item $\mathbf{u} = (u,v)$ is now defined leading up to Section 4.1.
\end{itemize}



\section*{Reviewer 2}
\noindent
\comment{The time scale of erosion totally disappears from the analysis.
How would this time scale be set in practise?}
\begin{itemize}
\item This is a great question and an issue that admittedly was skirted in the original manuscript. In the revised manuscript, we have more carefully explained the non-dimensionalization. See the text immediately below Eq.~(1). In particular, the erosion timescale is given by $T^* = {L^*}^2/(C_E \mu U^*)$, where $C_E$ is the erosion constant (a material property), $L^*$ is a characteristic lengthscale and $U^*$ a characteristic flow rate.
\end{itemize}

\noindent
\comment{On page 23: the scaling of Fd with 4*pi*U is not dimensionless.
It seems to me like the drag coefficient might be a better quantity to
base the discussion on.}
\begin{itemize}
\item Yes, we agree that it makes more sense to have this normalization be dimensionless.  Accordingly, we have now included the viscosity in the denominator, so that we are dealing with the quantity $F_D /(4 \pi \mu U)$. This modification does not actually change the values shown in the plot because we have already set $\mu =1$ in order to non-dimensionalize. However, it would make it easier to apply our results to laboratory or field studies, which is a driving interest of ours.

As a side-note, in case you are concerned that there should be a lengthscale included in the denominator, remember that, since these are 2D simulations, $F_D$ is actually the drag-per-unit span (as discussed in Section 5.2).
\end{itemize}

\noindent
\comment{The aft and rear stagnation points both recede even though they
are subject to zero shear. Is it just a consequence of the
regularization?}
\begin{itemize}
\item
This is a good question. Indeed, the recession of the stagnation points is a result of the numerical smoothing (both the curvature penalization and the smoothing of the stress distribution) in our simulations. Stagnation points, however, do actually recede in laboratory experiments on erosion, leading us to believe that the numerical smoothing represents a real mechanism. For example, the smoothing could plausibly represent small-scale drifting of the stagnation point due to noise, or perhaps a second-order pressure dependence in the erosion rate. If you are interested, various possibilities are discussed in Moore et al.~{\it POF} 2013.

We have added the following text at the end of section 5.1:
\begin{quotation}
We note that the recession of the stagnation points, since the stress
vanishes there, is due entirely to the numerical smoothing (both the
curvature penalization and the stress-distribution smoothing).
Interestingly, stagnation points do indeed recede in laboratory
experiments on erosion [46, 64], leading us to believe that the
numerical smoothing represents a real physical mechanism (see [46] for
discussion).
%
%We note that the recession of the stagnation points, since the stress vanishes there, is due entirely to the numerical smoothing (both the curvature penalization and the stress-distribution smoothing). Interestingly, stagnation points do indeed recede in laboratory experiments on erosion, leading us to believe that the numerical smoothing represents a real physical mechanism (see [46] for discussion).
\end{quotation}

\end{itemize}


\end{document}
