\documentclass[11pt]{article}


\usepackage{fullpage}
\usepackage{amsmath,amsfonts,amssymb,stmaryrd}
\usepackage{color}
\newcommand{\comment}[1]{{\color{blue} #1}}

\begin{document}

\noindent
We thank the reviewers for their reviews.  Below is a summary of the
changes.  The revised manuscript also addresses a typo that was in
equation (10).

\section*{Reviewer 1}

\noindent
\comment{In section 3.1, formula 29 is missing a $\Delta s_{j}$}
\begin{itemize}
  \item Fixed
\end{itemize}

\noindent
\comment{In section 4.1, the symbols $v$ and $u$ haven't been defined
previously}
\begin{itemize}
  \item $\mathbf{u} = (u,v)$ is now defined leading up to Section 4.1.
\end{itemize}



\section*{Reviewer 2}
\noindent
\comment{The time scale of erosion totally disappears from the analysis.
How would this time scale be set in practise?}
\begin{itemize}
  \item After equation (1), the erosion timescale is described.
\end{itemize}

\noindent
\comment{On page 23: the scaling of Fd with 4*pi*U is not dimensionless.
It seems to me like the drag coefficient might be a better quantity to
base the discussion on.}
\begin{itemize}
  \item 
\end{itemize}

\noindent
\comment{The aft and rear stagnation points both recede even though they
are subject to zero shear. Is it just a consequence of the
regularization?}
\begin{itemize}
\item
This is a good question. Indeed, the recession of the stagnation points is a result of the numerical smoothing (both the curvature penalization and the smoothing of the stress distribution). However, in laboratory experiments, stagnation points do indeed recede, which leads us to believe that the numerical smoothing represents a real mechanism. For example, the smoothing could plausibly represent small-scale drifting of the stagnation point resulting from noise, or perhaps a second-order pressure dependence in the erosion rate. If you are interested, there is a discussion of these possibilities in Moore et al.~{\it POF} 2013.

We have added the following text at the end of section 5.1:
\begin{quotation}
We note that the recession of the stagnation points, since the stress vanishes there, is due entirely to the numerical smoothing (both the curvature penalization and the stress-distribution smoothing). Interestingly, stagnation points do indeed recede in laboratory experiments on erosion, leading us to believe that the numerical smoothing represents a real physical mechanism (see [46] for discussion).
\end{quotation}

\end{itemize}


\end{document}
