\documentclass[preprint, 10pt]{elsarticle}

\newcommand{\mcaption}[2]{\caption{\small \em #1}\label{#2}}
\newcommand{\secref}[1]{\ref{#1}}

%\usepackage{algorithmic}
%\usepackage{algorithm}
\usepackage{amsfonts}
\usepackage[fleqn,reqno]{amsmath}
\usepackage{amssymb}
%\usepackage{amsthm}
\usepackage[titletoc]{appendix}
\usepackage{array}
%\usepackage{bm}
%\usepackage{caption}
%\usepackage[usenames]{color}
\usepackage{enumitem}
%\usepackage{epsfig}
%\usepackage{fancybox}
\usepackage{filecontents}
\usepackage[top=1.2in,bottom=1.2in,left=1in, right=1in]{geometry}
\usepackage{graphics}
%%\usepackage{ifthen}
\usepackage{lineno}
%\usepackage{mathrsfs}
%\usepackage{mdframed}
%\usepackage{multirow}
%\usepackage{palatino}
%\usepackage{showkeys} %To see the labels for now.  Will remove later
%\usepackage{stmaryrd}
%\usepackage{subfigure}
%\usepackage{paralist}
\usepackage{pgfplots}
%\usepackage{tabularx}
\usepackage{tikz}
\usepackage{todonotes}
\usetikzlibrary{arrows}
\usepackage{comment}

%%%%%%  pdftex  %%%%%%%%%%%%%%%%%%%%%%%%%%%%%%%%%%%%%%%%%%%%%%%%%%%%%%
\usepackage[pagebackref=false,bookmarks=false]{hyperref} 

\hypersetup{
  bookmarksnumbered=true,
  bookmarksopen=false,
  hypertexnames=false,      
  breaklinks=true,          
  unicode=false,
  pdffitwindow=true,        
  pdfnewwindow=true,        
  colorlinks=true,         
  linkcolor=dblue,
  anchorcolor=red,
  citecolor=dorange,
  filecolor=magenta,
  urlcolor=dblue,
  pdfstartview = FitH,
  pdfkeywords = {},
  pdfcreator = {LaTeX with hyperref package}
}



\newcommand{\bd}{{\partial}}
\newcommand{\bigO}{{\mathcal{O}}}
\newcommand{\cc}{{\mathbf{c}}}
\newcommand{\DD}{{\mathcal{D}}}
\newcommand{\eeta}{{\boldsymbol\eta}}
\newcommand{\ff}{{\mathbf{f}}}
\newcommand{\grad}{{\nabla}}
\newcommand{\II}{{\mathbf{I}}}
\newcommand{\iin}{\mathrm{in}}
\newcommand{\llambda}{{\boldsymbol\lambda}}
\newcommand{\nn}{{\mathbf{n}}}
\newcommand{\NN}{{\mathcal{N}}}
\newcommand{\out}{\mathrm{out}}
\newcommand{\rr}{{\mathbf{r}}}
\newcommand{\RR}{{\mathbb{R}}}
\renewcommand{\ss}{{\mathbf{s}}}
\newcommand{\ssigma}{{\boldsymbol\sigma}}
\newcommand{\tar}{\mathrm{tar}}
\newcommand{\uu}{{\mathbf{u}}}
\newcommand{\UU}{{\mathbf{U}}}
\newcommand{\vv}{{\mathbf{v}}}
\newcommand{\xx}{{\mathbf{x}}}
\newcommand{\xxi}{{\boldsymbol{\xi}}}
\newcommand{\yy}{{\mathbf{y}}}

\def\gap{\hspace*{.2in}}

% Derivatives
\newcommand{\pderiv}[2]{\frac{\partial #1}{\partial #2}}
\newcommand{\tderiv}[2]{\frac{d #1}{d #2}}
\newcommand{\ppd}[2]{\frac{\partial^2 #1}{{\partial #2}^2}}

% Nick's commands
\newcommand{\vsp}[1]{\vspace{#1 pc} \noindent}
\newcommand{\abs}[1]{\lvert #1 \rvert}
\newcommand{\mean}[1]{\left< #1 \right>}
\newcommand{\thL}{$\theta$--$L$}
\newcommand{\eps}{\varepsilon}
\newcommand{\Vn}{V_n}
\newcommand{\Vs}{V_s}
\newcommand{\atau}{\abs{\tau}}
\newcommand{\thalpha}{\pderiv{\theta}{\alpha}}
\newcommand{\elfun}{\zeta}
\newcommand{\thhat}{\hat{\theta}}
\newcommand{\Dt}{\Delta t}
\newcommand{\NLterm}{\mathcal{N}}
\newcommand{\Mterm}{\mathcal{M}}
\newcommand{\FourierSum}{ \sum_{k = -N_\iin /2}^{N_\iin /2-1} }
\newcommand{\atausig}{\atau^{(\sigma)}}
\newcommand{\Vnsig}{\Vn^{(\sigma)}}
\newcommand{\Vssig}{\Vs^{(\sigma)}}


\newcommand{\tauD}[1]{\tau_{#1\text{D}}}
\newcommand{\atauD}[1]{\abs{\tau_{#1\text{D}}}}

\begin{document}

\title{Methods paper for erosion}

\author[Nick]{M.~Nicholas J.~Moore}
\author[Bryan]{Bryan D.~Quaife}
\address[Nick]{Department of Mathematics and Geophysical Fluid Dynamics Institute, Florida State University, Tallahassee, FL, 32306.}
\address[Bryan]{Department of Scientific Computing and Geophysical Fluid Dynamics Institute, Florida State University, Tallahassee, FL, 32306.}

\begin{abstract} 
We consider two dimensional eroding bodies in Stokes flow
\end{abstract}

\begin{keyword}
  Stokes flow \sep Erosion \sep Boundary integral method \sep
  Fluid-structure interaction \sep Fast multipole methods 
\end{keyword}

\maketitle

%%%%%%%%%%%%%%%%%%%%%%%%%%%%%%%%%%%%%%%%%%%%%%%%%%%%%%%%%%%%%%%%%%%%%%%
\section{Introduction\label{s:intro}}

This is a methods paper
\begin{itemize}
  \item Boundary integral equation formulation
  \item How we compute the shear stress and pressure
  \item Multiple bodies
  \item Avoiding stokes paradox
  \item $\theta-L$ formulation
  \item Regularization
\end{itemize}

\cite{moo-ris-chi-zha-she2013}



%%%%%%%%%%%%%%%%%%%%%%%%%%%%%%%%%%%%%%%%%%%%%%%%%%%%%%%%%%%%%%%%%%%%%%%
\section{Formulation\label{s:formulation}} 
We consider multiple bodies being eroded by a two-dimensional Stokesian
fluid.  The Stokes paradox is circumvented by using a rounded-off
rectangle that encloses the eroding bodies.  Then, the flow is driven
by a boundary conditions, such as a Hagen-Poiseuille flow, on the
boundary of the outer geometry \todo[inline]{Need a figure here}


%%%%%%%%%%%%%%%%%%%%%%%%%%%%%%%%%%%%%%%%%%%%%%%%%%%%%%%%%%%%%%%%%%%%%%%
\subsection{Erosion Model} 
\todo[inline]{Governing equations}


%%%%%%%%%%%%%%%%%%%%%%%%%%%%%%%%%%%%%%%%%%%%%%%%%%%%%%%%%%%%%%%%%%%%%%%
\subsection{Boundary Integral Equation Formulation} 
The incompressible Stokes equations are
\begin{alignat*}{3}
  \mu\Delta \uu &= \grad p, \qquad && \xx \in \Omega, \\
  \grad \cdot \uu &= 0,   && \xx \in \Gamma.
\end{alignat*}
The viscosity $\mu$ sets the time scale, and we will assume that
$\mu=1$.  The double-layer potential 
\begin{align}
  \DD[\eeta](\xx) = \frac{1}{\pi} \int_{\Gamma} 
    \frac{\rr \cdot \nn}{\rho^2} \frac{\rr \otimes \rr}{\rho^2} 
    \rr \, ds_\yy, \quad \xx \in \Omega,
    \label{eqn:dlp}
\end{align}
where $\rr = \xx - \yy$, $\rho = \|\rr\|$, and $\nn$ is the unit outward
normal.  The double-layer potential can not capture velocity fields
corresponding to rigid body motions~\cite{pow-mir1987}.  A standard
method to complete the double-layer potential is to introduce a
Stokeslet and rotlet due to each interior body
\begin{align}
  S[\llambda_k,\cc_k] = \frac{1}{4\pi} \left(-\log \rho \II + 
    \frac{\rr \otimes \rr}{\rho^2} \right)\llambda_k
  \quad \text{and} \quad
  R[\xi_k,\cc_k] = \frac{\rr^\perp}{\rho^2}\xi_k,
  \label{eqn:stokeslet_rotlet}
\end{align}
where $\cc_k$ is a point in the interior of $\Omega_k$, $\rr = \xx -
\cc_k$, and $\rr^\perp = (r_2,-r_1)$.  

Then, the solution of the Stokes equations with a Dirichlet boundary
condition $\ff$ can be written in terms of the completed double-layer
potential
\begin{align}
  \uu(\xx) = \DD[\eeta](\xx) + \sum_{k=1}^{M} S[\llambda_k,\cc_k](\xx) +
              \sum_{k=1}^{M} R[\xi_k,\cc_k](\xx), \quad \xx \in \Omega,
  \label{eqn:completedDLP}
\end{align}
where the density function, Stokeslets, and rotlets satisfy the
second-kind integral equation
\begin{align}
  \ff(\xx) &= -\frac{1}{2}\eeta(\xx) + \DD[\eeta](\xx) + 
      \sum_{k=1}^{M} S[\llambda_k,\cc_k](\xx) +
      \sum_{k=1}^{M} R[\xi_k,\cc_k](\xx), \quad \xx \in \Gamma, \\
  \llambda_k &= \frac{1}{2\pi}\int_{\Gamma_k} \eeta(\yy) \, ds_\yy, \\
  \xi_k &= \frac{1}{2\pi}\int_{\Gamma_k} \yy^\perp \cdot \eeta(\yy)
  \, ds_\yy.
  \label{eqn:completed_DLP}
\end{align}
Finally, because of the incompressibility constraint, the boundary
condition must satisfy 
\begin{align}
  \int_{\Gamma} \ff \cdot \nn \, ds_\yy = 0.
\end{align}
These results in a rank-one null space in
equation~\eqref{eqn:completed_DLP} which is removed by adding the rank
one modification $\NN_0[\eeta](\xx) = \int_{\Gamma_0} (\nn(\xx) \times
\nn(\yy))\nn(\yy) \, ds_\yy$, for $\xx \in \Gamma_0$.




%%%%%%%%%%%%%%%%%%%%%%%%%%%%%%%%%%%%%%%%%%%%%%%%%%%%%%%%%%%%%%%%%%%%%%%
\subsection{Computing the Shear Stress, Pressure, and Drag} 
The governing equations~\eqref{eqn:} require the shear stress which is
defined on the boundary of each pore.  We are also interested in
computing the drag on each pore which requires the pressure.

The shear stress is given by
\begin{align}
  \tau = \left(\nabla \uu + \nabla \uu^T \right)\nn \cdot \ss,
  \label{eqn:shearStress}
\end{align}
where $\ss$ is the unit tangent vector.  We note that for the Stokes
equations, because of the no-slip boundary condition on the pores, the
shear stress is identical to the surface vorticity $\tau = u_y - v_x$.
However, we use equation~\eqref{eqn:shearStress} to compute the shear
stress since the jump conditions resulting from the double-layer
potential are known~\cite{qua-bir2014a}.  In particular, we start by
first computing the deformation tensor $\ssigma = \frac{1}{2}(\nabla\uu
+ \nabla\uu^T)$ of the double-layer potential~\eqref{eqn:dlp} at $\xx
\in \Omega$
\begin{align*}
  \ssigma(\xx) = \int_{\bd\Omega} K(\xx,\yy) \eeta(\yy) \,ds_\yy,
\end{align*}
where the kernel $K$ is given by
\begin{align*}
  K(\xx,\yy) = \frac{1}{2\pi}\left(
    2\frac{\rr \cdot \nn}{\rho^2} \frac{\rr \cdot \eeta}{\rho^2} \II + 
    \frac{\rr \cdot \eeta}{\rho^4} (\nn \otimes \rr + \rr \otimes \nn) +
    \frac{\rr \cdot \nn}{\rho^4} (\eeta \otimes \rr + \rr \otimes \eeta) - 
    8\frac{(\rr \cdot \nn)(\rr \cdot \eeta)}{\rho^6}(\rr \otimes \rr)
  \right).
\end{align*}
To compute the shear stress at boundary points $\xx_0 \in \bd\Omega$, we
require the jump in the shear stress as the target point approaches the
boundary.  The limiting value is given by
\begin{align}
  \lim_{\substack{\xx \rightarrow \xx_0 \\ \xx \in \Omega}}\sigma(\xx) =
  J[\eeta](\xx_0) + \ssigma(\xx_0) = 
  \frac{1}{2}\pderiv{\eeta}{\ss} \cdot \ss
  \left[ \begin{array}{cc}
    s_x^2 - s_y^2 & 2s_x s_y \\ 2s_x s_y & s_y^2 - s_x^2
  \end{array}\right] + \sigma(\xx_0), \quad \xx_0 \in \Gamma.
\end{align}
Finally, adding the deformation tensor due to the Stokeslets and rotlets
in equation~\eqref{eqn:stokeslet_rotlet}, the deformation tensor of the
complete double-layer potential formulation~\eqref{eqn:completedDLP} is
\begin{align}
  \ssigma(\xx_0) = \frac{\rr \cdot \llambda}{4\pi\rho^2} \left(
    \II - \frac{2}{\rho^2} \rr \otimes \rr \right) -
  \frac{\xi}{\rho^4} \left(\rr \otimes \rr^\perp + 
                           \rr^\perp \otimes \rr \right)
\end{align}
\todo[inline]{Put together the whole stress calculation correctly}







%%%%%%%%%%%%%%%%%%%%%%%%%%%%%%%%%%%%%%%%%%%%%%%%%%%%%%%%%%%%%%%%%%%%%%%
\section{Numerical Methods\label{s:method}} 


%%%%%%%%%%%%%%%%%%%%%%%%%%%%%%%%%%%%%%%%%%%%%%%%%%%%%%%%%%%%%%%%%%%%%%%
\subsection{Quadrature} 
There are several integrals that we need to compute including the
double-layer potential velocity field~\eqref{eqn:dlp}, the shear
stress~\eqref{}, and the pressure~\eqref{}.  By using carefully choosing
the quadrature rules, high-order solutions can be computed.  For all the
integrals, the trapezoid rule is used since it achieves spectral
accuracy for smooth periodic functions.  However, care must be taken
when the source point and target points coincide.

For the 



%%%%%%%%%%%%%%%%%%%%%%%%%%%%%%%%%%%%%%%%%%%%%%%%%%%%%%%%%%%%%%%%%%%%%%%
\subsection{Time Stepping} 
We use a quasi-static formulation ....

%%%%%%%%%%%%%%%%%%%%%%%%%%%%%%%%%%%%%%%%%%%%%%%%%%%%%%%%%%%%%%%%%%%%%%%
\subsection{$\theta-L$ Formulation} 


%%%%%%%%%%%%%%%%%%%%%%%%%%%%%%%%%%%%%%%%%%%%%%%%%%%%%%%%%%%%%%%%%%%%%%%
\subsection{Fast Summation} 

%%%%%%%%%%%%%%%%%%%%%%%%%%%%%%%%%%%%%%%%%%%%%%%%%%%%%%%%%%%%%%%%%%%%%%%
\subsection{Artificial Diffusion} 

%%%%%%%%%%%%%%%%%%%%%%%%%%%%%%%%%%%%%%%%%%%%%%%%%%%%%%%%%%%%%%%%%%%%%%%
\subsection{Gaussian Filter} 

%%%%%%%%%%%%%%%%%%%%%%%%%%%%%%%%%%%%%%%%%%%%%%%%%%%%%%%%%%%%%%%%%%%%%%%
\section{Results\label{s:results}} 

%%%%%%%%%%%%%%%%%%%%%%%%%%%%%%%%%%%%%%%%%%%%%%%%%%%%%%%%%%%%%%%%%%%%%%%
\section{Conclusions\label{s:conclusions}}


%%%%%%%%%%%%%%%%%%%%%%%%%%%%%%%%%%%%%%%%%%%%%%%%%%%%%%%%%%%%%%%%%%%%%%%
\begin{appendices}
\section{Error estimates for near-singular integration \label{A:AppendixA}} 
\end{appendices}


\bibliographystyle{plainnat} 
\bibliography{refs}
\biboptions{sort&compress}
\end{document}


