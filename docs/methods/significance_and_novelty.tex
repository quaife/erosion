\documentclass[11pt]{article}

\begin{document}
\begin{center}
  {\bf Significance and Novelty} \\
  A boundary-integral framework to simulate \\ viscous erosion of a
  porous medium \\
  Bryan D.~Quaife and M.~Nicholas J.~Moore
\end{center}
In this paper we describe a numerical method to simulate viscous erosion
of a two-dimensional porous medium with a long-term goal of simulating
groundwater erosion.  Based on previous experimental results, the
erosion rate is assumed to be proportional to the magnitude of the shear
stress.  To increase the numerical stability, this law is supplemented
with a regularization term that penalizes high curvature regions.  We
avert stiffness with a change of variables that results in a system of
differential equations with a stiff linear, and these are solved with a
second-order exponential time stepping method.  

To compute the shear stress, we assume the fluid is viscous---a
standard model for groundwater flow.  Then, the governing equations
(incompressible Stokes) are formulated as a well-conditioned boundary
integral equation formulation.  The integral equation resolves
inter-body interactions and is discretized with spectral accuracy and
solved with fast-multipole-accelerated GMRES.  Then, to further
characterize the flow, we compute the drag of the bodies and the
vorticity in the bulk which is used to characterize the erosion rate.

The numerical simulation of a single eroding body is compared with
analytical solutions.  We show good agreement between the eroding body's
numerical and analytical opening angle, area, and drag.  Then, erosion
is simulated for multi-body problems. We observe channelization of the
flow caused by long flat faces that develop between bodies that are
initially close together.  This has a large effect on the time-dependent
drag and resistance of the flow.




\end{document}
