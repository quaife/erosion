\documentclass[11pt]{article}

\begin{document}
\begin{center}
  {\bf Significance and Novelty} \\
  A boundary-integral framework to simulate \\ viscous erosion of a
  porous medium \\
  Bryan D.~Quaife and M.~Nicholas J.~Moore
\end{center}
In this paper we describe a numerical method to simulate viscous erosion
of a two-dimensional porous medium.  This is a step towards a long-term
goal of developing stable, accurate, and fast numerical methods to
simulate and understand groundwater erosion.

Based on previous experimental results, the erosion rate is assumed to
be proportional to the magnitude of the shear stress.  To increase the
numerical stability, this law is supplemented with a regularization term
that penalizes high curvature regions and smooths the erosion rate at
regions where the shear stress changes sign.  We use a change of
variables to linearize stiffness, and the resulting system of
differential equations is solved with a second-order exponential time
stepping method.

To compute the shear stress, we assume the fluid is viscous---a standard
model for groundwater flow.  The governing equations (incompressible
Stokes) are formulated as a well-conditioned boundary integral equation.
The integral equation discretion resolves the non-negligible inter-body
interactions with spectral accuracy and is solved with
fast-multipole-accelerated GMRES.  To further characterize the flow, we
compute the drag of the bodies and the vorticity in the bulk which is
used to characterize the erosion rate.

The numerical simulation of a single eroding body is compared with
analytical solutions.  We show good agreement between the eroding body's
numerical and analytical opening angle, area, and drag.  Then, erosion
is simulated for multi-body problems. We observe channelization of the
flow caused by long flat faces that develop between bodies that are
initially close together.  This has a large effect on the time-dependent
drag and resistance of the flow.




\end{document}
